A few example questions that might show up in the exam are:
\begin{itemize}
    \item Reused homework tasks from the Moodle Quizzes and Code Expert (e.g. the GPS task)
    \item Reused tasks from the mock exam
    \item Doing quadrature on an $\arctan$ function, for which you have to use Gauss-Newton quadrature
\end{itemize}
Knowing \texttt{sympy} and having read through the docs of it (at least partially) can help tremendously.
Additionally, try to use the script with a horrible (i.e. slow and not context-aware (i.e. it start search from the start)) PDF reader to get used to that.
You can get such a search function in a PDF reader by installing Firefox ESR 72 (on which to my knowledge, Safe Exam Browser is based).

For the multiple-choice tasks, if applicable, use properties of the formulas
(e.g. for the Radau quadrature, use the fact that one of the edges of the interval has to be included and the weight computation is different for it)
or otherwise use the Jupyter notebook to run functions that are already implemented in the script, or better yet, \texttt{sympy}.
