\subsubsection{Normalengleichung}
\setLabelNumber{all}{9}
\fancydef{Normalengleichung} $A^H Ax = A^H b$

\inlineremark $A^H A$ ist Hermite-Symmetrisch, und falls $A$ vollen Rank hat, dannn ist $A^H A$ positiv-definit und die Normalengleichung hat eine eindeutige Lösung.
Jedoch ist die Normalengleichung schlecht konditioniert (es gilt: $\cond(A^H A) = \cond(A)^2$).
Für gut konditionierte Matrizen ist dies kein Problem, jedoch ist die Normalengleichung für schlecht konditionierte Matrizen ungeeignet.


\inlineremark Man kann die Normalengleichung auch ohne die Berechnung von $A^H A$ berechnen:
\rmvspace
\begin{align*}
    A^H Ax = A^H b
    \Longleftrightarrow B
    \begin{bmatrix}
        r \\ x
    \end{bmatrix}
    :=
    \begin{bmatrix}
        -I  & A \\
        A^H & 0
    \end{bmatrix}
    \begin{bmatrix}
        r \\ x
    \end{bmatrix}
    =
    \begin{bmatrix}
        b \\ 0
    \end{bmatrix}
\end{align*}

\drmvspace
für $r := \frac{1}{a} (Ax - b)$ mit $a > 0$, dann können wir $B$ in obiger Gleichung durch
$B_a =
    \begin{bmatrix}
        -aI & A \\
        A^H & 0
    \end{bmatrix}
$ 
ersetzen, wobei wir $a$ so wählen, dass $\kappa(B_a)$ minimal wird (Zur Erinnerung, $\kappa$ ist die Konditionszahl der Matrix).
