\subsubsection{Totale Ausgleichsrechnung}
Es kann vorkommen, dass sowohl die Matrix $A$, wie auch der Vektor $b$ fehlerhaft sind.
Dann ersetzen wir das System $Ax = b$ durch ein neues System $\hat{A}\hat{x} = \hat{b}$,
welches so nah wie möglich am ursprünglichen System liegt und so für welches gilt $\hat{b} \in \text{Bild}(\hat{A})$.

Wir versuchen also die folgende Norm zu minimieren:
\begin{align*}
    ||C - \hat{C}||_F
    =
    \left|\left|
    \begin{bmatrix}
        A & b
    \end{bmatrix}
    -
    \begin{bmatrix}
        \hat{A} & \hat{b}
    \end{bmatrix}
    \right|\right|_F
\end{align*}

\drmvspace
Das Problem lässt sich umschreiben als
\rmvspace
\begin{align*}
    \min_{\text{Rang}(\hat{C}) = n} ||C - \hat{C}||_F
\end{align*}

\drmvspace
Theorem \ref{all:7-1-50} liefert die Lösung. Die Singulärwertzerlegung
\rmvspace
\begin{align*}
    C = U\Sigma V^H = \sum_{j = 1}^{n + 1} \sigma_j (u)_j (v)_j^H
\end{align*}

\drmvspace
gibt das Optimum
\rmvspace
\begin{align*}
    \hat{C} = \sum_{j = 1}^{n} \sigma_j (u)_j (v)_j^H
\end{align*}
