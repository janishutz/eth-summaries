\subsubsection{Gauss-Newton Verfahren}
Direkt das Newton-Verfahren auf ein Problem anzuwenden kann unmöglich oder schwer praktikabel sein.

Die Idee des Gauss-Newton Verfahrens ist es, die komplizierte Funktion $F(x)$ lokal durch eine lineare Funktion approximiert, also:
\begin{align*}
    F(x) \approx F(y) + DF(y) (x - y) = F(y) + DF(y)x - DF(y)y
\end{align*}
Falls man $A := DF(y)$ und $b = DF(y)y - F(y)$ definiert, so erhält man ein lineares Ausgleichsproblem:
\rmvspace
\begin{align*}
    \argmin{x \in \R^n} \frac{1}{2} ||F(x)||^2_2 \approx \argmin{x \in \R^n} \frac{1}{2} ||F(y) + DF(y) x||^2_2 = \argmin{x \in \R^n} \frac{1}{2} ||Ax - b||^2_2
\end{align*}

\drmvspace
wobei $y$ eine Näherung der Lösung $x$ ist.
Die Iterationsvorschrift ist gegeben durch:
\rmvspace
\begin{align*}
    x^{(k + 1)} = x^{(k)} - s \smallhspace \text{ mit } s := \argmin{z \in \R^n} ||F(x^{(k)}) - DF(x^{(k)})z||^2_2
\end{align*}
