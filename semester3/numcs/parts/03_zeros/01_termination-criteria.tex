% ┌                                                ┐
% │     AUTHOR: Janis Hutz<info@janishutz.com>     │
% └                                                ┘

\newsection
\subsection{Abbruchkriterien}
Wir müssen irgendwann unsere Iteration abbrechen können, dazu haben wir folgende Möglichkeiten:
\begin{fullTable}{p{2.5cm}p{5.5cm}p{3cm}p{4.5cm}}{Typ & Idee                                                                                    & Vorteile                  & Nachteile}{Vergleich der Abbruchkriterien}
    \bi{A priori}                        & Fixe Anzahl $k_0$ Schritte                                                              & Einfach zu implementieren & Zu ungenau                                                                        \\
    \bi{A posteriori}                    & Berechnen bis Toleranz $\varepsilon < \tau$ erreicht                                    & Präzise                   & Man kennt $x^*$ nicht                                                             \\
    \bi{Ungefähr gleich}                 & Itaration bis$x^{(k + 1)} \approx x^{(k)}$                                              & Keine Voraussetzungen     & Ineffizient                                                                       \\
    \bi{Residuum}                        & Abbruch wenn $||F(x^{(k)})|| < \tau$ (wir also fast bei $0$ sind mit dem Funktionswert) & Einfach zu implementieren & Bei flachen Funktionen kann $||F(x^{(k)}||$ klein sein, aber $\varepsilon$ gross) \\
\end{fullTable}

\drmvspace
\inlineremark Für das \textit{a posteriori} Abbruchkriterium mit linearer Konvergenz und bekanntem $L$ gilt die Abschätzung aus Lemma \ref{all:6-3-6} mit Korollar \ref{all:6-3-17}
