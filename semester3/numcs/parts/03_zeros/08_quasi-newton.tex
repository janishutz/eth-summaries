\newsectionNoPB
\subsection{Quasi-Newton-Verfahren}
Falls $DF(x)$ zu teuer ist oder nicht zur Verfügung steht, können wir im Eindimensionalen das Sekantenverfahren verwenden.

Im höherdimensionalen Raum ist dies jedoch nicht direkt möglich und wir erhalten die Broyden-Quasi-Newton Methode:
\rmvspace
\begin{align*}
    J_{k + 1} := J_k + \frac{F(x^{(k + 1)}) (\Delta x^{(k)})^\top}{||\Delta x^{(k)}||_2^2}
\end{align*}

\drmvspace
Dabei ist $J_0$ z.B. durch $DF(x^{(0)})$ definiert.
% Page 222
