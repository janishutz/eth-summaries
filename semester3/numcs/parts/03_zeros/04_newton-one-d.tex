\newsectionNoPB
\subsection{Newtonverfahren in 1D}
Beim Newtonverfahren verwendet man für jeden Iterationsschritt die lineare Funktion $\tilde{F} = F(x^(k)) + F'(x^{(k)})(x - x^{(k)})$. Die Nullstelle ist dann:
\rmvspace
\begin{align*}
    x^{(k + 1)} := x^{(k)} - \frac{F(x^{(k)})}{F'(x^{(k)})}, \mediumhspace \text{falls } F'(x^{(k)}) \neq 0
\end{align*}

\stepLabelNumber{all}
\inlineremark Die Newton-Iteration ist eine Fixpunktiteration mit quadratischer lokaler Konvergenz, mit
\rmvspace
\begin{align*}
    \phi(x) = x - \frac{F(x)}{F'(x)} \Longrightarrow \phi'(x) = \frac{F(x) F''(x)}{(F'(x))^2} \Longrightarrow \phi'(x^*) = 0
\end{align*}

\drmvspace
falls $F(x^*) = 0$ und $F^(x^*) \neq 0$

\newpage

\innumpy Ist das Newton-Verfahren mit sehr wenig code implementierbar:

\begin{code}{python}
def newton_method(f, df, x: float, tol=1e-12, maxIter=100):
    """ Use Newton's method to find zeros, requires derivative of f """
    k = 0
    fx = f(x)
    dfx = df(x)
    while (np.abs(fx) > tol and k < maxIter):
        x -= fx / dfx
        k += 1
    
    return x, k
\end{code}
