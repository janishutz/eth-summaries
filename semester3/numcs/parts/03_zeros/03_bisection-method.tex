\newsection
\subsection{Intervallhalbierungsverfahren}
Die Idee hier ist, das Intervall immer weiter zu halbieren und ein bekannterer Namen für dieses Verfahren ist \bi{Bisektionsverfahren}.

\innumpy haben wir \texttt{scipy.optimize.bisect} und \texttt{scipy.optimize.fsolve}, wobei \texttt{fsolve} ein alter Algorithmus ist.

Im Skript auf Seiten 206 - 207 findet sich eine manuelle implementation des Bisektionsverfahren.
Der Code ist jedoch (at the time of writing) nicht ausführbar aufgrund von \texttt{IndentationErrors}

Das Bisektionsverfahren konvergiert linear und kann nur für Funktionen verwenden, bei welchen die Nullstellen auf beiden Seiten jeweils ungleiche Vorzeichen haben.

% TODO: Need to add the formula from SPAM script
