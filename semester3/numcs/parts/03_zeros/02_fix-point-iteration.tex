% ┌                                                ┐
% │     AUTHOR: Janis Hutz<info@janishutz.com>     │
% └                                                ┘

\rmvspace\newsectionNoPB
\subsection{Fixpunktiteration}
Ein $1$-Punkt-Verfahren benötigt nur den vorigen Wert: $x^{(k + 1)} = \phi(x^{(k)})$

\inlinedef Eine Fixpunktiteration heisst konsistent mit $F(x) = 0$ falls $F(x) = 0 \Leftrightarrow \phi(x) = x$

\inlineex Für $F(x) = xe^x - 1$ mit $x \in [0, 1]$ liefert $\phi_1(x) = e^{-x}$ lineare Konvergenz,
$\phi_2(x) = \frac{1 + x}{1 + e^x}$ quadratische Konvergenz und $\phi_3(x) = x + 1 - xe^x$ eine divergente Folge.

\setLabelNumber{all}{5}
\fancydef{Kontraktion} $\phi$ falls es ein $L < 1$ gibt, so dass $||\phi(x) - \phi(y)|| \leq L||x - y|| \ \forall x, y$

\inlineremark Falls $x^*$ ein Fixpunkt der Kontraktion $\phi$ ist, dann ist
\drmvspace
\begin{align*}
    ||x^{(k + 1)} - x^*|| = ||\phi(x^{(k)}) - \phi(x^*)|| \leq L||x^{(k)} - x^*||
\end{align*}

\drmvspace
\begin{theorem}[]{Banach'scher Fixpunktsatz}
    Sei $D \subseteq \K^n$ ($\K = \R, \C$) mit $D$ abgeschlossen und $\phi: D \rightarrow D$ eine Kontraktion.
    Dann existiert ein eindeutiger Fixpunkt $x^*$, für welchen also gilt, dass $\phi(x^*) = x^*$.
    Dieser ist der Grenzwert der Folge $x^{(k + 1)} = \phi(x^{(k)})$.
\end{theorem}

\inlinelemma Für $U \subseteq \R^n$ konvex und $\phi : U \rightarrow \R^n$ stetig differenzierbar mit $L := \sup_{x \in U} ||D_\phi(x)|| < 1$
($D_\phi(x)$ ist die Jacobi-Matrix von $\phi(x)$).
Wenn $\phi(x^*) = x^*$ für $x^* \in U$, dann konvergiert $x^{(k + 1)} = \phi(x^{(k)})$ gegen $x^*$ lokal mindestens linear.
Dies ist eine hinreichende (= sufficient) Bedingung.

\setLabelNumber{all}{11}
\inlinelemma Für $\phi : \R^n \rightarrow \R^n$ mit $\phi(x^*) = x^*$ und $\phi$ stetig differenzierbar in $x^*$.
Ist $||D_\phi(x^*)|| < 1$, dann konvergiert $x^{(k + 1)} = \phi(x^{(k)})$ lokal und mindestens linear mit $L = ||D_\phi(x^*)||$

\stepLabelNumber{all}
\fancytheorem{Satz von Taylor} Sei $I \subseteq \R$ ein Intervall, $\phi : I \rightarrow \R$ $(m + 1)$-mal differenzierbar und $x \in I$.
Dann gilt für jedes $y \in I$
\drmvspace
\begin{align*}
    \phi(y) - \phi(x) = \sum_{k = 1}^{m} \frac{1}{k!} \left( \phi^{(k)}(x) (y - x)^k \right) + \tco{|y - x|^{m + 1}}
\end{align*}

\drmvspace
\inlinelemma Sei $I$ und $\phi$ wie in Satz \ref{all:6-3-13}. Sei zudem $\phi^{(l)}(x^*) = 0$ für $l \in \{ 1, \ldots, m \}$ mit $m \geq 1$.
Dann konvergiert $x^{(k + 1)} = \phi(x^{(k)})$ lokal gegen $x^*$ mit Ordnung $p \geq m + 1$

\stepLabelNumber{all}
\inlinelemma Konvergiert $\phi$ linear mit $L < 1$, dann gilt:
\drmvspace
\begin{align*}
    ||x^{*} - x^{(k)}|| \leq \frac{L^{k - l}}{1 - L} ||x^{(l + 1)} - x^{(l)}||
\end{align*}

\drmvspace
\inlinecorollary für $l = 0$ haben wir ein \textit{a priori} und für $l = k - 1$ ein \textit{a posteriori} Abbruchkriterium:
\drmvspace
\begin{align*}
    ||x^* - x^{(k)}|| \leq \frac{L^k}{1 - L} ||x^{(1)} - x^{(0)}|| \leq \tau & & ||x^* - x^{(k)}|| \leq \frac{L}{1 - L} ||x^{(k)} - x^{(k - 1)}|| \leq \tau
\end{align*}
