\newsectionNoPB
\subsection{Splines}
\begin{definition}[]{Raum der Splines}
    Sei $[a, b] \subseteq \R$ ein Intervall, sei $\mathcal{G} = \{ a = x_0 < x_1 < \ldots < x_N = b \}$ und sei $d \geq 1 \in \N$.
    Die Menge
    % FIXME: What's up with the |[x_{j - 1}, x_j] Notation? Isn't there a | missing? And what does it even mean?
    \begin{align*}
        \mathcal{S}_{d, \mathcal{G}} = \{ s \in C^{d - 1}[a, b], \smallhspace s_j := s_{|[x_{j - 1}, x_j]|} \text{ ist ein polynom von Grad höchstens } d \}
    \end{align*}
    ist die Menge aller auf $[a, b]$ $(d - 1)$ mal stetig ableitbaren Funktionen, die auf $\mathcal{G}$ aus stückweisen Polynomen von Grad höchtens $d$ bestehen
    und wir der Raum der Splines vom Grad $d$, oder der Ordnung $(d + 1)$ genannt
\end{definition}

\inlineremark Obige Definition ist undefiniert für $d = 0$, aber $\mathcal{S}_{d, \mathcal{G}}$ kann als die Menge der stückweise Konstanten Funktionen betrachtet werden.
Im Vergleich zu den Kubischen Hermite-Interpolanten sind die Kubischen-Splines (für $d = 3$) \textit{zweimal} Ableitbar statt nur \textit{einmal}

\inlineremark $\dim(\mathcal{S}_{d, \mathcal{G}}) = N + d$. Es werden oft kubische Splines in Anwendungen verwendet, also ist\\
$\dim(\mathcal{S}_{d, \mathcal{G}}) = N + 3$, wir haben aber nur $N + 1$ Funktionswerte, also beleiben noch zwei Freiheitsgrade übrig.

Dies bedeutet, dass wir ein underdeterminiertes lineares Gleichungssystem haben für $h_j = x_j - x_{j - 1}$:
\rmvspace
\begin{align*}
    \begin{bmatrix}
        b_0    & a_1   & b_1    & 0      & \dots  &           & \dots     & 0         \\
        0      & b_1   & a_2    & b_2    &        &           &           &           \\
               & 0     & \ddots & \ddots & \ddots &           &           & \vdots    \\
        \vdots &       &        & \ddots & \ddots & \ddots    &           &           \\
               &       &        &        & \ddots & a_{N - 2} & b_{N - 2} & 0         \\
        0      & \dots &        & \dots  & 0      & b_{N - 2} & a_{N - 1} & b_{N - 1}
    \end{bmatrix}
    \begin{bmatrix}
        c_0 \\ c_1\\ \\[0.2cm] \vdots \\ \\[0.2cm] c_{N - 1} \\ c_N
    \end{bmatrix}
    =
    \begin{bmatrix}
        3 \left( \frac{y_1 - y_0}{h_1^2} + \frac{y_2 - y_1}{h_2^2} \right)                         \\
        \\ \vdots \\ \\
        3 \left( \frac{y_{N - 1} - y_{N - 2}}{h_{N - 1}^2} + \frac{y_N - y_{N - 1}}{h_N^2} \right) \\
    \end{bmatrix}
\end{align*}

\rmvspace
Wobei im Resultatvektor Einträge der Form
\rmvspace
\begin{align*}
    3 \left( \frac{y_j - y_{j - 1}}{h_j^2} + \frac{y_{j + 1} - y_j}{h^2_{j + 1}} \right)
\end{align*}
enthalten sind und mit $a_j := \frac{2}{h_j} + \frac{2}{h_{j + 1}}$ und $b_j := \frac{1}{h_{j + 1}}$ für $j = 0, 1, \ldots, N - 1$

Wir müssen also zwei weitere Gleichungen finden (oder zwei Freiheitsgrade eliminieren).

\fancydef{Vollständige kubische Spline-Interpolation} Falls wir die zusätzlichen Bedingungen $s'(x_0) = c_0$ und $s'(x_N) = c_N$ mit gegebenen $c_0$ und $c_N$ haben.
Sie ist auch bekannt als \textit{clamped cubic spline}.
In der obigen Matrix können dann die erste und letzte Spalte weggelassen werden.

\fancydef{Natürliche kubische Spline-Interpolation} Falls wir die zusätzlichen Bedingungen $s''(x_0) = 0$ und $s''(x_N) = 0$ haben.
Dann fügen wir obigem SLE zwei Zeilen hinzu (1. und $(N + 1)$-te), die $2, 1, 0, 0, \ldots = \frac{y_1 - y_0}{h_1}$ und $0, \ldots, 0, 1, 2 = \frac{y_N - y_{N - 1}}{h_N}$.
Die Matrix ist nun also positive-definite und symmetrisch

\vspace{-1.25pc}
{
    \begin{wrapfigure}[6]{l}{0.5\textwidth}
        \fancydef{Periodische kubische Spline-Interpolation} Falls wir die zusätzlichen Bedingungen $s'(x_0) = s'(x_N)$ und $s''(x_0) = s''(x_N)$ haben.
        Dies macht nur Sinn, wenn $y_0 = y_N$, also nehmen wir das an und wir haben eine Spalte weniger und eine Reihe mehr, also ist die Systemmatrix rechts
    \end{wrapfigure}
    \begin{align*}
        A :=
        \begin{bmatrix}
            a_1 & b_1    & 0      & \dots  & 0         & b_0       \\
            b_1 & a_2    & b_2    &        &           & 0         \\
            0   & \ddots & \ddots & \ddots &           & \vdots    \\
            0   &        &        & \ddots & a_{N - 1} & b_{N - 1} \\
            b_0 & 0      & \dots  & 0      & b_{N - 1} & a_0
        \end{bmatrix}
    \end{align*}
}

\inlineremark Die SLE können in $\tco{n}$ gelöst werden.

\inlineremark Mit der ``not-a-knot''-Bedingung $s'''$ ist stetig in $x_1$ und $x_{N - 1}$ braucht man mindestens $4$ Knoten.
Da wir kubische Splines betrachten erzwing die Bedingung dass ein Polynom nur in den ersten beiden und ein anderes in den letzten beiden Subintervallen erscheint,
also gilt $s_1 = s_2$ und $s_{N - 1} = s_N$

\inlineremark Der natürliche Spline minimiert die Gesamtkrümmung des Funktionsgraphen:
\begin{align*}
    \int_{a}^{b} |s''(x)|^2 \dx x \leq \int_{a}^{b} |g''(x)|^2 \dx x
\end{align*}
für alle Funktionen zweimal stetig differenzierbaren Funktionen $g$, für welche $g(x_j) = y_j$ gilt für jedes $j = 0, \ldots, N$

\begin{theorem}[]{Interpolationsfehler vollständiger kubischer Splines}
    Wenn $f \in C^4[a, b]$ und $s$ der vollständige kubische Spline-Interpolation von $f$ auf einem äquidistantem Gitter mit Gitterweite $h$ ist, 
    dann ist der Fehler für $k = 0, 1, 2, 3$:
    \begin{align*}
        ||f^{(k)} - s^{(k)}||_{L^\infty} \leq \frac{5}{384} h^{4 - k} ||f^{(4)}||_{L^\infty}
    \end{align*}
\end{theorem}

\innumpy verwendet \texttt{scipy.interpolate.CubicSpline} aktuell die ``not-a-knot''-Bedingung. 
Es ist möglich mithilfe von \texttt{bc\_type} beim Instanziieren der Klasse die Art des Splines zu ändern.
Folgende (relevante) Optionen stehen laut 
\color{NavyBlue}\href{https://docs.scipy.org/doc/scipy/reference/generated/scipy.interpolate.CubicSpline.html}{Dokumentation}\color{black}\smallhspace
zur Verfügung: \texttt{"not-a-knot"} (was der Default ist), \texttt{"periodic"}, \texttt{"clamped"} und \texttt{"natural"}

Auf Seite 114-115 im Skript finden sich einige Abbildungen zur Konvergenz der verschiedenen Varianten des \texttt{CubicSplices}
