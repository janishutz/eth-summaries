% ┌                                                ┐
% │              Author: Robin Bacher              │
% └                                                ┘
\subsection{Interpolation und Polynome}
Bei der Interpolation versuchen wir eine Funktion $\tilde{f}$ durch eine Menge an Datenpunkten einer Funktion $f$ zu finden.\\
Die $x_i$ heissen Stützstellen/Knoten, für welche $\tilde{f}(x_i) = y_i$ gelten soll. (Interpolationsbedingung)
\begin{align*}
	\begin{bmatrix}
		x_0 & x_1 & \ldots & x_n \\
		y_0 & y_1 & \ldots & y_n
	\end{bmatrix},
	\quad x_i, y_i \in \mathbb{R}
\end{align*}

Normalerweise stellt $f$ eine echte Messung dar, d.h. es macht Sinn anzunehmen dass $f$ glatt ist.

Die informelle Problemstellung oben lässt sich durch Vektorräume formalisieren:

$f \in \mathcal{V}$, wobei $\mathcal{V}$ ein Vektorraum mit $\dim(\mathcal{V}) = \infty$ ist.

Wir suchen also $\tilde{f}$ in einem Unterraum $\mathcal{V}_n$ mit endlicher $\dim(\mathcal{V}_n) = n$.
Sei $B_n = \{b_1,\ldots,b_n\}$ eine Basis für $\mathcal{V}_n$.
Dann lässt sich der Bezug zwischen $f$ und $\tilde{f} = f_n(x)$ so ausdrücken:
\begin{align*}
	f(x) \approx f_n(x) = \sum_{j=1}^n \alpha_j b_j(x)
\end{align*}

\setLabelNumber{all}{1}
\inlineremark Unterräume $\mathcal{V}_n$ existieren nicht nur für Polynome, wir beschränken uns aber auf $b_j(x) = x^{i-1}$.
Andere Möglichkeiten: $b_j = \cos((j-1)\cos^{-1}(x))$ \textit{(Chebyshev)} oder $b_j = e^{i2\pi j x}$ \textit{(Trigonometrisch)}

\fancytheorem{Peano} $f$ stetig $\implies \exists p(x)$ welches $f$ in $||\cdot||_\infty$ beliebig gut approximiert.

\setLabelNumber{all}{5}
\fancydef{Raum der Polynome} $\mathcal{P}_k := \{ x \mapsto \sum_{j = 0}^{k} \alpha_j x^j \}$
\fancydef{Monom} $f: x \mapsto x^k$

\fancytheorem{Eigenschaft von $\mathcal{P}_k$} $\mathcal{P}_k$ ist ein Vektorraum mit $\dim(\mathcal{P}_k) = k+1$.
