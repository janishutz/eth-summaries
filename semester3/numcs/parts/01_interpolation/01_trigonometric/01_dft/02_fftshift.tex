\subsection{DFT in Numpy}

Sei $y$ in der Standardbasis, und $c = \mathcal{F}_N(y)$, also $y$ in der trig. Basis.
$$
    c = F_N \times y = \verb|fft|(y)\quad \textit{(DFT in numpy)} \quad \quad \quad y = \frac{1}{N}F_N^Hc = \verb|ifft|(c)\quad \textit{(Inverse DFT in numpy)}
$$

Um zur ursprünglichen Darstellung des trig. Polynoms zurück zu kommen, müssen wir die Koeffizienten umsortieren: \\
Seien $z = \frac{1}{N} F_N y$ und  $\zeta = \verb|fft.fftshift|(z)$. 
\begin{align*}
    f(x) \approx \underbrace{\sum_{k=-N/2}^{N/2-1} \zeta_k \cdot e^{2 \pi ikx} }_{\text{Form des trig. Polynoms}}
\end{align*}

\setcounter{all}{13}
\inlineremark Man kann mit dieser Approximation einfach die $L^2$-Norm und Ableitungen berechnen:
\begin{multicols}{2}
\begin{align*}
    ||f||^2_{L^2} \approx \left\Vert \sum_{k=-N/2}^{N/2-1} \zeta_k \cdot e^{2 \pi ikx} \right\Vert^2_{L^2} = \sum_{k=-N/2}^{N/2-1} |\zeta_k|^2 = \Vert z \Vert^2_{L^2}
\end{align*}

\newcolumn

\begin{align*}
    f'(t) \approx \sum_{k=-N/2}^{N/2-1} (2\pi ik) \zeta_k \cdot e^{2 \pi ikx} 
\end{align*}
\end{multicols}

Im Skript S. 78 - 83 befinden sich einige sehr gute Anwendungsbeispiele.