\subsection{Diskrete Fourier Transformation}

% NOTE: I'll do these 2 subchapters. Based on the lecture, we can leave out quite a lot here.
% Lecture: 3.2.1 eigentlich nur Endergebnis wichtig. 3.2.2 viel anschaulicher und theo. Grundlage für Anwendung. 3.2.3 zeigt kurz code.
% 1/sqrt(N) ist numerisch sehr schwer, d.h. wenn möglich nie nutzen. (d.h. ist bem 3.2.8 genau so definiert)
% Viele gute Bsps in 2.3.3, würde ich aber nicht hier übernehmen 
% NOTE: script p.74 sum transformation has errors. he said he'll fix.

% 2.3.4 relativ wichtig (Einführung Faltungen), tendenziell viel
% 2.4 FFT wichtig, aber kurz
