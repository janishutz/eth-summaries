\newsection
\subsection{Diskrete Fourier Transformation}

% NOTE: I'll do these 2 subchapters. Based on the lecture, we can leave out quite a lot here.
% Lecture: 3.2.1 eigentlich nur Endergebnis wichtig. 3.2.2 viel anschaulicher und theo. Grundlage für Anwendung. 3.2.3 zeigt kurz code.
% 1/sqrt(N) ist numerisch sehr schwer, d.h. wenn möglich nie nutzen. (d.h. ist bem 3.2.8 genau so definiert)
% Viele gute Bsps in 2.3.3, würde ich aber nicht hier übernehmen 
% NOTE: script p.74 sum transformation has errors. he said he'll fix.

% 2.3.4 relativ wichtig (Einführung Faltungen), tendenziell viel
% 2.4 FFT wichtig, aber kurz

\subsubsection{Motivation}

Nutzen wir die Trapezregel um approximativ die Fourierkoeffizienten $\hat{f}_N(k)$ auf äquidistanten Punkten $l_t=\frac{l}{N}$ $(0 \leq l \leq N-1)$ zu bestimmen, erhalten wir tatsächlich ein Polynom $p_{N-1}$ welches die Interpolationsbedingung erfüllt:
\begin{align*}
    p_{N-1}(t) = \sum_{k=-\frac{N}{2}}^{\frac{N}{2}-1} \hat{f}_N(k)e^{2\pi ikt}
\end{align*}

Der Beweis hierfür ist im Skript auf p. $71$. Die $N$-te Einheitswurzel wird hier definiert:

\fancydef{$N$-te Einheitswurzel} $\omega_N := \exp(\frac{-2\pi i}{N})$

\fancyremark{Eigenschaften von $\omega_N$}
\vspace{-1.5pc}
\begin{multicols}{3}
\begin{align*}
    \forall j,k \in \mathbb{Z}:\quad                 & \omega_N^{k+jN}=\omega_N^k \\
    \forall k \in \mathbb{Z}, t \in \mathbb{R}:\quad & \omega_N^{t+kN}=\omega_N^t 
\end{align*}

\newcolumn
\begin{align*}
    \omega_N^N=1 \\
    \omega_N^{N/2}=-1
\end{align*}

\newcolumn
\begin{align*}
    \sum_{k=0}^{N-1} \omega_N^{kj} = \begin{cases}
        N, & j \equiv_N 0 \\
        0, & \text{ sonst}
    \end{cases}
\end{align*}
\newcolumn
\end{multicols}
\vspace{-1.5pc}
