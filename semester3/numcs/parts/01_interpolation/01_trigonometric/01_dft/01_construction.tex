\subsubsection{Konstruktion}

Wir definieren die Trigonometrische Basis. Den Basiswechsel zu dieser Basis nennen wir diskrete Fourier Transformation.

\fancydef{Trigonometrische Basis}\\
\begin{align*}
    \{v_0, \ldots, v_{N-1}\} \text{ ist eine Basis von } \mathbb{C}^N, \text{ wobei } v_k =
    \begin{bmatrix}
        \omega_N^{0\cdot k} \\ \omega_N^{1\cdot k} \\ \vdots \\ \omega_N^{(N-1)\cdot k}
    \end{bmatrix}
    \in \mathbb{C}^N
\end{align*}

Die symmetrische, nicht hermitesche Matrix $V = [v_0,\ \ldots\ , v_{N-1}]$ ist eine orthogonale Basis für $\mathbb{C}^N$: $V^HV = N\cdot I_N$.\\
Ebenfalls ist $V$ die Basiswechsel Matrix Trigonometrische Basis ($z$) $\mapsto$ Standardbasis ($y$).\\
An Hand von $V$ definieren wir gleich die Fourier-Matrix $F_N$.
\begin{align*}
    y = Vz \implies z = V^{-1}y = \frac{1}{N}V^Hy = \frac{1}{N}\underbrace{F_N}_{:= V^H} y
\end{align*}

Der Eintrag $y_l$ enstspricht einem Glied der Fourier-Reihe ausgewertet in $\frac{l}{N} \in [0,1)$. \\
Die diskreten Fourier-Koeffizienten $\gamma_k$ sind eine Umsortierung der Koeffizienten der trigonometrischen Basis.
\begin{multicols}{2}
    \begin{align*}
        y = \underbrace{\sum_{k=0}^{N-1} y_k e_{k+1}}_{y \text{ in Komponenten}} = \underbrace{\sum_{k=0}^{N-1} z_k v_k}_{\text{in Trig. Basis}} = \sum_{k=0}^{N-1} z_k
        \begin{bmatrix}
            \omega_N^{0 \cdot k} \\ \omega_N^{1 \cdot k} \\ \omega_N^{2 \cdot k} \\ \vdots \\ \omega_N^{(N-1) \cdot k}
        \end{bmatrix}
    \end{align*}
    \newcolumn

    \begin{align*}
        y_l                   & = \sum_{k=0}^{N-1} z_k \omega_N^{l \cdot k} \overset{\text{S. 75}}{=} \sum_{k=-N/2}^{N/2-1} \gamma_k \cdot \exp(\frac{2\pi i}{N}lk) \\
        \text{wobei }\gamma_k & =
        \begin{cases}
            z_k,   & 0 < k \leq \frac{N}{2}-1 \\
            z_k+N, & -\frac{N}{2} \leq k < 0
        \end{cases}
    \end{align*}
\end{multicols}

\fancydef{Fourier-Matrix}
\begin{align*}
    F_N := V^H = [v_0, \ldots, v_{N-1}]^H =
    \begin{bmatrix}
        \omega_N^0 & \omega_N^0     & \cdots & \omega_N^0         \\
        \omega_N^0 & \omega_N^1     & \cdots & \omega_N^{N-1}     \\
        \omega_N^0 & \omega_N^2     & \cdots & \omega_N^{2(N-1)}  \\
        \vdots     & \vdots         &        & \vdots             \\
        \omega_N^0 & \omega_N^{N-1} & \cdots & \omega_N^{(N-1)^2}
    \end{bmatrix}
    =
    \begin{bmatrix}
        \omega_N^{jk}
    \end{bmatrix}^{N-1}_{j,k = 0}
    \in \mathbb{C}^{N\times N}
\end{align*}

Die skalierte Fourier-Matrix $\frac{1}{\sqrt{N}}F_N$ hat einige besondere Eigenschaften.

\setcounter{all}{6}
\inlinetheorem Die skalierte Fourier-Matrix $\frac{1}{\sqrt{N}}F_N$ ist unitär: $F_N^{-1} = \frac{1}{N} F_N^H = \frac{1}{N} \overline{F_N}$

\fancyremark{Eigenwerte von $\frac{1}{\sqrt{N}}F_N$} Die $\lambda$ von $\frac{1}{\sqrt{N}}F_N$ liegen in $\{1,-1,i,-i\}$.

Die diskrete Fourier-Transformation ist nun einfach die Anwendung der Basiswechsel-Matrix $F_N$.

\setcounter{all}{5}
\fancydef{Diskrete Fourier-Transformation} $\mathcal{F}_N: \mathbb{C} \to \mathbb{C}$ s.d. $\mathcal{F}_N(y) = F_Ny$
\begin{align*}
    \text{Für } c = \mathcal{F}_N(y) \text{ gilt: }\quad c_k = \sum_{j=0}^{N-1} y_j \omega_N^{kj}
\end{align*}

$c$ lässt sich als Repräsentation von $y$ im Frequenzbereich interpetieren. Durch die DFT können wir nun jederzeit zwischen der normalen und der Frequenz-perspektive wechseln. Das ermöglicht einige interessante Anwendungen.
