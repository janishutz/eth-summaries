\subsection{DFT und Chebyshev-Interpolation}
Mithilfe der DFT können günstig und einfach die Chebyshev-Koeffizienten berechnet werden.
Die Idee basiert auf dem Satz 2.4.16, durch welchen schon schnell klar wird, dass es eine Verbindung zwischen den Fourier-Koeffizienten und Chebyshev-Koeffizienten gibt.

Die Chebyshev-Knoten sind folgendermassen definiert:
\begin{align*}
    t_k := \cos\left( \frac{2k + 1}{2(n + 1)} \pi \right), \smallhspace k = 0, \ldots, n
\end{align*}
Mit den Hilfsfunktionen $g: [-1, 1] \rightarrow \C, s \mapsto f(\cos(2\pi s))$ und $q: [-1, 1] \rightarrow \C, s \mapsto p(\cos(2\pi s))$,
können wir folgendes mit der Interpolationsbedingung $f(t_k) = p(t_k)$ tun:
\begin{align*}
    f(t_k) = p(t_k)
\end{align*}
