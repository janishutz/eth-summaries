\subsection{DFT und Chebyshev-Interpolation}
Mithilfe der DFT können günstig und einfach die Chebyshev-Koeffizienten berechnet werden.
Die Idee basiert auf dem Satz 2.4.16, durch welchen schon schnell klar wird, dass es eine Verbindung zwischen den Fourier-Koeffizienten und Chebyshev-Koeffizienten gibt.

Die Chebyshev-Knoten sind folgendermassen definiert:
\begin{align*}
    t_k := \cos\left( \frac{2k + 1}{2(n + 1)} \pi \right), \smallhspace k = 0, \ldots, n
\end{align*}
Mit den Hilfsfunktionen $g: [-1, 1] \rightarrow \C, s \mapsto f(\cos(2\pi s))$ und $q: [-1, 1] \rightarrow \C, s \mapsto p(\cos(2\pi s))$,
können wir folgendes mit der Interpolationsbedingung $f(t_k) = p(t_k)$ tun:
\begin{align*}
    f(t_k) = p(t_k) \Longleftrightarrow g\left( \frac{2k + 1}{4(n + 1)} \right) = p\left( \frac{2k + 1}{4(n + 1)} \right)
\end{align*}
Wir wenden nun die Translation $s^* = s + \frac{1}{4n + 1}$ an, die Hilfsfunktionen sind dann $g*(s) = g(s^*)$ und $q^*(s) = q(s^*)$
und man kann zeigen (Seite 101 im Skript), dass $q^*$ das trigonometrische Interpolationspolynom von $g^*$ ist,
also kann man eine Chebyshev-Interpolation durch eine DFT durchführen.
Folglich überträgt sich auch die Fehlerabschätzung. Die Interpolationsbedingungen sind folgendermassen definiert:
\begin{align*}
    q\left( \frac{k}{2(n + 1)} + \frac{1}{4(n + 1)} \right) = z_k :=
    \begin{cases}
        y_k            & \text{ für } k = 0, \ldots, n \\
        y_{2n + 1 - k} & \text{ für } k = n, \ldots, 2n + 1
    \end{cases}
\end{align*}

Um das ganze zu implementieren ist eine andere Darstellung nützlich.
