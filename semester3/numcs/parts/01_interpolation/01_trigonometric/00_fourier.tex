% ┌                                                ┐
% │     AUTHOR: Janis Hutz<info@janishutz.com>     │
% └                                                ┘

% Lecture: Wir besitzen nicht das komplette Vorwissen in der Analysis für dieses Kapitel, d.h. wird totales Verständnis nicht 
%
% Lecture: Intuitiv wird Fourier-Trans. zur Kompression genutzt, z.b. jpg format.
\subsection{Fourier-Reihen}
Eine Anwendung der Schnellen Fourier-Transformation (FFT) ist die Komprimierung eines Bildes und sie wird im JPEG-Format verwendet.

\fancydef{Trigonometrisches Polynom von Grad $\leq m$} Die Funktion:
\rmvspace
\begin{align*}
	p_m(t) := t \mapsto \sum_{j = -m}^{m} \gamma_j e^{2 \pi ijt} \text{ wobei } \gamma_j \in \C \text{ und } t \in \R
\end{align*}
%
%
\inlineremark $p_m : \R \rightarrow \C$ ist periodisch mit Periode $1$.
Falls $\gamma_{-j} = \overline{\gamma_j}$ für alle $j$, dann ist $p_m$ reellwertig und
% NOTE: Uhh... do we want to use the fancy symbols for real and imaginary part or just use $\text{Re}$?
$p_m$ kann folgendermassen dargestellt werden ($a_0 = 2\gamma_0, a_j = 2\Re(\gamma_j)$ und $b_j = -2\Im(\gamma_j)$):
\rmvspace
\begin{align*}
	p_m(t) = \frac{a_0}{2} + \sum_{j = 1}^{m} (a_j \cos(2\pi jt) + b_j \sin(2\pi jt))
\end{align*}

\begin{definition}[]{$L^2$-Funktionen}
	Wir definieren die $L^2$-Funktionen auf dem Intervall $(0, 1)$ als
	\rmvspace
	\begin{align*}
		L^2(0, 1) := \{ f: (0, 1) \rightarrow \C \divides ||f||_{L^2(0, 1)} < \infty \}
	\end{align*}
	während die $L^2$-Norm auf $(0, 1)$ durch das Skalarprodukt
	\rmvspace
	\begin{align*}
		\langle g, f \rangle_{L^2(0, 1)} := \int_{0}^{1} \overline{g(x)} f(x) \dx x
	\end{align*}
	über $||f||_{L^2(0, 1)} = \sqrt{\langle f, f \rangle_{L^2(0, 1)}}$ induziert wird
\end{definition}

\inlineremark $L^2(a, b)$ lässt sich analog definieren mit
\rmvspace
\begin{align*}
	\langle g, f \rangle_{L^2(a, b)} & := \int_{a}^{b} \overline{g(x)} f(x) \dx x                              \\
	                                 & = (b - a) \int_{0}^{1} \overline{g(a + (b - a)t)} f(a + (b - a)t) \dx t
\end{align*}
In Anwendungen findet sich oft das Intervall $\left[ -\frac{T}{2}, \frac{T}{2} \right]$.
Dann verwandeln sich die Integrale in die Form $\frac{1}{T} \int_{\frac{T}{2}}^{-\frac{T}{2}} (\ldots) \dx t$ und $\exp(2\pi ijt)$ durch $\exp(i \frac{2\pi j}{T} t)$ ersetzt wird.

\stepcounter{all}
\inlineremark Die Funktionen $\varphi_k(x) = \exp(2\pi ikx)$ sind orthogonal bezüglich des $L^2(0, 1)$-Skalarprodukts, bilden also eine Basis für den Unterraum der trigonometrischen polynome.


\inlinedef Eine Funktion $f$ ist der $L^2$-Grenzwert von Funktionenfolgen $f_n \in L^2(0, 1)$, wenn für $n \rightarrow \infty$ gilt, dass $||f - f_n||_{L^2(0, 1)} \rightarrow 0$


\begin{theorem}[]{Fourier-Reihe}
    Jede Funktion $f \in L^2(0, 1)$ ist der Grenzwert ihrer Fourier-Reihe:
    \rmvspace
    \begin{align*}
        f(t) = \sum_{k = -\infty}^{\infty} \hat{f}(k) e^{2\pi ikt}
    \end{align*}
    wobei die Fourier-Koeffizienten 
    \rmvspace
    \begin{align*}
        \hat{f}(k) = \int_{0}^{1} f(t)e^{-2\pi ikt} \dx t \smallhspace k \in \Z
    \end{align*}
    definiert sind. Es gilt die Parseval'sche Gleichung:
    \rmvspace
    \begin{align*}
        \sum_{k = -\infty}^{\infty} |\hat{f}(k)|^2 = ||f||_{L^2(0, 1)}^2
    \end{align*}
\end{theorem}
\inlineremark Oder viel einfacher und kürzer: Die Funktionen $\varphi_k(x)$ bilden eine vollständige Orthonormalbasis in $L^2(0, 1)$.

\setcounter{all}{14}
\inlineremark Die Parseval'sche Gleichung beschreibt einfach gesagt einen ``schnellen'' Abfall der $\hat{f}(k)$.
Genauer gesagt, klingen die Koeffizienten schneller als $\frac{1}{\sqrt{k}}$ ab.
Sie sagt zudem aus, dass die $L^2$-Norm der Funktion aus einer Summe berechnet werden kann (nicht nur als Integral). 
Wenn wir die Fourier-Reihe nach $t$ ableiten, erhalten wir
\rmvspace
\begin{align*}
    f'(t) = \sum_{k = -\infty}^{\infty} 2\pi ik\hat{f}(k)e^{2\pi ikt}
\end{align*}

\begin{theorem}[]{Fourier-Reihe}
    Seien $f$ und $f'$ integrierbar auf $(0, 1)$, dann gilt $\hat{f'}(k) = 2\pi ik\hat{f}(k)$ für $k \in \Z$.

    Falls die Operationen erlaubt sind, dann gilt zudem:
    \rmvspace
    \begin{align*}
        \hat{f^{(n)}} = (2\pi ik)^n \hat{f}(k) \text{ und } ||f^{(n)}||_{L^2}^2 = (2\pi)^{2n} \sum_{k = -\infty}^{\infty} k^{2n} |\hat{f}(k)|^2
    \end{align*}
\end{theorem}


\inlinetheorem Wenn $\displaystyle \int_{0}^{1} |f^{(n)}(t)|\dx t < \infty$, dann ist $\hat{f}(k) = \tco{k^{-n}}$
