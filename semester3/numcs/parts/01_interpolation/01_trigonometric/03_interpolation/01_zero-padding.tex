\newpage
\subsubsection{Zero-Padding-Auswertung}
Ein trigonometrisches Polynom $p_{N - 1}(t)$ kann effizient an den äquidistanten Punkten $\frac{k}{M}$ mit $M > N$ ausgewertet werden, für $k = 0, \ldots, M - 1$.
Dazu muss das Polynom $p_{N - 1} \in \mathcal{T}_N \subseteq \mathcal{T}_M$ in der trigonometrischen Basis $\mathcal{T}_M$ neugeschrieben werden,
in dem man \bi{Zero-Padding} verwendet, also Nullen im Koeffizientenvektor an den Stellen höheren Frequenzen einfügt.

\TODO Insert cleaned up code from Page 95 (part of exercises)
