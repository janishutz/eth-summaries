\subsection{Grundlagen}
\inlineremark Eine Tabelle mit invertierbaren und nicht invertierbaren Matrizen findet sich unten:
\begin{tables}{ll}{Invertierbar                                             & Nicht Invertierbar}
              $A$ ist regulär                                               & $A$ ist singulär                                                  \\
              Spalten sind linear unabhängig                                & Spalten sind linear abhängig                                      \\
              Zeilen sind linear unabhängig                                 & Zeilen sind linear abhängig                                       \\
              $\det(A) \neq 0$                                              & $\det(A) = 0$                                                     \\
              $Ax = 0$ hat eine Lösung $x = b$                              & $Ax = 0$ hat unendlich viele Lösungen                             \\
              $Ax = b$ hat eine Lösung $x = A^{-1}b$                        & $Ax = b$ hat keine oder unendlich viele Lösungen                  \\
              $A$ hat vollen Rang                                           & $A$ hat Rang $r < n$                                              \\
              $A$ hat $n$ non-zero Pivots                                   & $A$ hat $r < n$ Pivots                                            \\
              $\text{span}\{A_{:, 1}, \ldots, A_{:, n}\}$ hat Dimension $n$ & $\text{span}\{A_{:, 1}, \ldots, A_{:, n}\}$ hat Dimension $r < n$ \\
              $\text{span}\{A_{1, :}, \ldots, A_{n, :}\}$ hat Dimension $n$ & $\text{span}\{A_{1, :}, \ldots, A_{n, :}\}$ hat Dimension $r < n$ \\
              Alle Eigenwerte von $A$ sind nicht Null                       & $0$ ist der Eigenwert von $A$                                     \\
              $0 \notin \sigma(A) =$ Spektrum von $A$                       & $0 \in \sigma(A)$                                                 \\
              $A^H A$ ist symmetrisch positiv definit                       & $A^H A$ ist nur semidefinit                                       \\
              $A$ hat $n$ (positive) Singulärwerte                          & $A$ hat $r < n$ (positive) Singulärwerte                          \\
\end{tables}

\fancydef{Orthogonale Vektoren} Vektoren $q_1, \ldots, q_n$ heissen \bi{orthogonal}, falls
\rmvspace
\begin{align*}
    q_i^H \cdot q_j = 0 \smallhspace \forall i, j \leq n \text{ with } i \neq j
\end{align*}

\drmvspace
Wenn sie zudem normiert sind (also $||q_i||_2 = 1 \smallhspace \forall i \leq n$), dann heissen sie \bi{orthonormal}

\inlineremark In der vorigen Definition wird die \bi{Euklidische Norm} $||q||_2^2 = q^H \cdot q$ verwendet

\setLabelNumber{all}{7}
\fancyremark{Rotationen} Die Rotationsmatrix für eine Rotation um Winkel $\theta$ ist gegeben durch:
\rmvspace
\begin{align*}
    R_\theta =
    \begin{bmatrix}
        \cos(\theta) & 0 & -\sin(\theta) \\
        0            & 1 & 0             \\
        \sin(\theta) & 0 & \cos(\theta)
    \end{bmatrix}
\end{align*}


\drmvspace
\shade{gray}{Perturbierte LGS}
\setLabelNumber{all}{18}

Statt $Ax = b$ ist das LGS ungenau gegeben: $(A + \Delta A)(\tilde{x} - x) = \Delta b - \Delta Ax$.

\fancydef{Konditionszahl} $\text{cond}(A) := ||A^{-1}|| \cdot ||A||$. Manchmal auch mit $\kappa(A)$ notiert

Auch hier gibt es sie wieder für verschiedene Normen:
\begin{itemize}[noitemsep]
    \item $\kappa_2(A) = \frac{\sigma_\text{max} (A)}{\sigma_\text{min}(A)}$ (Spektralnorm mit Singulärwerten)
    \item $\kappa_\infty(A) = ||A||_\infty \cdot ||A^{-1}||_\infty$
    \item $\kappa_1(A) = ||A||_1 \cdot ||A^{-1}||_1$
\end{itemize}

$\text{cond}(A) \gg 1$ bedeutet intuitiv: kleine Änderung der Daten $\mapsto$ grosse Änderung in der Lösung

Zudem haben wir folgende Eigenschaften:
\drmvspace
\begin{multicols}{2}
    \begin{itemize}[noitemsep]
        \item $\kappa(A) \geq 1$
        \item $\kappa(cA) = \kappa(A) \smallhspace \forall c \neq 0$
        \item $\kappa(A) = \kappa(A^{-1})$
        \item Für orthogonale und unitäre Matrizen $Q$: $\kappa_2(Q) = 1$
    \end{itemize}
\end{multicols}



\drmvspace
\shade{gray}{Grosse Matrizen}

Passen oft nicht (direkt) in den Speicher: effizientere Speicherung nötig, möglich für z.B. Diagonalmatrizen, Dreiecksmatrizen. Auch für Cholesky möglich.


% TODO: Update with notes from TA and script
\textbf{Dünnbesetzte Matrizen}

$\text{nnz}(A) := |\{ (i,j) \ |\ a_{ij} \in A, a_{ij} \neq 0 \}| \ll m\cdot n$

$\limit{l}{\infty} \frac{\text{nnz}(A^{(l)})}{n_l m_l} = 0$

Einfacher zu speichern: \verb|val, col, row| sind Vektoren so dass \verb|val[k]| $ = a_{ij}$, wobei $i=$ \verb|row[k]|, $j=$ \verb|col[k]|. (nur $a_{ij} \neq 0$)

Es gibt viele Formate, je nach Anwendung sind gewisse sinnvoller als andere. (Siehe Tabelle, NumCSE) % TODO: Insert here

\verb|scipy.sparse.csr_matrix(A)| $\mapsto$ Dramatische Speichereinsparung.\\
Deprecated: \verb|bsr_array| und \verb|coo_array| verwenden, kompatibel mit \verb|numpy| arrays.

\verb|CSC, CSR| erlauben weitere Optimierungen, je nach Gewichtung der $a_{ij}$ auf Zeilen, Spalten.
