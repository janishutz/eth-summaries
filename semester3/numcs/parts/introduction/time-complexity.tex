\newsection
\subsection{Rechenaufwand}
In NumCS wird die Anzahl elementarer Operationen wie Addition, Multiplikation, etc benutzt, um den Rechenaufwand zu beschreiben. 
Wie in Algorithmen und * ist auch hier wieder $\tco{\ldots}$ der Worst Case.
Teilweise werden auch andere Funktionen wie $\sin, \cos, \sqrt{\ldots}, \ldots$ dazu gezählt.

Die Basic Linear Algebra Subprograms (= BLAS), also grundlegende Operationen der Linearen Algebra, wurden bereits stark optimiert und sollten wann immer möglich verwendet werden und man sollte auf keinen Fall diese selbst implementieren.

Dieser Kurs verwendet \texttt{numpy}, \texttt{scipy}, \texttt{sympy} (collection of implementations for symbolic computations) und \texttt{matplotlib}.
Dieses Ecosystem ist eine der Stärken von Python und ist interessanterweise zu einem Grossteil nicht in Python geschrieben, da dies sehr langsam wäre.
