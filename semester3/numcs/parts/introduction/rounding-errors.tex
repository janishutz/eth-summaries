\subsection{Rundungsfehler}

\begin{definition}[]{Absoluter \& Relativer Fehler}
    \begin{multicols}{2}
        \begin{itemize}
            \item \bi{Absoluter Fehler}: $||\tilde{x} - x||$
            \item \bi{Relativer Fehler}: $\displaystyle \frac{||\tilde{x} - x||}{||x||}$ für $||x|| \neq 0$
        \end{itemize}
    \end{multicols}
    wobei $\tilde{x}$ eine Approximation an $x \in \R$ ist
\end{definition}

Rundungsfehler entstehen durch die (verhältnismässig) geringe Präzision die man mit der Darstellung von Zahlen auf Computern erreichen kann.
Zusätzlich kommt hinzu, dass durch Unterläufe (in diesem Kurs ist dies eine Zahl die zwischen $0$ und der kleinsten darstellbaren, positiven Zahl liegt) Präzision verloren gehen kann.

Überläufe hingegen sind konventionell definiert, also eine Zahl, die zu gross ist und nicht mehr dargestellt werden kann.


\begin{remark}[]{Auslöschung}
    Bei der Subtraktion von zwei ähnlich grossen Zahlen kann es zu einer Addition der Fehler der beiden Zahlen kommen, was dann den relativen Fehler um einen sehr grossen Faktor vergrössert.
    Die Subtraktion selbst hat einen vernachlässigbaren Fehler
\end{remark}

% Page 10


\subsection{Richardson Konvergenzbeschleunigung}
\begin{align*}
    y f'(x) & = y d\left(\frac{h}{2}\right) + \frac{1}{6} f'''(x) h^2 + \frac{1}{480}f^{(s)} h^4 + \ldots - f'(x) \\
            & = -d(h) - \frac{1}{6}f'''(x) h^2 + \frac{1}{120} f^{(s)}(x) h^n \Leftrightarrow 3 f'(x)             \\
            & = 4 d\left(\frac{h}{2}\right)  d(h) + \text{\tco{h^4}} \Leftrightarrow
\end{align*}


\fhlc{Cyan}{Schema}

\begin{align*}
    d(h) = \frac{f(x + h) - f(x - h)}{2h}
\end{align*}

wobei im Schema dann
\begin{align*}
    R_{l, 0} = d\left( \frac{h}{2^l} \right)
\end{align*}
und
\begin{align*}
    R_{l, k} = \frac{4^k \cdot R_{l, k - 1} - R_{l - 1, k - 1}}{4^k - 1}
\end{align*}
und $f'(x) = R_{l, k} + C \cdot \left( \frac{h}{2^l} \right)^{2k + 2}$
