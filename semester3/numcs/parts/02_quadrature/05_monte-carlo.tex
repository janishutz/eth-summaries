\newsection
\newcommand{\tsigma}{\tilde{\sigma}}
\subsection{Monte-Carlo Quadratur}
Bei der Monte-Carlo Quadratur wird, wie bei anderen Monte-Carlo-Algorithmen der Zufall genutzt

\inlineremark Die Konvergenz ist sehr langsam ($\sqrt{N}$), aber nicht abhängig von der Dimension oder Glattheit.
Zudem kann das Ergebnis falsch sein, da es probabilistisch ist.

Jede Monte-Carlo-Methode benötigt folgendes mit $X = [I_N - \tsigma_N, I_N + \tsigma_N]$:
\rmvspace
\begin{multicols}{2}
    \begin{itemize}[noitemsep]
        \item ein Gebiet für das ``Experiment'', hier $[0, 1]^d$
        \item gute Zufallszahlen
        \item gute deterministische Berechnungen, hier $\tsigma_N$ und $I_N$
        \item Darstellung des Ergebnis, hier $\Pr[I \in X] = 0.683$
    \end{itemize}
\end{multicols}
Mit $\displaystyle I_N = \int_{0}^{1} z(t) \dx t = \frac{1}{N} \sum_{i = 1}^{N} z(t_i)$, wobei $t_i$ Zufallszahlen sind und
\rmvspace
\begin{align*}
    \tsigma_N = \sqrt{\frac{\frac{1}{N} \sum_{i = 1}^{N} z(t_i)^2 - \left( \frac{1}{N} \sum_{i = 1}^{N} z(t_i) \right)^2}{N - 1}} = \frac{\sigma_N}{\sqrt{N}}
\end{align*}

% TODO: Consider adding some of the theory on random variables here (especially consider normal distribution)


Das Monte-Carlo-Verfahren beruht auf folgendem:
\rmvspace
\begin{align*}
    \int_{[0, 1]^d} z(x) \dx x = \E z(\mathcal{X}) \text{ mit } \mathcal{X} \sim \mathcal{U}([0, 1]^d)
\end{align*}
% TODO: Clarify what \mathcal{U} is

\drmvspace
Das Ziel der Monte-Carlo-Methode ist es, den Erwartungswert durch den Mittelwert der Funktionswerte der simulierten Zufallsvariable mit einem Schätzer $m_N(z(\mathcal{X}))$, bzw. einer Schätzung $m_N(z(x))$ zu approximieren:
\rmvspace
\begin{align*}
    m_N(z(\cX)) & := \frac{1}{N} \sum_{i = 1}^{N} z(\cX_i) &
    m_N(z(x))           & := \frac{1}{N} \sum_{i = 1}^{N} z(x_i)
\end{align*}

\setLabelNumber{all}{16}
\inlineremark Wir verwenden $m_N(z(x))$ für das $z(x)$ im obigen Integral:
\rmvspace
\begin{align*}
    \E m_N(z(\cX)) = N \frac{1}{N} \E z(\cX) = \int_{[0, 1]^d} z(x) \dx x
\end{align*}

\drmvspace
Die Approximation ist besser, je kleiner die Varianz ist:
\rmvspace
\begin{align*}
    \V m_N(z(\cX)) = \V \left( \frac{1}{N} \sum_{i = 1}^{N} z(\cX_i) \right) = \frac{1}{N^2} N \V(z(\cX)) = \frac{1}{N} \V(z(\cX)) \rightarrow 0
\end{align*}
