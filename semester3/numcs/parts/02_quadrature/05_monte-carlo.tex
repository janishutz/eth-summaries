\newsection
\newcommand{\tsigma}{\tilde{\sigma}}
\subsection{Monte-Carlo Quadratur}
Bei der Monte-Carlo Quadratur wird, wie bei anderen Monte-Carlo-Algorithmen der Zufall genutzt

\inlineremark Die Konvergenz ist sehr langsam ($\sqrt{N}$), aber nicht abhängig von der Dimension oder Glattheit.
Zudem kann das Ergebnis falsch sein, da es probabilistisch ist.

Jede Monte-Carlo-Methode benötigt folgendes mit $X = [I_N - \tsigma_N, I_N + \tsigma_N]$:
\rmvspace
\begin{multicols}{2}
    \begin{itemize}[noitemsep]
        \item ein Gebiet für das ``Experiment'', hier $[0, 1]^d$
        \item gute Zufallszahlen
        \item gute deterministische Berechnungen, hier $\tsigma_N$ und $I_N$
        \item Darstellung des Ergebnis, hier $\Pr[I \in X] = 0.683$
    \end{itemize}
\end{multicols}
