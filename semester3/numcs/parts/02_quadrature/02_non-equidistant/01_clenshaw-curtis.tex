\subsubsection{Clenshaw-Curtis Quadraturformel}
Die erste Quadraturformel von Fejér benutzt die Chebyshev-Knoten (Nullstellen der Chebyshev-Polynome erster Art), welche aber nicht verschachtelt sind.
Die zweite Quadraturformel von Fejér benutzt die Filippi-Knoten $x_k = \cos\left( k \frac{\pi}{n} \right)$ für $k = 1, \ldots, n - 1$
und Clenshaw und Curtis haben dann zusätzlich noch die Endknoten hinzugefügt (also $k = 0, \ldots, n$).
Die Clenshaw-Curtis-Knoten sind die Chebyshev-Abszissen und die Formel verhält sich mit den entsprechenden Gewichten ähnlich gleich wie die Gauss-Quadratur.

Da die Clenshaw-Curtis-Quadratur mithilfe der DFT berechnet werden kann ist sie sehr effizient.
Dazu müssen wir aber zuerst etwas umformen, mit $x = \cos(\theta)$, so dass das Integral eine periodische Funktion wird:
\rmvspace
\begin{align*}
    \int_{-1}^{1} f(x) \dx x = \int_{0}^{\pi} f(\cos(\theta)) \sin(\theta) \dx \theta = f(\cos(\theta))
\end{align*}

\drmvspace
$F(\theta)$ ist $2\pi$-periodisch und gerade, kann sich also in eine Kosinius-Reihe entwickeln, also: $F(\theta) = \sum_{k = 0}^{\infty} a_k \cos(k \theta)$, woraus folgt, dass
\drmvspace
\begin{align*}
    \int_{0}^{\pi} F(\theta) \sin(\theta) \dx \theta = \ldots = a_0 + \sum_{2 \leq k \text{ gerade}} \frac{2a_k}{1 - k^2}
\end{align*}

\drmvspace
wobei sich die Koeffizienten $a_k$ mit FFT oder DCT berechnen lassen

% TODO: Insert code from TA slides here

Eine wichtige Erkenntnis ist, dass die Newton-Cotes bei grösserer Ordnung komplett unbrauchbar werden, wie das in Abbildung 5.5.24 im Skript zu sehen ist,
während die Clenshaw-Curtis-Quadratur ähnlich gut ist wie die Gauss-Quadratur (gleiche Konvergenzordnung).

\begin{fullTable}{llllll}{Quadratur         & Intervall     & Gewichtsfunktion               & Polynom      & Notation                & \texttt{scipy.special.}}
    {Gewichtsfunktionen für Quadraturformeln}
    Gauss                      & $(-1, 1)$     & $1$                            & Legendre     & $P_k$                   & \texttt{roots\_legendre}    \\
    Chebyshev I                & $(-1, 1)$     & $\frac{1}{\sqrt{1 - x^2}}$     & Chebyshev I  & $T_k$                   & \texttt{roots\_chebyt}      \\
    Chebyshev II               & $(-1, 1)$     & $\sqrt{1 - x^2}$               & Chebyshev II & $U_k$                   & \texttt{roots\_chebyu}      \\
    Jacobi $\alpha, \beta > 1$ & $(-1, 1)$     & $(1 - x)^\alpha (1 + x)^\beta$ & Jacobi       & $P_k^{(\alpha, \beta)}$ & \texttt{roots\_jacobi}      \\
    Hermite                    & $\R$          & $e^{-x^2}$                     & Hermite      & $H_k$                   & \texttt{roots\_hermite}     \\
    Laguerre                   & $(0, \infty)$ & $x^\alpha e^{-x^2}$            & Laguerre     & $L_k$                   & \texttt{roots\_genlaguerre} \\
\end{fullTable}
