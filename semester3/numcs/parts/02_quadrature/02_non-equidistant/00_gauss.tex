\newsection
\subsection{Nicht äquidistante Stützstellen}
Alternativ zur Unterteilung des Intervalls können wir andere Quadraturpunkte erlauben

\subsubsection{Gauss Quadratur}
In diesem Kapitel werden die Gewichte mit $b_1, b_2, \ldots, b_s$ und die Knoten auf unserem Referenzintervall,
welches hier $[0, 1]$ ist, mit $c_1, c_2, \ldots, c_s \in [0, 1]$ bezeichnet.

% TODO: It would probably be a good idea to link the document together much better (and maybe create an index in the end?)
Wir möchten unsere Gewichte $b_i$ und Knoten $c_i$ so bestimmen, dass die Quadraturordnung maximal ist.

Wir definieren die Notation $\langle M, g \rangle = \int_{0}^{1} M(t) g(t) \dx t$ (also das Skalarprodukt).

\begin{theorem}[]{Ordnung der Quadraturformel}
    Die Ordnung ist $s + m$ genau dann, wenn $\langle M, g \rangle = 0$ für alle Polynome $g$ mit $\deg(g) \leq m - 1$ und 
    $M(t) = (t - c_1) \cdot (t - c_2) \cdot \ldots \cdot (t - c_s)$ für $s$. Also steht $M$ senkrecht zu allen $g$.
\end{theorem}

\fancytheorem{Maximale Ordnung einer Quadraturformel} Die Ordnung einer Quadraturformel mit $s$ Knoten ist $\leq 2s$


\fhlc{lime}{Orthogonale Polynome}

Für $I = ]a, b[$ sei $w: I \rightarrow \R$ eine stetige Gewichtsfunktion mit $w(x) > 0 \smallhspace \forall x \in I$
