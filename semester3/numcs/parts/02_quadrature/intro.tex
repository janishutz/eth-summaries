\setcounter{subsection}{2}
\subsection{Grundbegriffe und -Ideen}
Es ist oft nicht möglich oder sinnvoll einen Integral analytisch zu berechnen.
Mit Methoden der Quadratur können wir Integrale nummerisch berechnen.

\innumpy kann \texttt{scipy.integrate.quad} verwendet werden.
Falls man jedoch eine manuelle Implementation erstellen will, so nutzt man oft die Trapez- oder Simpson-Regel, da sie sowohl einfach zu implementieren, wie auch effizient sind.
In gewissen Anwendungen sind Gauss-Quadratur-Formeln nützlich, welche man durch Spektralmethoden ersetzen kann, welche die FFT verwenden und effizienter sind.

\begin{definition}[]{Quadratur}
    Ein Integral kann durch eine gewichtete Summe von Funktionswerten der Funktion $f$ an verschiedenen Stellen $c_i^n$ approximiert werden:
    \begin{align*}
        \int_{a}^{b} f(x) \dx \approx Q_n(f; a, b) := \sum_{i = 1}^{n} \omega_i^n f(c_i^n)
    \end{align*}
    wobei $\omega_i^n$ die \textit{Gewichte} und $c_i^n \in [a, b]$ die \textit{Knoten} der Quadraturformel sind.
\end{definition}

Wir wollen natürlich wieder $c_i^n \in [a, b]$ und $w_i^n$ so wählen, dass der Fehler minimiert wird.
\begin{definition}[]{Fehler}
    Der Fehler der Quadratur $Q_n(f)$ ist
    \begin{align*}
        E(n) = \left| \int_{a}^{b} f(x) \dx - Q_n(f; a, b) \right|
    \end{align*}
    Wir haben \bi{algebraische Konvergenz} wenn $E(n) = \tco{\frac{1}{n^p}}$ mit $p > 0$ und 
    \bi{exponentielle Konvergenz} wenn $E(n) = \tco{q^n}$ mit $0 \leq q < 1$
\end{definition}

Die Idee, den Integral einer schweren Funktion zu berechnen, ist diese mit einer einfachen Funktion, die analytisch integrierbar ist, zu approximieren.
Wenn wir diese Funktion geschickt wählen, dann ist es sogar möglich, dass wir nur eine solche Funktion für alle Funktionen $f$ benötigen.

% Der Polynom, das Ansatz, yep... excellent German there
