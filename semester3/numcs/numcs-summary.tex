\documentclass{article}

\newcommand{\dir}{~/projects/latex}
\input{\dir/include.tex}
\load{full}

\usepackage{lmodern}
\setFontType{sans}

\renewcommand{\authorTitle}{Robin Bacher, Janis Hutz\\\url{https://github.com/janishutz/eth-summaries}}
\renewcommand{\authorHeaders}{Robin Bacher, Janis Hutz}
\setLang{de}

\setup{Numerical Methods for Computer Science}

\begin{document}
\startDocument
\usetcolorboxes
\setcounter{numberingConfig}{3}
\setcounter{numberSubsections}{1}

%          ╭────────────────────────────────────────────────╮
%          │                   Title page                   │
%          ╰────────────────────────────────────────────────╯
\vspace{2cm}
\begin{Huge}
    \begin{center}
        TITLE PAGE COMING SOON
    \end{center}
\end{Huge}


\vspace{4cm}
\begin{center}
    \begin{Large}
        ``\textit{Denken vor Rechnen}''
    \end{Large}

    \hspace{3cm} - Vasile Grudinaru, 2025
\end{center}

\vspace{3cm}
\begin{center}
    HS2025, ETHZ\\[0.2cm]
    \begin{Large}
        Summary of the Script and Lectures
    \end{Large}\\[0.2cm]
\end{center}


% ────────────────────────────────────────────────────────────────────

%          ╭────────────────────────────────────────────────╮
%          │               Table of Contents                │
%          ╰────────────────────────────────────────────────╯
\newpage
\printtoc{ForestGreen}


% ────────────────────────────────────────────────────────────────────

%          ╭────────────────────────────────────────────────╮
%          │                  Main content                  │
%          ╰────────────────────────────────────────────────╯

% ── Introduction ────────────────────────────────────────────────────
\newsection
\section{Einführung}
% ┌                                                ┐
% │     AUTHOR: Janis Hutz<info@janishutz.com>     │
% └                                                ┘

\subsection{Rundungsfehler}

\begin{definition}[]{Absoluter \& Relativer Fehler}
    \begin{multicols}{2}
        \begin{itemize}
            \item \bi{Absoluter Fehler}: $||\tilde{x} - x||$
            \item \bi{Relativer Fehler}: $\displaystyle \frac{||\tilde{x} - x||}{||x||}$ für $||x|| \neq 0$
        \end{itemize}
    \end{multicols}
    wobei $\tilde{x}$ eine Approximation an $x \in \R$ ist
\end{definition}

Rundungsfehler entstehen durch die (verhältnismässig) geringe Präzision die man mit der Darstellung von Zahlen auf Computern erreichen kann.
Zusätzlich kommt hinzu, dass durch Unterläufe (in diesem Kurs ist dies eine Zahl die zwischen $0$ und der kleinsten darstellbaren, positiven Zahl liegt) Präzision verloren gehen kann.

Überläufe hingegen sind konventionell definiert, also eine Zahl, die zu gross ist und nicht mehr dargestellt werden kann.


\setcounter{all}{9}
\begin{remark}[]{Auslöschung}
    Bei der Subtraktion von zwei ähnlich grossen Zahlen kann es zu einer Addition der Fehler der beiden Zahlen kommen, was dann den relativen Fehler um einen sehr grossen Faktor vergrössert.
    Die Subtraktion selbst hat einen vernachlässigbaren Fehler
\end{remark}

\setcounter{all}{18}
\fancyex{Ableitung mit imaginärem Schritt} Als Referenz in Graphen wird hier oftmals die Implementation des Differenzialquotienten verwendet.

Der Trick hier ist, dass wir mit Komplexen Zahlen in der Taylor-Approximation einer glatten Funktion in $x_0$ einen rein imaginären Schritt durchführen können:
\begin{align*}
    f(x_0 + ih) = f(x_0) + f'(x_0)ih - \frac{1}{2} f''(x_0)h^2 - iC \cdot h^3 \text{ für } h \in \R \text{ und } h \rightarrow 0
\end{align*}
Da $f(x_0)$ und $f''(x_0)h^2$ reell sind, verschwinden die Terme, wenn wir nur den Imaginärteil des Ausdruckes weiterverwenden. Nach weiteren Vereinfachungen und Umwandlungen erhalten wir
\begin{align*}
    f'(x_0) \approx \frac{\text{Im}(f(x_0 + ih))}{h}
\end{align*}

Falls jedoch hier die Auswertung von $\text{Im}(f(x_0 + ih))$ nicht exakt ist, so kann der Fehler beträchtlich sein.


\setcounter{all}{20}
\fancyex{Konvergenzbeschleunigung nach Richardson}
\begin{align*}
    y f'(x) & = y d\left(\frac{h}{2}\right) + \frac{1}{6} f'''(x) h^2 + \frac{1}{480}f^{(s)} h^4 + \ldots - f'(x) \\
            & = -d(h) - \frac{1}{6}f'''(x) h^2 + \frac{1}{120} f^{(s)}(x) h^n \Leftrightarrow 3 f'(x)             \\
            & = 4 d\left(\frac{h}{2}\right)  d(h) + \tco{h^4} \Leftrightarrow
\end{align*}
% TODO: Need to finish with notes from the exercise sessions


\fhlc{Cyan}{Schema}

\begin{align*}
    d(h) = \frac{f(x + h) - f(x - h)}{2h}
\end{align*}

wobei im Schema dann
\begin{align*}
    R_{l, 0} = d\left( \frac{h}{2^l} \right)
\end{align*}
und
\begin{align*}
    R_{l, k} = \frac{4^k \cdot R_{l, k - 1} - R_{l - 1, k - 1}}{4^k - 1}
\end{align*}
und $f'(x) = R_{l, k} + C \cdot \left( \frac{h}{2^l} \right)^{2k + 2}$

% ┌                                                ┐
% │     AUTHOR: Janis Hutz<info@janishutz.com>     │
% └                                                ┘

\newsection
\subsection{Rechenaufwand}
In NumCS wird die Anzahl elementarer Operationen wie Addition, Multiplikation, etc benutzt, um den Rechenaufwand zu beschreiben. 
Wie in Algorithmen und * ist auch hier wieder $\tco{\ldots}$ der Worst Case.
Teilweise werden auch andere Funktionen wie $\sin, \cos, \sqrt{\ldots}, \ldots$ dazu gezählt.

Die Basic Linear Algebra Subprograms (= BLAS), also grundlegende Operationen der Linearen Algebra, wurden bereits stark optimiert und sollten wann immer möglich verwendet werden und man sollte auf keinen Fall diese selbst implementieren.

Dieser Kurs verwendet \texttt{numpy}, \texttt{scipy}, \texttt{sympy} (collection of implementations for symbolic computations) und \texttt{matplotlib}.
Dieses Ecosystem ist eine der Stärken von Python und ist interessanterweise zu einem Grossteil nicht in Python geschrieben, da dies sehr langsam wäre.

% ┌                                                ┐
% │     AUTHOR: Janis Hutz<info@janishutz.com>     │
% └                                                ┘

\newsection
\subsection{Rechnen mit Matrizen}
Wie in Lineare Algebra besprochen, ist das Resultat der Multiplikation einer Matrix $A \in \C^{m \times n}$ und einer Matrix $B \in \C^{n \times p}$ ist eine Matrix $AB = \in \C^{m \times p}$

\innumpy haben wir folgende Funktionen:
\begin{itemize}
    \item \verb|b @ a| (oder \verb|np.dot(b, a)| oder \verb|np.einsum('i,i', b, a)| für das Skalarprodukt
    \item \verb|A @ B| (oder \verb|np.einsum('ik,kj->ij', )|) für das Matrixprodukt
    \item \verb|A @ x| (oder \verb|np.einsum('ij,j->i', A, x)|) für Matrix $\times$ Vektor
    \item \verb|A.T| für die Transponierung
    \item \verb|A.conj()| für die komplexe Konjugation (kombiniert mit \verb|.T| = Hermitian Transpose)
    \item \verb|np.kron(A, B)| für das Kroneker Produkt
    \item \verb|b = np.array([4.j, 5.j])| um einen Array mit komplexen Zahlen zu erstellen (\verb|j| ist die imaginäre Einheit, aber es muss eine Zahl direkt daran geschrieben werden)
\end{itemize}


\setLabelNumber{all}{4}
\fancyremark{Rang der Matrixmultiplikation} $\text{Rang}(AX) = \min(\text{Rang}(A), \text{Rang}(X))$

\setLabelNumber{all}{7}
\fancyremark{Multiplikation mit Diagonalmatrix $D$} $D \times A$ skaliert die Zeilen von $A$ während $A \times D$ die Spalten skaliert

\stepcounter{all}
\inlineex \verb|D @ A| braucht $\tco{n^3}$ Operationen, wenn wir jedoch \verb|D.diagonal()[:, np.newaxis] * A| verwenden, so haben wir nur noch $\tco{n^2}$ Operationen, da wir die vorige Bemerkung Nutzen und also nur noch eine Skalierung vornehmen.
So können wir also eine ganze Menge an Speicherzugriffen sparen, was das Ganze bedeutend effizienter macht

\setLabelNumber{all}{14}
\inlineremark Wir können bestimmte Zeilen oder Spalten einer Matrix skalieren, in dem wir einer Identitätsmatrix im unteren Dreieck ein Element hinzufügen.
Wenn wir nun diese Matrix $E$ (wie die in der $LU$-Zerlegung) linksseitig mit der Matrix $A$ multiplizieren (bspw. $E^{(2, 1)}A$), dann wird die zugehörige Zeile skaliert.
Falls wir aber $AE^{(2, 1)}$ berechnen, so skalieren wir die Spalte

\fancyremark{Blockweise Berechnung} Man kann das Matrixprodukt auch Blockweise berechnen.
Dazu benutzen wir eine Matrix, deren Elemente andere Matrizen sind, um grössere Matrizen zu generieren.
Die Matrixmultiplikation funktioniert dann genau gleich, nur dass wir für die Elemente Matrizen und nicht Skalare haben.

% ────────────────────────────────────────────────────────────────────
\hspace{1mm}
\hrule
\hspace{1mm}
Untenstehend eine Tabelle zum Vergleich der Operationen auf Matrizen

\begin{tables}{lcccc}{Name           & Operation & Mult  & Add         & Komplexität}
              Skalarprodukt          & $x^H y$   & $n$   & $n - 1$     & $\tco{n}$    \\
              Tensorprodukt          & $x y^H$   & $nm$  & $0$         & $\tco{mn}$   \\
              Matrix $\times$ Vektor & $Ax$      & $mn$  & $(n - 1)m$  & $\tco{mn}$   \\
              Matrixprodukt          & $AB$      & $mnp$ & $(n - 1)mp$ & $\tco{mnp}$  \\
\end{tables}
\inlineremark Das Matrixprodukt kann mit Strassen's Algorithmus mithilfe der Block-Partitionierung in $\tco{n^{\log_2(7)}} \approx \tco{n^{2.81}}$ berechnet werden.

\fancyremark{Rang 1 Matrizen} Können als Tensorprodukt von zwei Vektoren geschrieben werden.
Dies ist beispielsweise hierzu nützlich:

Sei $A = ab^\top$. Dann gilt $y = Ax \Leftrightarrow y = a(b^\top x)$, was dasselbe Resultat ergibt, aber nur $\tco{m + n}$ Operationen und nicht $\tco{mn}$ benötigt wie links.

\inlineex Für zwei Matrizen $A, B \in \R^{n \times p}$ mit geringem Rang $p \ll n$, dann kann mithilfe eines Tricks die Rechenzeit von \verb|np.triu(A @ B.T) @ x| von $\tco{pn^2}$ auf $\tco{pn}$ reduziert werden.
Die hier beschriebene Operation berechnet $\text{Upper}(AB^\top) x$ wobei $\text{Upper}(X)$ das obere Dreieck der Matrix $X$ zurück gibt.
Wir nennen diese Matrix hier $R$.

\innumpy können wir den folgenden Ansatz verwenden, um die Laufzeit zu verringern:
Da die Matrix $R$ eine obere Dreiecksmatrix ist, ist das Ergebnis die Teilsummen von unserem Umgekehrten Vektor $x$,
also können wir mit \verb|x[::-1].cumsum(axis=0)[::-1]| die Kummulative Summe berechnen.
Das \verb|[::-1]| dient hier lediglich dazu, den Vektor $x$ umzudrehen, sodass das richtige Resultat entsteht und die \texttt{axis=0} muss nur spezifiziert werden,
falls wir nicht den Default von \texttt{None} wollen, welcher die \texttt{cumsum} auf \texttt{x.flat} ausführt.
Die vollständige Implementation sieht so aus:
\begin{code}{python}
    def low_rank_matrix_vector_product(A: np.ndarray, B: np.ndarray, x: np.ndarray):
        n = A.shape[0]
        y = np.zeros(n) # Results vector

        # Compute B * x with broadcasting (x needs to be reshaped to 2D)
        v = B * x[:, None]

        # s is defined as the reverse cummulative sum of our vector
        # (and we need it reversed again for the final calculation to be correct)
        s = v[::-1].cumsum(axis=0)[::-1]

        y = np.sum(A * s)
\end{code}


\setLabelNumber{all}{21}
\fancydef{Kronecker-Produkt} Das Kronecker-Produkt ist eine $(ml) \times (nk)$-Matrix, für $A \in \R^{m \times n}$ und $B \in \R^{l \times k}$.

\innumpy können wir dieses einfach mit \verb|np.kron(A, B)| berechnen (ist jedoch nicht immer ideal):
\begin{align*}
    A \otimes B :=
    \begin{bmatrix}
        (A)_{1, 1} B & (A)_{1, 2}B & \ldots & \ldots & (A)_{1, n} B \\
        (A)_{2, 1} B & (A)_{2, 2}B & \ldots & \ldots & (A)_{2, n} B \\
        \vdots       & \vdots      & \ddots & \ddots & \vdots       \\
        (A)_{m, 1} B & (A)_{m, 2}B & \ldots & \ldots & (A)_{m, n} B \\
    \end{bmatrix}
\end{align*}

\fancyex{Multiplikation des Kronecker-Produkts mit Vektor} Wenn man $A \otimes B \cdot x$ berechnet, so ist die Laufzeit $\tco{m \times n \times l \times k}$, aber wenn wir den Vektor $x$ in $n$ gleich grosse Blöcke aufteilen (was man je nach gewünschter nachfolgender Operation in NumPy in $\tco{1}$ machen kann mit \verb|x.reshape(n, x.shape[0] / n)|), dann ist es möglich das Ganze in $\tco{m \cdot l \cdot k}$ zu berechnen.

Die vollständige Implementation ist auch hier nicht schwer und sieht folgendermassen aus:
\begin{code}{python}
    def fast_kron_vector_product(A: np.ndarray, B: np.ndarray, x: np.ndarray):
        # First multiply Bx_i, (and define x_i as a reshaped numpy array to save cost (as that will create a valid array))
        # This will actually crash if x.shape[0] is not divisible by A.shape[0]
        bx = B * x.reshape(A.shape[0], round(x.shape[0] / A.shape[0]))
        # Then multiply a with the resulting vector
        y = A @ bx
\end{code}

Um die oben erwähnte Laufzeit zu erreichen muss erst ein neuer Vektor berechnet werden, oben im Code \verb|bx| genannt, der eine Multiplikation von \verb|Bx_i| als Einträge hat.


% ── polynomial interpolation ────────────────────────────────────────
\newsection
\section{Polynomiale Interpolation}
\newsection
\subsection{Trigonometrische Interpolation}
\subsubsection{Von Approximation zur Interpolation}
Wir erinnern uns daran, dass wir die Fourier-Approximation durch den Abbruch der unendlichen Fourier-Reihe erhalten, oder in anderen Worten, wir verkleinern die Limiten der Summe.

\fancyremark{DFT mit $N = 2n$ Koeffizienten an Punkten $\frac{l}{N}$ für $l = 0, 1, \ldots, N - 1$}

Der Shift ist hier gegeben durch (für $k \geq 0$ ist $\gamma_k = \hat{f}_N(k)$ und für $k < 0$ ist $\gamma_k = \hat{f}_N(N + k)$)
\begin{align*}
    f_{N - 1}(x)                 & = \sum_{k = -n}^{n - 1} \gamma_k e^{2 \pi ikx} = \sum_{k = 0}^{n - 1} \gamma_k e^{2\pi ikx} + \sum_{k = -n}^{-1} \gamma_k e^{2\pi ikx} \\
    \Leftrightarrow f_{N - 1}(x) & = \frac{1}{N} \left( \sum_{j = 0}^{N - 1} \left( f\left( \frac{j}{n} \right)
        \sum_{k = -n}^{n - 1} e^{2\pi ik \left( x - \frac{j}{N} \right)} \right) \right)
\end{align*}

\vspace{-1pc}

Wenn wir die Funktion nun an der Stelle $\frac{l}{N}$ auswerten so erhalten wir:
\rmvspace
\begin{align*}
    f_{N - 1}\left( \frac{l}{N} \right) = \ldots = f\left( \frac{l}{N} \right)
\end{align*}

\vspace{-1.8pc}
was aufgrund der Orthogonalität der diskreten Fourier-Vektoren funktioniert, welche besagt, dass $\displaystyle \sum_{k = -n}^{n - 1} \omega_N^{k(j - l)} = 0$, für alle $j \neq l$.
Für $j = l$ ergibt die Summe $N$.

Dies heisst also, dass die Fourier-Approximation die Interpolationsbedingungen an den Punkten $\frac{l}{N}$ erfüllt,
also können wir die Lösung der Interpolationsaufgabe $p_{N - 1} \left( \frac{l}{N} \right) = f\left( \frac{l}{N} \right)$ f $l = 0, 1, \ldots, N - 1$ im Raum
\rmvspace
\begin{align*}
    \mathcal{T}_N = \text{span}\{ e^{2\pi ijt} \divides j = - \floor{\frac{N - 1}{2}}, \ldots, \floor{\frac{N}{2}} \}
\end{align*}

\rmvspace\rmvspace
folgendermassen finden können: 
\begin{enumerate}[label=(\arabic*)]
    \item Mittels Gleichungssystem $\sum_{j} \gamma_j e^{2\pi ijt_l} = f(t_l)$ für $l = 0, \ldots, N - 1$. Operationen: $\tco{N^3}$
    \item Mittels FFT in $\tco{N \log(N)}$ Operationen, aber nur falls die Punkte äquidistant sind, also $t_l = \frac{l}{N}$.
        Dann ist die Matrix des obigen Gleichungssystems $F^{-1}_N$
\end{enumerate}

\vspace{0.2cm}

Unten findet sich Python code der mit den unterschiedlichen Methoden die Koeffizienten des Trigonometrischen Polynoms bestimmt.
\rmvspace
\begin{code}{python}
    def get_coeff_trig_poly(t: np.ndarray, y: np.ndarray):
    N = y.shape[0]
    if N % 2 == 1:
    n = (N - 1.0) / 2.0
    M = np.exp(2 * np.pi * 1j * np.outer(t, np.arange(-n, n + 1)))
    else:
    n = N / 2.0
    M = np.exp(2 * np.pi * 1j * np.outer(t, np.arange(-n, n)))
    c = np.linalg.solve(M, y)
    return c

    N = 2**12
    t = np.linspace(0, 1, N, endpoint=False)
    y = np.random.rand(N)
    direct = get_coeff_trig_poly(t, y)
    using_fft = np.fft.fftshift(np.fft.fft(y) / N)
    using_ifft = np.conj(np.fft.fftshift(np.fft.ifft(y)))
\end{code}

\subsection{Monombasis}

\fancytheorem{Eindeutigkeit} $p(x) \in \mathcal(P)_k$ ist durch $k+1$ Punkte $y_i = p(x_i)$ eindeutig bestimmt.

Dieser Satz kann direkt angewendet werden zur Interpolation, in dem man $p(x)$ als Gleichungssystem schreibt.
% FIXME: It'd probably be better to use align* environment in general, it's much more flexible
% FIXME: Having a new line before $$ (or align* environment for that matter) makes the space between text and math env larger!
$$
	p_n(x) = \alpha_n x^n + \cdots + \alpha_0 x^0 \quad \iff \quad
	\underbrace{
		\begin{bmatrix}
			1      & x_0    & \cdots & x_0^n  \\
			1      & x_1    & \cdots & x_1^n  \\
			\vdots & \vdots & \ddots & \vdots \\
			1      & x_n    & \cdots & x_n^n  \\
		\end{bmatrix}
	}_\text{Vandermonde Matrix}
	\begin{bmatrix}
		\alpha_0 \\
		\alpha_1 \\
		\vdots   \\
		\alpha_n
	\end{bmatrix}
	=
	\begin{bmatrix}
		y_0    \\
		y_1    \\
		\vdots \\
		y_n
	\end{bmatrix}
$$

Um $\alpha_i$ zu finden ist die Vandermonde Matrix unbrauchbar, da die Matrix schlecht konditioniert ist.

Zur Auswertung von $p(x)$ kann man direkt die Matrix-darstellung nutzen, oder effizienter:

\fancydef{Horner Schema} $p(x) = (x \ldots x ( x (\alpha_n x + \alpha_{n-1}) + \ldots + \alpha_1) + \alpha_0)$

\fhlc{Cyan}{In NumPy} \verb|polyfit| liefert die direkte Auswertung, \verb|polyval| wertet Polynome via Horner-Schema aus. (Gemäss Script, in der Praxis sind diese Funktionen \verb|deprecated|)

\newpage
\subsection{Newton Basis}
% Session: Herleitung unwichtig, konzentrieren auf Funktion/Eigenschaften von Newton/Lagrange.

Die Newton-Basis hat den Vorteil, dass sie leichter erweiterbar als die Monombasis ist.

Die Konstruktion verläuft iterativ, und vorherige Datenpunkte müssen nicht neuberechnet werden.

\begin{align*}
    p_0(x) &= y_0 &\textit{(Anfang: triviales Polynom)} \\
    p_1(x) &= p_0(x) + (x-x_0)\frac{(y_1-y_0)}{(x_1-x_0)} & \textit{(Addition des zweiten Datenpunktes)} \\
    p_2(x) &= p_1(x) + \frac{\frac{(y_2-y_1)}{(x_2-x_1)}-\frac{(y_1-y_0)}{x_1-x_0}}{x_2-x_0} (x-x_0)(x-x_1) & \textit{(Schema lässt sich beliebig weiterführen)}\\
    p_3(x) &= p_2(x) + \ldots
\end{align*}

\setcounter{all}{3}
\fancytheorem{Newton-Basis} $\{ N_0,\ \ldots\ ,N_n\}$ ist eine Basis von $\mathcal{P}_n$
\begin{align*}
    N_0(x) &:= 1 \quad
    N_1(x) := x - x_0 \quad
    N_2(x) := (x-x_0)(x-x_1) \quad \ldots \\
    N_n(x) &:= \prod_{i=0}^{n-1} (x-x_i)
\end{align*}

\subsubsection{Koeffizienten}

Wegen Satz 2.2.3 lässt sich jedes $p_n \in \mathcal{P}_n$ als $p_n(x) =\displaystyle\sum_{i=0}^{n} \beta_i N_i(x)$ darstellen. Ein Gleichungssystem liefert alle $\beta_i$:
\begin{align*}
    \begin{bmatrix}
        1 & 0   & \cdots & 0 \\
        1 & N_0 & \cdots & 0 \\
        \vdots & \vdots & \ddots & \vdots \\
        1 & N_0 & \cdots & N_n
    \end{bmatrix}
    \begin{bmatrix}
        \beta_0 \\
        \beta_1 \\
        \vdots \\
        \beta_n
    \end{bmatrix}
    =
    \begin{bmatrix}
        y_0 \\
        y_1 \\
        \vdots \\
        y_n
    \end{bmatrix}
\end{align*}

Die Matrixmultiplikation in $\mathcal{O}(n^3)$ ist aber nicht nötig: Es gibt ein effizienteres System. 

\setcounter{all}{5}
\fancydef{Dividierte Differenzen} 
\begin{multicols}{2}
    \begin{align*}
        y[x_i] &:= y_i \\
        y[x_i,\ \ldots\ ,x_{i+k}] &\overset{\text{Rec.}}{:=} \frac{y[x_{i+1},\ \ldots\ , x_{i+k}] - y[x_i,\ \ldots\ , x_{i+k-1}]}{x_{i+k}-x_i}
    \end{align*}

    \newcolumn

    \begin{center}
    \begin{tabular}{l|llll}
        $x_0$ & $y[x_0]$ \\
            &          & $>y[x_0,x_1]$        \\
        $x_1$ & $y[x_1]$ &  & $>y[x_0,x_1,x_2]$ \\ 
            &          & $>y[x_1,x_2]$        \\
        $x_2$ & $y[x_2]$ &  & $>y[x_1,x_2,x_3]$ \\
            &          & $>y[x_2,x_3]$        \\
        $x_3$ & $y[x_3]$                        \\
    \end{tabular}
    \end{center}
\end{multicols}

\fancyremark{Äquidistante Stellen}

Falls $x_j = x_0 + \underbrace{j \cdot h}_{:= \Delta^j}$ gilt vereinfacht sich einiges:
\begin{align*}
    y[x_0,x_1] &= \frac{1}{h}\Delta y_0 \\
    y[x_0,x_1,x_2] &= \frac{1}{2!h} \Delta^2 y_0 \\
    y[x_0,\ \ldots\ , x_n] &= \frac{1}{n! h^n} \Delta^n y_0 
\end{align*}

\setcounter{all}{8}
\fancytheorem{Newton} Falls $\beta_j = y[x_0,\ \ldots\ , x_j]$ geht das resultierende Polynom durch alle $(x_i,y_i)$.\\
\footnotesize
(D.h. die dividierten Differenzen sind korrekt.)
\normalsize


\newpage
\begin{multicols}{2}

    Matrixmultiplikation in $\mathcal{O}(n^3)$, Speicher $\mathcal{O}(n^2)$

    \begin{code}{python}
    # Slow matrix approach
    def divdiff_slow(x,y):
        n = y.size
        T = np.zeros((n,n))
        T[:,0] = y

        for l in range(1,n):
            for i in range(n-l):
                T[i, l] = (T[i+1,l-1] - T[i, l-1])
                T[i, l] /= (x[i+l] - x[i])
        
        return T[0,:]
    \end{code}

    \newcolumn

    % Add the vectorized HW solution here

    Vektorisierter Ansatz in $\mathcal{O}(n^2)$, Speicher $\mathcal{O}(n)$

    \begin{code}{python}
    # Fast vectorized approach
    def divdiff_fast(x,y):
        n = y.shape[0]

        for k in range(1, n):
            y[k:] = (y[k:] - y[(k-1):n-1]) 
            y[k:] /= (x[k:] - x[0:n-k])
    
        return y
    \end{code}

\end{multicols}

\subsubsection{Auswertung}

Auswertung eines Newton-Polynoms funktioniert in $\mathcal{O}(n)$ durch ein modifiziertes Horner-Schema:

\begin{multicols}{2}
    
\begin{align*}
    p_0 &:= \beta_n \\
    p_1 &:= (x - x_{n-1})p_0 + \beta_{n-1} \\
    p_2 &:= (x - x_{n-2})p_1 + \beta_{n-2} \\
    \vdots \\
    p_n &= p(x)
\end{align*}

\newcolumn

\begin{code}{python}
    def evalNewton(x_data, beta, x):
        p = np.zeros(x.shape[0])
        p += beta[beta.shape[0]-1]
    
        for i in range(1, n+1):
            p = (x - x_data[n-i])*p + beta[n-i]
      
        return p
\end{code}

\end{multicols}


\subsubsection{Fehler}

\setcounter{all}{11}
\inlinetheorem $f$ $n$-mal diff.-bar, $y_i = f(x_i) \implies \exists \xi \in (\min_i x_i, \max_i x_i)$ s.d. $y[x_0,x_1,\ldots,x_n] = \frac{f^{(n)}(\xi)}{(n+1)!}$

\fancytheorem{Fehler} $f: [a,b] \to \R$ ist $(n+1)$-mal diff.-bar, $p$ ist das Polynom zu $f$ in $x_0,\ldots,x_n \in [a,b]$. 
$$
    \forall x \in [a,b]\ \exists \xi \in (a,b):\quad\quad \underbrace{f(x)-p(x)}_{\text{Fehler}} = \prod_{i=0}^{n}(x-x_i)\cdot\frac{f^{(n+1)}(\xi)}{(n+1)!}
$$

Man bemerke: Die Wahl der Stützpunkte hat direkten Einfluss auf den Fehler.



\subsection{Baryzentrische Formel}
Session: Gemäss TA sehr gut beschrieben im alten Script

\newsection
\subsection{Chebychev Interpolation}

Session: Chebyshev Pol. : Abzisse = Extrema, Knoten = Nullstellen

Lecture: Orthogonalität ist eine wichtige Eigenschaft: Siehe Lecture notes (handgeschr.) für Veranschaulichung. \\
$\rightarrow$ Orth. liefert die Koeff. ohne Rechenaufwand.

Lecture: Clenshaw-Alg. relativ zentral (Taschenrechner nutzen diesen intern)



% ── trigonometric interpolation ─────────────────────────────────────
\newsection
\section{Trigonometrische Interpolation}
\newsection
\subsection{Trigonometrische Interpolation}
\subsubsection{Von Approximation zur Interpolation}
Wir erinnern uns daran, dass wir die Fourier-Approximation durch den Abbruch der unendlichen Fourier-Reihe erhalten, oder in anderen Worten, wir verkleinern die Limiten der Summe.

\fancyremark{DFT mit $N = 2n$ Koeffizienten an Punkten $\frac{l}{N}$ für $l = 0, 1, \ldots, N - 1$}

Der Shift ist hier gegeben durch (für $k \geq 0$ ist $\gamma_k = \hat{f}_N(k)$ und für $k < 0$ ist $\gamma_k = \hat{f}_N(N + k)$)
\begin{align*}
    f_{N - 1}(x)                 & = \sum_{k = -n}^{n - 1} \gamma_k e^{2 \pi ikx} = \sum_{k = 0}^{n - 1} \gamma_k e^{2\pi ikx} + \sum_{k = -n}^{-1} \gamma_k e^{2\pi ikx} \\
    \Leftrightarrow f_{N - 1}(x) & = \frac{1}{N} \left( \sum_{j = 0}^{N - 1} \left( f\left( \frac{j}{n} \right)
        \sum_{k = -n}^{n - 1} e^{2\pi ik \left( x - \frac{j}{N} \right)} \right) \right)
\end{align*}

\vspace{-1pc}

Wenn wir die Funktion nun an der Stelle $\frac{l}{N}$ auswerten so erhalten wir:
\rmvspace
\begin{align*}
    f_{N - 1}\left( \frac{l}{N} \right) = \ldots = f\left( \frac{l}{N} \right)
\end{align*}

\vspace{-1.8pc}
was aufgrund der Orthogonalität der diskreten Fourier-Vektoren funktioniert, welche besagt, dass $\displaystyle \sum_{k = -n}^{n - 1} \omega_N^{k(j - l)} = 0$, für alle $j \neq l$.
Für $j = l$ ergibt die Summe $N$.

Dies heisst also, dass die Fourier-Approximation die Interpolationsbedingungen an den Punkten $\frac{l}{N}$ erfüllt,
also können wir die Lösung der Interpolationsaufgabe $p_{N - 1} \left( \frac{l}{N} \right) = f\left( \frac{l}{N} \right)$ f $l = 0, 1, \ldots, N - 1$ im Raum
\rmvspace
\begin{align*}
    \mathcal{T}_N = \text{span}\{ e^{2\pi ijt} \divides j = - \floor{\frac{N - 1}{2}}, \ldots, \floor{\frac{N}{2}} \}
\end{align*}

\rmvspace\rmvspace
folgendermassen finden können: 
\begin{enumerate}[label=(\arabic*)]
    \item Mittels Gleichungssystem $\sum_{j} \gamma_j e^{2\pi ijt_l} = f(t_l)$ für $l = 0, \ldots, N - 1$. Operationen: $\tco{N^3}$
    \item Mittels FFT in $\tco{N \log(N)}$ Operationen, aber nur falls die Punkte äquidistant sind, also $t_l = \frac{l}{N}$.
        Dann ist die Matrix des obigen Gleichungssystems $F^{-1}_N$
\end{enumerate}

\vspace{0.2cm}

Unten findet sich Python code der mit den unterschiedlichen Methoden die Koeffizienten des Trigonometrischen Polynoms bestimmt.
\rmvspace
\begin{code}{python}
    def get_coeff_trig_poly(t: np.ndarray, y: np.ndarray):
    N = y.shape[0]
    if N % 2 == 1:
    n = (N - 1.0) / 2.0
    M = np.exp(2 * np.pi * 1j * np.outer(t, np.arange(-n, n + 1)))
    else:
    n = N / 2.0
    M = np.exp(2 * np.pi * 1j * np.outer(t, np.arange(-n, n)))
    c = np.linalg.solve(M, y)
    return c

    N = 2**12
    t = np.linspace(0, 1, N, endpoint=False)
    y = np.random.rand(N)
    direct = get_coeff_trig_poly(t, y)
    using_fft = np.fft.fftshift(np.fft.fft(y) / N)
    using_ifft = np.conj(np.fft.fftshift(np.fft.ifft(y)))
\end{code}



\end{document}
