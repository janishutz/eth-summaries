\newsection
\section{Computability}
\stepcounter{subsection}
\subsection{Diagonalization}
The \bi{set of binary encodings of all TMs} is denoted $\text{KodTM}$ and $\text{KodTM} \subseteq \wordbool$ and the upper bound of the cardinality is $|\wordbool|$,
as there are infinitely many TMs.

Below is a list of countable objects. They all have corresponding Lemmas in the script, but omitted here:
\drmvspace
\begin{multicols}{4}
    \begin{itemize}
        \item $\word$ for any $\Sigma$
        \item $\text{KodTM}$
        \item $\N \times \N$
        \item $\Q^+$
    \end{itemize}
\end{multicols}

\rmvspace
\drmvspace
The following objects are uncountable: $[0, 1]$, $\R$, $\cP(\wordbool)$

\inlinecorollary $|\text{KodTM}| < |\cP(\wordbool)|$ and thus there exist infinitely many not recursively enumerable languages over $\alphabetbool$

\fhlc{Cyan}{Proof of $L$ (not) recursively enumerable}

Proving that a language \textit{is} recursively enumerable is as difficult as providing a Turing Machine that accepts it.

Proving that a language is \textit{not} recursively enumerable is likely easier. For it, let $d_{ij} = 1 \Longleftrightarrow M_i$ accepts $w_j$.

\inlineex Assume towards contradiction that $L_\text{diag} \in \cL_{RE}$. Let
\rmvspace
\begin{align*}
    L_{\text{diag}} & = \{ w \in \wordbool \divides w = w_i \text{ for an } i \in \N - \{ 0 \} \text{ and $M_i$ does not accept } w_i \} \\
                    & = \{ w \in \wordbool \divides w = w_i \text{ for an } i \in \N - \{ 0 \} \text{ and } d_{ii} = 0\}
\end{align*}
Thus assume that, $L_\text{diag} = L(M)$ for a Turing Machine $M$.
Since $M$ is a Turing Machine in the canonical ordering of all Turing Machines, so there exists an $i \in \N - \{ 0 \}$, such that $M = M_i$.

This however leads to a contradiction, as $w_i \in L_\text{diag} \Longleftrightarrow d_{ii} = 0 \Longleftrightarrow w_i \notin L(M_i)$.

In other words, $w_i$ is in $L_\text{diag}$ if and only if $w_i$ is not in $L(M_i)$, which contradicts our statement above, in which we assumed that $L_\text{diag} \in \cL_{RE}$

In other, more different, words, $w_i$ being in $L_\text{diag}$ implies (from the definition) that $d_{ii} = 0$, which from its definition implies that $w_i \notin L(M_i)$

\setLabelNumber{theorem}{3}
\inlinetheorem $L_\text{diag} \notin \cL_{RE}$

% ────────────────────────────────────────────────────────────────────


\subsection{Reductions}
This is the start of the topics that are part of the endterm.

First off, a list of important languages for this and the next section:
\begin{itemize}
    \item $L_U = \{ \text{Kod}(M)\# w \divides w \in \wordbool \text{ and TM $M$ accepts } w \}$ ($\in \cL_{RE}$, but $\notin \cL_R$)
    \item $L_H = \{ \text{Kod}(M)\# x \divides x \in \wordbool \text{ and TM $M$ halts on } x \}$ ($\in \cL_{RE}$, but $\notin \cL_R$)
    \item $L_{\text{diag}} = \{ w \in \wordbool \divides w = w_i \text{ for an } i \in \N - \{ 0 \} \text{ and $M_i$ does not accept } w_i \}$ ($\notin \cL_{RE}$ and thus $\notin \cL_R$)
    \item $(L_{\text{diag}})^C$ ($\in \cL_{RE}$, but $\notin \cL_R$)
    \item $L_{EQ} = \{ \text{Kod}(M)\# \text{Kod}(\overline{M}) \divides L(M) = L(\overline{M}) \}$ ($\in \cL_{RE}$, but $\notin \cL_R$)
    \item $\lempty = \{ \text{Kod}(M) \divides L(M) = \emptyset \}$ ($\in \cL_{RE}$, but $\notin \cL_R$)
    \item $(\lempty)^C = \{ x \in \wordbool \divides x \notin \text{Kod}(\overline{M}) \forall \text{ TM } \overline{M} \text{ or } x = \text{Kod}(M) \text{ and } L(M) \neq \emptyset \}$
          ($\in \cL_{RE}$, but $\notin \cL_R$)
    \item $L_{H, \lambda} = \{ \text{Kod}(M) \divides M \text{ halts on } \lambda \}$ ($\in \cL_{RE}$, but $\notin \cL_R$)
\end{itemize}

\setLabelNumber{theorem}{6}
\fancytheorem{Universal TM} A TM $U$, such that $L(U) = L_U$

% ────────────────────────────────────────────────────────────────────

\subsection{Rice's Theorem}
\setLabelNumber{theorem}{9}
\fancytheorem{Rice's Theorem}
