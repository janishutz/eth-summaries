\newsection
\section{Finite Automata}
\stepcounter{subsection}
\subsection{Representation}
We can note the automata using graphical notation similar to graphs or as a series of instructions like this:
\rmvspace
\begin{align*}
    \texttt{select } input & = a_1 \texttt{ goto } i_1 \\[-0.2cm]
    \vdots                                             \\
    input                  & = a_k \texttt{ goto } i_k
\end{align*}

\drmvspace
\fancydef{Finite Automaton} $A = (Q, \Sigma, \delta, q_0, F)$ with
\drmvspace
\begin{multicols}{2}
    \begin{itemize}[noitemsep]
        \item $Q$ set of states
        \item $\Sigma$ input alphabet
        \item $\delta(q, a) = p$ transition from $q$ on reading $a$ to $p$
        \item $q_0$ initial state
        \item $F \subseteq Q$ accepting states
        \item $\cL_{EA}$ regular languages (accepted by FA)
    \end{itemize}
\end{multicols}

\drmvspace
$\hat{\delta}(q_0, w) = p$ is the end state reached when we process word $w$ from state $q_0$, and $(q, w) \bigvdash{M}{*} (p, \lambda)$ is the formal definition,
with $\bigvdash{M}{*}$ representing any number of steps $\bigvdash{M}{}$ executed (transitive hull).

The class $\class[q_i]$ represents all possible words for which the FA is in this state.
Be cautious when defining them, make sure that no extra words from other classes could appear in the current class, if this is not intended.

Sometimes, we need to combine two (or more) FA to form one larger one.
We can do this easily with product automata. To create one from two automata $M_1$ (states $q_i$) and $M_2$ (states $p_j$) we do the following steps:
\rmvspace
\begin{enumerate}[noitemsep]
    \item Write down the states as tuples of the form $(q_i, p_j)$ (i.e. form a grid by writing down one of the automata vertically and the other horizontally)
    \item From each state, the automata on the horizontal axis decides for the input symbol if we move left or right,
          whereas the automata on the vertical axis decides if we move up or down.
\end{enumerate}

\begin{center}
    \begin{tikzpicture}[node distance = 1cm and 2cm, >={Stealth[round]}]
        \node[state, initial left, accepting] (q0p0) {$q_0, p_0$};
        \node[state] (q0p1) [right=of q0p0] {$q_0, p_1$};
        \node[state] (q0p2) [right=of q0p1] {$q_0, p_2$};
        \node[state, accepting] (q1p0) [below=of q0p0] {$q_1, p_0$};
        \node[state] (q1p1) [right=of q1p0] {$q_1, p_1$};
        \node[state] (q1p2) [right=of q1p1] {$q_1, p_2$};
        \node[state, accepting] (q2p0) [below=of q1p0] {$q_2, p_0$};
        \node[state, accepting] (q2p1) [right=of q2p0] {$q_2, p_1$};
        \node[state, accepting] (q2p2) [right=of q2p1] {$q_2, p_2$};

        \path[->]
        % Level 0
        (q0p0) edge node [above] {a} (q0p1)
        (q0p1) edge node [above] {a} (q0p2)
        (q0p2) edge [bend right] node [above] {a} (q0p0)
        % Level 0 to level 1
        (q0p0) edge node [right] {b} (q1p0)
        (q0p1) edge node [right] {b} (q1p1)
        (q0p2) edge node [right] {b} (q1p2)
        % Level 1 to level 2
        (q1p0) edge node [above] {a} (q2p1)
        (q1p1) edge node [above] {a} (q2p2)
        (q1p2) edge node [right, xshift=0.3cm] {a} (q2p0)
        % Level 2 to level 1
        (q2p0) edge node [right] {b} (q1p0)
        (q2p1) edge node [above left, yshift=0.1cm] {b} (q1p1)
        (q2p2) edge node [right] {b} (q1p2)
        % Level 2
        (q2p0) edge node [above] {a} (q2p1)
        (q2p1) edge node [above] {a} (q2p2)
        (q2p2) edge [bend left] node [below] {a} (q2p0)
        % ────────────────────────────────────────────────────────────────────
        % Loops on level 1
        (q1p0) edge [loop left] node {b} ()
        (q1p1) edge [loop left] node {b} ()
        (q1p2) edge [loop left] node {b} ();
    \end{tikzpicture}
\end{center}

For the automata
\begin{figure}[h!]
    \begin{subfigure}{0.49\textwidth}
        \begin{center}
            \begin{tikzpicture}[node distance = 1cm, >={Stealth[round]}]
                \node[state, initial left, accepting] (p_0) {$p_0$};
                \node[state] (p_1) [right=of p_0] {$p_1$};
                \node[state] (p_2) [right=of p_1] {$p_2$};

                \path[->]
                (p_0) edge node [above] {a} (p_1)
                (p_1) edge node [above] {a} (p_2)
                (p_2) edge [bend left] node [below] {a} (p_0)
                (p_0) edge [loop above] node {b} ()
                (p_1) edge [loop above] node {b} ()
                (p_2) edge [loop above] node {b} ();
            \end{tikzpicture}
        \end{center}
        \caption{Module to compute $|w|_b \equiv |w| (\text{mod } 3$). States $q \in Q_a$}
    \end{subfigure}
    \begin{subfigure}{0.49\textwidth}
        \begin{center}
            \begin{tikzpicture}[node distance = 1cm, >={Stealth[round]}]
                \node[state, initial left] (q_0) {$q_0$};
                \node[state] (q_1) [right=of q_0] {$q_1$};
                \node[state, accepting] (q_2) [right=of q_1] {$q_2$};

                \path[->]
                (q_0) edge node [above] {b} (q_1)
                (q_1) edge [bend left] node [above] {a} (q_2)
                (q_2) edge [bend left] node [below] {b} (q_1)
                (q_0) edge [loop above] node {a} ()
                (q_1) edge [loop above] node {b} ()
                (q_2) edge [loop above] node {a} ();
            \end{tikzpicture}
        \end{center}
        \caption{Module to compute $w$ contains sub. $ba$ and ends in $a$. States $p \in Q_b$}
    \end{subfigure}
    \caption{Graphical representation of the Finite Automaton of Task 9 in 2025}
\end{figure}



\stepcounter{subsection}
\subsection{Proofs of nonexistence}
We have three approaches to prove non-regularity of words.
Below is an informal guide as to how to do proofs using each of the methods and possible pitfalls.

For all of them start by assuming that $L$ is regular.

\fhlc{Cyan}{Lemma 3.3}
\setLabelNumber{lemma}{3}
\begin{lemma}[]{Regular words}
    Let $A$ be a FA over $\Sigma$ and let $x \neq y \in \Sigma^*$, such that $\hdelta_A (q_0, x) = \hdelta(q_0, y)$.
    Then for each $z \in \Sigma^*$ there exists an $r \in Q$, such that $xz, yz \in \class[r]$, and we thus have
    \rmvspace
    \begin{align*}
        xz \in L(A) \Longleftrightarrow yz \in L(A)
    \end{align*}
\end{lemma}
\begin{enumerate}[noitemsep]
    \item Pick a FA $A$ over $\Sigma$ and say that $L(A) = L$
    \item Pick $|Q| + 1$ words $x$ such that $xy = w \in L$ with $|y| > 0$.
    \item State that via pigeonhole principle there exists w.l.o.g $i < j \in \{ 1, \ldots, |Q| + 1 \}$, s.t. $\hdelta_A(q_0, x_i) = \hdelta_A(q_0, x_j)$.
    \item Build contradiction by picking $z$ such that $x_i z \in L$.
    \item Then, if $z$ was picked properly, since $i < j$, we have that $x_j z \notin L$, since the lengths do not match
\end{enumerate}

\rmvspace
That is a contradiction, which concludes our proof


\fhlc{Cyan}{Pumping Lemma}
\begin{lemma}[]{Pumping-Lemma für reguläre Sprachen}
    Let $L$ be regular. Then there exists a constant $n_0 \in \N$, such that each word $w \in \word$ with $|w| \geq n_0$ can be decomposed into $w = yxz$, with
    \drmvspace
    \begin{multicols}{2}
        \begin{enumerate}[label=\textit{(\roman*)}]
            \item $|yx| \leq n_0$
            \item $|x| \geq 1$
            \item For $X = \{ yx^kz \divides k\in \N \}$ \textit{either} $X \subseteq L$ or $X \cap L = \emptyset$ applies
        \end{enumerate}
    \end{multicols}
\end{lemma}

\begin{enumerate}[noitemsep]
    \item State that according to Lemma 3.4 there exists a constant $n_0$ such that $|w| \geq n_0$.
    \item Choose a word $w \in L$ that is sufficiently long to enable a sensible decomposition for the next step.
    \item Choose a decomposition, such that $|yx| = n_0$ (makes it quite easy later). Specify $y$ and $x$ in such a way that for $|y| = l$ and $|x| = m$ we have $l + m \leq n_0$
    \item According to Lemma 3.4 (ii), $m \geq 1$ and thus $|x| \geq 1$. Fix $z$ to be the suffix of $w = yxz$
    \item Then according to Lemma 3.4 (iii), fill in for $X = \{ yx^k z \divides k \in \N \}$ we have $X \subseteq L$.
    \item This will lead to a contradiction commonly when setting $k = 0$, as for a language like $0^n1^n$, we have $0^{(n_0 - m) + km}1^{n_0}$ as the word (with $n_0 - m = l$),
          which for $k = 0$ is $u= 0^{n_0 - m} 1^{n_0}$ and since $m \geq 1$, $u \notin L$ and thus by Lemma 3.4, $X \cap L = \emptyset$
\end{enumerate}


\fhlc{Cyan}{Kolmogorov Complexity}
\begin{enumerate}[noitemsep]
    \item We first need to choose an $x$ such that $L_x = \{ y \divides xy \in L \}$.
          If not immediately apparent, choosing $x = a^{\alpha + 1}$ for $a \in \Sigma$
          and $\alpha$ being the exponent of the exponent of the words in the language after a variable rename.
          For example, for $\{ 0^{n^2 + 2n} \divides n \in \N \}$, $\alpha(m) = m^2 + 2m$.
          Another common way to do this is for languages of the form $\{ a^n b^n \divides n \in \N \}$ to use $x = a^m$ and
          $L_{0^m} = \{ y \divides 0^m y \in L \} = \{ 0^j 1^{m + j} \divides j \in \N \}$.
    \item Find the first word $y_1 \in L_x$. In the first example, this word would be $y_1 = 0^{(m + 1)^2 \cdot 2(m + 1) - m^2 \cdot 2m + 1}$,
          or in general $a^{\alpha(m + 1) - \alpha(m) + 1}$.
          For the second example, the word would be $y_1 = 1^m$, i.e. with $j = 0$
    \item According to Theorem 3.1, there exists constant $c$ such that $K(y_k) \leq \ceil{\log_2(k + 1)} + c$. We often choose $k = 1$,
          so we have $K(y_1) \leq \ceil{\log_2(1 + 1)} + c = 1 + c$ and with $d = 1 + c$, $K(y_1) \leq d$
    \item This however leads to a contradiction, since the number of programs with length $\leq d$ is at most $2^d$ and thus finite, but our set $L_x$ is infinite.
\end{enumerate}


\newpage
\fhlc{Cyan}{Minimum number of states}

To show that a language needs \textit{at least} $n$ states, use Lemma 3.3 and $n$ words. We thus again do a proof by contradiction:
\begin{enumerate}
    \item Assume that there exists FA with $|Q| < n$. We now choose $n$ words (as short as possible), as we would for non-regularity proofs using Lemma 3.3 (i.e. find some prefixes).
        It is usually beneficial to choose prefixes with $|w|$ small (consider just one letter, $\lambda$, then two and more letter words)
    \item Construct a table for the suffixes using the $n$ chosen words such that one of the words at entry $x_{ij}$ is in the language and the other is not. ($n \times n$ matrix, see below in example)
    \item Conclude that we have reached a contradiction as every field $x_{ij}$ contains a suffix such that one of the two words is in the language and the other one is not.
\end{enumerate}
\inlineex Let $L = \{ x1y \divides x \in \wordbool, y \in \{ 0, 1 \}^2 \}$. Show that any FA that accepts $L$ needs at least four states.

Assume for contradiction that there exists EA $A = (Q, \alphabetbool, \delta_A, q_0, F)$ with $|Q| < 4$.
Let's take the $4$ words $00, 01, 10, 11$. Then according to Lemma 3.3, there needs to exist a $z$ such that $xz \in L(A) \Longleftrightarrow yz \in L(A)$
with $\hdelta_A(q_0, x) = \hdelta_A(q_0, y)$ for $x, y \in \{ 00, 01, 10, 11 \}$.

This however is a contradiction, as we can find a $z$ for each of the pairs $(x, y)$, such that $xz \in L(A)$, but $yz \notin L(A)$. 
See for reference the below table (it contains suffixes $z$ fulfilling prior condition):

\begin{tables}{c|cccc}{ & $00$ & $01$ & $10$ & $11$}
              $00$      & -    & $00$ & $0$  & $0$   \\
              $01$      &      & -    & $0$  & $0$   \\
              $10$      &      &      & -    & $00$   \\
              $11$      &      &      &      & -     \\
\end{tables}
Thus, all four words have to lay in pairwise distinct states and we thus need at least $4$ states to detect this language.




\subsection{Non-determinism}
The most notable differences between deterministic and non-deterministic FA is that the transition function maps is different: $\delta: Q \times \Sigma \rightarrow \cP(Q)$.
I.e., there can be any number of transitions for one symbol from $\Sigma$ from each state.
This is (in graphical notation) represented by arrows that have the same label going to different nodes.

It is also possible for there to not be a transition function for a certain element of the input alphabet.
In that case, regardless of state, the NFA rejects, as it ``gets stuck'' in a state and can't finish processing.

Additionally, the NFA accepts $x$ if it has at least one accepting calculation on $x$.

\stepLabelNumber{theorem}
\inlinetheorem For every NFA $M$ there exists a FA $A$ such that $L(M) = L(A)$. They are then called \bi{equivalent}


\fhlc{Cyan}{Potenzmengenkonstruktion}
States are no now sets of states of the NFA in which the NFA could be in after processing the preceding input elements and we have a special state called $q_{\text{trash}}$.

For each state, the set of states $P = \hdelta(q_0, z)$ for $|z| = n$ represents all possible states that the NFA could be in after doing the first $n$ calculations.

Correspondingly, we add new states if there is no other state that is in the same branch of the calculation tree $\cB_M(x)$.
So, in other words, we execute BFS on the calculation tree.
