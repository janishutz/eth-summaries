\documentclass{article}

\newcommand{\dir}{~/projects/latex}
\input{\dir/include.tex}
\load{recommended}

\setupCheatSheet{Analysis II}

\begin{document}
\maketitle
\usetcolorboxes
\setcounter{numberingConfig}{3}
\setcounter{numberSubsections}{1}


%          ╭────────────────────────────────────────────────╮
%          │                   Title page                   │
%          ╰────────────────────────────────────────────────╯
\vspace{2cm}
\begin{Huge}
    \begin{center}
        TITLE PAGE COMING SOON
    \end{center}
\end{Huge}


\vspace{4cm}
\begin{center}
    \begin{Large}
        ``\textit{Some funny quote from the lecture still needed}''
    \end{Large}

    \hspace{3cm} - Özlem Imamoglu, 2025
\end{center}

\vspace{3cm}
\begin{center}
    HS2025, ETHZ\\[0.2cm]
    \begin{Large}
        Cheat-Sheet based on Lecture notes and Script
    \end{Large}\\[0.2cm]

    \url{https://metaphor.ethz.ch/x/2025/hs/401-0213-16L/sc/script-analysis-II.pdf} 
\end{center}


\newpage
\printtoc{Cyan}


\newpage
\section{Introduction}
This Cheat-Sheet does not serve as a replacement for solving exercises and getting familiar with the content.
There is no guarantee that the content is 100\% accurate, so use at your own risk. 
If you discover any errors, please open an issue or fix the issue yourself and then open a Pull Request here:

\url{https://github.com/janishutz/eth-summaries}

This Cheat-Sheet was designed with the HS2025 page limit of 10 A4 pages in mind. 
Thus, the whole Cheat-Sheet can be printed full-sized, if you exclude the title page, contents and this page.
You could also print it as two A5 pages per A4 page and also print the 
\color{MidnightBlue}\fbox{\href{https://github.com/janishutz/eth-summaries/blob/master/semester2/analysis-i/cheat-sheet.pdf}{Analysis I summary}}\color{black} 
\smallhspace in the same manner, allowing you to bring both to the exam



%          ╭────────────────────────────────────────────────╮
%          │                    Content                     │
%          ╰────────────────────────────────────────────────╯
\newsection
\section{Differential Equations}
\stepcounter{subsection}
\subsection{Continuity}
\compactdef{Convergence in $\R^n$} Let $(x_k)_{k \in \N}$ where $x_k \in \R^n$ with $x_k = (x_{k, 1}, \ldots, x_{k, n})$ and let $y = (y_1, \ldots, y_n) \in \R^n$.
$(x_k)$ converges to $y$ as $k \rightarrow +\infty$ if $\forall \varepsilon > 0 \smallhspace \exists N \geq 1$ s.t. $\forall n \geq N$ we have $||x_k - y|| < \varepsilon$
% ────────────────────────────────────────────────────────────────────
\shortlemma $(x_k)$ converges to $y$ as $k \rightarrow +\infty$ iff one of following equiv. statements holds:
\bi{(1)} $\forall 1 \leq i \leq n$, the sequence $(x_{k, i})$ with $x_{k, i} \in \R$ converges to $y_i$
\bi{(2)} $(||x_k - y||)$ converges to $0$ as $k \rightarrow +\infty$
% ────────────────────────────────────────────────────────────────────
\compactdef{Continuity} Let $X \subseteq \R^n$ and $f: X \rightarrow \R^m$.
\bi{(1)} Let $x_0 \in X$. $f$ continuous in $\R^n$ if $\forall \varepsilon > 0 \smallhspace \exists \delta > 0$ s.t. if $x \in X$ satisfies $||x - x_0|| < \delta$,
then $||f(x) - f(x_0)|| < \varepsilon$
\bi{(2)} $f$ continuous \textit{on} $X$ if continuous at $x_0 \smallhspace \forall x_0 \in X$
% ────────────────────────────────────────────────────────────────────
\shortproposition Let $X$ and $f$ as prev. Let $x_0 \in X$. $f$ continuous at $x_0$ iff $\forall (x_k)_{k \geq 1}$ in $X$ s.t.
$x_k \rightarrow x_0$ as $k \rightarrow +\infty$, $(f(x_k))_{k \geq 1}$ in $\R^m$ converges to $f(x)$\\
% ────────────────────────────────────────────────────────────────────
\compactdef{Limit} Let $X$, $f$ and $x_0$ as prev. and $y \in \R^m$. $f$ \textit{has limit} $y$ as $x \rightarrow x_0$ with $x \neq x_0$ if
$\forall \varepsilon > 0 \smallhspace \exists \delta > 0$ s.t. $\forall x \neq x_0 \in X, ||x - x_0|| < \delta$ we have $||f(x) - y|| < \varepsilon$.
We write $\lim_{\elementstack{x \rightarrow x_0}{x \neq x_0}} f(x) = y$
\shortremark Also possible without ass. that $x_0 \in X$
% ────────────────────────────────────────────────────────────────────
\shortproposition Let $X$, $f$, $x_0$ and $y$ as prev. We have $\lim_{\elementstack{x \rightarrow x_0}{x \neq x_0}} f(x) = y$ 
iff $\forall (x_k)$ in $X$ s.t. $x_k \rightarrow x$ as $k \rightarrow +\infty$ and $x_k \neq x_0$ $(f(x_k))$ in $\R^m$ converges to $y$
\stepLabelNumber{all}
\shortproposition Let $X \subseteq \R^n$, $y \subseteq \R^m$, $p \in \N$ and let $f: X \rightarrow Y$ and $g: Y \rightarrow \R^p$ be cont. Then $g \circ f$ is continuous




\end{document}
