\newsectionNoPB
\subsection{Taylor polynomials}
\compactdef{Taylor polynomials}
Let $f : X \rightarrow \R$ with $f \in C^k(X, \R)$ and $y \in X$. The Taylor-Polynomial of order $k$ of $f$ at $y$ is:
\vspace{-0.75pc}
\begin{align*}
    T_k f(y; x - y) = \sum_{|i| \leq k} \frac{\partial_i f(y)(x - y)^i}{i!}
\end{align*}

\drmvspace\rmvspace
% TODO: Find out what the \partial_1 notation means (likely TA notes 09)
where $i$ is a \textit{multi-index}, so:
\drmvspace
\begin{multicols}{3}
    \begin{itemize}[noitemsep]
        \item $i = (i_1, \ldots, i_n)$ (each $i_j \geq 0$)
        \item $|i| = i_1 + \ldots + i_n$
        \item $\partial_i = \partial_1^{i_1} \ldots \partial_n^{i_n}$
        \item $(x - y)^i = (x_1 - y_1)^{i_1} \cdot \ldots \cdot (x_n - y_n)^{i_n}$
        \item $i! = i_1! \cdot \ldots \cdot i_n!$
    \end{itemize}
\end{multicols}

\drmvspace\rmvspace
The concept this formula uses is that we iterate through all possible partial derivatives of $f$ and assigns each a multi-index $i$.
To denote that we want to take the partial derivative $\partial_{112}$, we use $i = (2, 1, 0)$, since we take the derivative of the first variable twice,
of the second variable once and never of the third variable.
So the expression is thus now:
\mrmvspace
\begin{align*}
    \frac{\partial_{112} f(y) (x_1 - y_1)^2 (x_2 - y_2)^1 (x_3 y_3)^0}{2!1!0!} = \frac{\partial_{112} f(y) (x_1 - y_1)^2 (x_2 - y_2)}{2}
\end{align*}

\drmvspace
