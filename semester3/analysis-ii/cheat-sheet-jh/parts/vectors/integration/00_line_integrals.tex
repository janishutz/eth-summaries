\newsectionNoPB
\subsection{Line integrals}
\shortdef Let $I = [a, b]$ be a closed and bounded interval in $\R$. $f : I \rightarrow \R$ with $f(t) = (f_1(t), \ldots, f_n(t))$ continuous (also $f_i$ cont.).
\mrmvspace
\begin{enumerate}[noitemsep, label=(\arabic*)]
    \item Then $\displaystyle \int_{a}^{b} f(t) \dx t = \left( \int_{a}^{b} f_1(t), \ldots, \int_{a}^{b} f_n(t) \right)$
    \item \bi{Parametrized Curve} in $\R^n$ is a continuous map $\gamma: I \rightarrow \R^n$, piecewise in $C^1$, i.e. for $k \geq 1$, we have partition $a = t_0 < t_1 < \ldots < t_k = b$,
          such that if $f$ is restricted to interval $]t_{j - 1}, t_j$, restriction is $C^1$. $\gamma$ is a \textit{path} between $\gamma(a)$ and $\gamma(b)$
    \item \bi{Line integral} $X \subseteq \R^n$ is the image of $\gamma$ as above and $f : X \rightarrow \R^n$ cont.\\
          Integral $\displaystyle\int_{a}^{b} f(\gamma(t)) \cdot \gamma'(t) \dx t \in \R$
          is line integral of $f$ along $\gamma$, denoted $\displaystyle\int_{\gamma} f(s) \dx s$ or $\displaystyle\int_{\gamma} f(s) \dx \vec{s}$ or $\displaystyle\int_{\gamma} \omega$,\\
          with $\omega = f_1(x) \dx x_1 + \ldots f_n(x) \dx x_n$
\end{enumerate}

\rmvspace
We usually call $f : X \rightarrow \R^n$ a \bi{vector field}, which maps each point $x \in X$ to a vector in $\R^n$, displayed as originating from $x$

\setLabelNumber{all}{4}
\compactdef{Oriented reparametrization} of $\gamma$ is parametrized curve $\sigma : [c, d] \rightarrow \R^n$ s.t $\sigma = \gamma \circ \varphi$, with $\varphi : [c, d] \rightarrow I$ cont. map,
differentiable on $]a, b[$ and for which $\varphi(a) = c$ and $\varphi(b) = d$.
Conversely, $\gamma = \sigma \circ \varphi^{-1}$

\rmvspace
\shortproposition For $f : X \rightarrow \R^n$ with $X$ containing the image of of $\gamma$ and equivalently $\sigma$,
we have $\displaystyle\int_{\gamma} f(s) \cdot \dx \vec{s} = \int_{\sigma} f(s) \cdot \dx \vec{s}$

\mrmvspace
\setLabelNumber{all}{8}
\compactdef{Conservative Vector Field} If for any $x_1, x_2 \in X$ the line integral $\displaystyle\int_{\gamma} f(s) \dx \vec{s}$ is independent choice of $\gamma$ in $X$

\mrmvspace
\shortremark $f$ conservative iff $\int_{\gamma} f(s) \dx \vec{s} = 0$ for a \textit{closed} ($\gamma(a) = \gamma(b)$) parametrized curve\\
%
\shorttheorem Let $X$ be open set, $f$ conservative vector field. Then $\exists C^1$ function $g$ s.t. $f = \nabla g$.
If any two points of $X$ can be joined by a parametrized curve, then $g$ is unique up to a constant: if $\nabla g_1 = f$, then $g - g_1$ is constant on $X$

\shortremark If $f$ vec. field on $X$, then $g$ is called a \bi{potential} for $f$
