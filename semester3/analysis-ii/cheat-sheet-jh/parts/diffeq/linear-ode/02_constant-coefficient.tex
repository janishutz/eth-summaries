\newsectionNoPB
\subsection{Linear differential equations with constant coefficients}
The coefficients $a_i$ are constant functions of form $a_i(x) = k$ with $k$ constant, where $b(x)$ can be any function.\\
%
\shade{gray}{Homogeneous Equation}\rmvspace
\begin{enumerate}[noitemsep]
    \item Find \bi{characteristic polynomial} (of form $\lambda^k + a_{k - 1} \lambda^{k - 1} + \ldots + a_1 \lambda + a_0$ for order $k$ lin. ODE with coefficients $a_i \in \R$).
    \item Find the roots of polynomial. The solution space is given by $\{ C_j \cdot x^{v_j - 1} e^{\gamma_i x} \divides v_j \in \N, \gamma_i \in \R \}$
          where $v_j$ is the multiplicity of the root $\gamma_i$ and $C_j$ is a constant.
          For $\gamma_i = \alpha + \beta i \in \C$, we have $C_1 \cdot e^{\alpha x}\cos(\beta x)$, $C_2 \cdot e^{\alpha x}\sin(\beta x)$,
          representing the two complex conjugated solutions.
\end{enumerate}
\rmvspace

The homogeneous equation will then be all the elements of the set summed up.\\
\shade{gray}{Inhomogeneous Equation}\rmvspace
\begin{enumerate}[noitemsep]
    \item \bi{(Case 1)} $b(x) = c x^d e^{\alpha x}$, with special cases $x^d$ and $e^{\alpha x}$:
          $f_p = Q(x) e^{\alpha x}$ with $Q$ a polynomial with $\deg(Q) \leq j + d$,
          where $j$ is multiplicity of root $\alpha$ (if $P(\alpha) \neq 0$, then $j = 0$) of characteristic polynomial $P$
    \item \bi{(Case 2)} $b(x) = c x^d \cos(\alpha x)$, or $b(x) = c x^d \sin(\alpha x)$:
          $f_p = Q_1(x) \cdot \cos(\alpha_1 x) + Q_2(x) \cdot \sin(\alpha_2 x))$,
          where $Q_i(x)$ a polynomial with $\deg(Q_i) \leq d + j$,
          where $j$ is the multiplicity of root $\alpha_i$ (if $P(\alpha_i) \neq 0$, then $j = 0$) of characteristic polynomial $P$
    \item \bi{(Case 3)} $b(x) = c e^{\alpha x} \cos(\beta x)$, or $b(x) = c e^{\alpha x} \sin(\beta x)$, use the Ansatz
          $Q_1(x) e^{\alpha x} \sin(\beta x) + Q_2(x) e^{\alpha x} \cos(\beta x)$, again with the same polynomial.
          Often, it is sufficent to have a polynomial of degree 0 (i.e. constant)
\end{enumerate}
\rmvspace

\hl{Often}, as polynomial $Q$ choosing a simple constant suffices.
For inhomogeneous parts with addition or subtraction, the above cases can be combined.
For two of case 2 added, only use one.
For any cases not covered, start with the same form as the inhomogeneous part has (for trigonometric functions, duplicate it with both $\sin$ and $\cos$).

\rmvspace\shade{gray}{Other methods}\rmvspace
\begin{itemize}[noitemsep]
    \item \bi{Change of variable} Apply substitution method here, substituting for example for $y' = f(ax + by + c)$ $u = ax + by$ to make the integral simpler.
          Mostly intuition-based (as is the case with integration by substitution)
    \item \bi{Separation of variables} For equations of form $y' = a(y) \cdot b(x)$ (Note: Not linear),
          we transform into $\frac{y'}{a(y)} = b(x)$ and then integrate by substituting $y'(x) dx = dy$, changing the variable of integration.
          Solution: $A(y) = B(x) + c$, with $A = \int \frac{1}{a}$ and $B(x) = \int b(x)$.
          To get final solution, solve the above equation for $y$.
\end{itemize}
