\newpage
\subsection{Nichtdeterministische Turingmaschinen}
Die Ideen sind hier sehr ähnlich wie der Übergang zwischen deterministischen und nichtdeterministischen Endlichen Automaten.

\begin{definition}[]{Nichtdeterministische Turingmaschine (NTM)}
    \begin{scriptsize}
        Hier werden nur die wichtigsten Unterschiede aufgezeigt. Formale Definition auf Seiten 113ff. (= Seiten 127ff im PDF) im Buch.
    \end{scriptsize}

    Die Übergangsfunktion geht wieder in die Potenzmenge, also gilt:
    \rmvspace
    \begin{align*}
        \delta : (Q - \{ \qacc, \qrej \}) \times \Gamma \rightarrow \mathcal{P}(Q \times \Gamma \times \{ L, R, N \})
    \end{align*}

    \rmvspace
    und $\delta(p, \cent) \subseteq (\{ (q, \cent, X) \divides q \in Q, X \in \{R, N\} \})$\\

    Die von der NTM $M$ akzeptierte Sprache ist:
    \rmvspace
    \begin{align*}
        L(M) = \{ w \in \word \divides q_0\cent w \bigvdash{M}{*}y\qacc z \text{ für irgendwelche } y, z \in \Gamma^* \}
    \end{align*}
\end{definition}
Ein gutes Beispiel für eine NTM findet sich auf Seiten 114ff. im Buch (= Seite 128ff. im PDF)


\begin{definition}[]{Berechnungsbaum}
    Ein Berechnungsbaum $T_{M, x}$ von $M$ (eine NTM) auf $x$ (Wort aus Eingabealphabet von $M$) ist ein (potentiell un)gerichteter Baum mit einer Wurzel:
    \begin{enumerate}[label=\textit{(\roman*)}]
        \item Jeder Knoten von $T_{M, x}$ ist mit einer Konfiguration beschriftet
        \item Die Wurzel ist der einzige Knoten mit $\deg_{\text{in}}(v) = 0$, ist die Startkonfiguration $q_0\cent x$
        \item Jeder mit $C$ beschriftete Knoten hat genauso viele Kinder wie $C$ Nachfolgekonfigurationen hat und die Kinder sind mit diesen Nachfolgekonfigurationen markiert.
    \end{enumerate}
\end{definition}
Diese Bäume können natürlich auch für nichtdeterministischen MTM verwendet werden.

Im Vergleich zu den Berechnungsbäumen von NEA sind die Bäume von NTM nicht immer endlich.

\inlinetheorem Sei $M$ eine NTM. Dann existiert eine TM $A$, so dass $L(M) = L(A)$ 
und falls $M$ keine unendlichen Berechnungen auf Wörtern aus $(L(M))^C$ hat, dann hält $A$ immer.

\inlineproof Auf Seite 117 im Buch (= 131 im PDF). Die Idee zur Umwandlung von $M$ in die TM $A$ ist, dass $A$ Breitensuche im Berechnungsbaum von $M$ durchführt.
