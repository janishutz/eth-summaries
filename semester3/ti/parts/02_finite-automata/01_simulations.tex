% Starting P63 = P78
\newpage
\subsection{Simulationen}
Der Begriff der Simulation ist nicht ein formalisiert, da er je nach Fachgebiet, eine etwas andere Definition hat.
Die engste Definition fordert, dass jeder elementare Schritt der zu Berechnung, welche simuliert wird, durch eine Berechnung in der Simulation nachgemacht wird.
Eine etwas schwächere Forderung legt fest, dass in der Simulation auch mehrere Schritte verwendet werden dürfen.

Es gibt auch eine allgemeinere Definition, die besagt, dass nur das gleiche Eingabe-Ausgabe-Verhalten gilt und der Weg, oder die Berechnungen, welche die Simulation geht, respektive durchführt, wird ignoriert, respektive wird nicht durch die Definition beschränkt.

\textit{Hier werden wir aber die enge Definition verwenden}

\inlinelemma Wir haben zwei EA $M_1 = (Q_1, \Sigma, \delta_1, q_{01}, F_1)$ und $M_2 = (Q_2, \Sigma, \delta_2, q_{02}, F_2)$, die auf dem Alphabet $\Sigma$ operieren.
Für jede Mengenoperation $\odot \in \{ \cup, \cap, - \}$ existiert ein EA $M$, so dass $L(M) = L(M_1) \odot L(M_2)$

Was dieses Lemma nun aussagt ist folgendes: Man kann einen endlichen Automaten bauen, so dass das Verhalten von zwei anderen EA im Bezug auf die Mengenoperation simuliert wird.
Ein guter, ausführlicher Beweis dieses Lemmas findet sich im Buch auf Seite 64 (= Seite 79 im PDF)

Dieses Lemma hat weitreichende Nutzen. Besonders ist es also möglich einen modularen EA zu bauen, in dem Teile davon in kleinere und einfachere EA auszulagern, die dann wiederverwendet werden können.

\stepcounter{examples}
\inlineex Dieses Beispiel im Buch ist sehr gut erklärt und findet sich auf Seiten 65, 66 \& 67 (= Seite 80, 81 \& 82 im PDF)

% TODO: Continue from page 83 (PDF)
