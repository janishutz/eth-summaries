\newpage
\subsection{Beweise der Nichtexistenz}
Im Gegensatz zum Beweis, dass eine bestimmte Klasse von Programmen (Algorithmen) ein Problem lösen kann
(was ein einfacher Existenzbeweis ist, bei welchem man eine korrekte Implementation liefern kann),
ist der Beweis, dass diese Klasse von Programmen (Algorithmen) dies nicht tun kann viel schwieriger,
da man (logischerweise) nicht für alle (undendlich vielen) Programme zeigen kann, dass sie das Problem nicht lösen.

In diesem Kurs werden wir aber vorerst nur die Klasse der endlichen Automaten behandlen, welche sehr stark eingeschränkt sind,
was diese Beweise verhältnismässig einfach macht.
Falls also ein EA $A$ für zwei unterschiedliche Wörter $x$ und $y$ im gleichen Zustand endet (also $\hdelta(q_0, x) = \hdelta(q_0, y))$),
so heisst das für uns von jetzt an, dass $A$ nicht zwischen $x$ und $y$ unterscheiden kann:

\begin{lemma}[]{Unterscheidung von Wörtern}
    Sei $A$ ein EA über $\Sigma$ und $x \neq y \in \Sigma^*$ so dass
    \begin{align*}
        (q_0, x) \bigvdash{A}{*} (p, \lambda) \text{ und } (q_0, y) \bigvdash{A}{*} (p, \lambda)
    \end{align*}
    für ein $p \in Q$ (also $\hdelta_A (q_0, x) = \hdelta(q_0, y) = p(x, y \in \class [p])$).
    Dann existiert für jedes $z \in \Sigma^*$ ein $r \in Q$, so dass $xz, yz \in \class[p]$, also gilt insbesondere
    \begin{align*}
        xz \in L(A) \Longleftrightarrow yz \in L(A)
    \end{align*}
\end{lemma}
Das obenstehende Lemma 3.3 ist ein Spezialfall einer Eigenschaft, die für jedes (deterministische) Rechnermodell gilt.
Es besagt eigentlich nichts anderes, als dass wenn das Wort $xz$ akzeptiert wird, so wird auch das Wort $yz$

Mithilfe von Lemma 3.3 kann man für viele Sprachen deren Nichtregularität beweisen.

\inlineex Sei $L = \{ 0^n1^n \divides n \in \N \}$.
Intuitiv ist diese Sprache Nichtregulär, da $n$ undendlich gross sein kann, aber ein EA logischerweise endlich ist.
Wir müssen hier nur formal ausdrücken, dass das Zählen benötigt wird, dass $L$ akzeptiert wird:

Dazu benutzen wir einen indirekten Beweis. Sei $A$ ein EA über $\alphabets{bool}$ und $L(A) = L$.
Wir betrachten die Wörter $0^1, 0^2, \ldots, 0^{|Q| + 1}$.
Weil wir $|Q| + 1$ Wörter haben, existiert $i, j \in \{ 1, 2, \ldots, |Q| + 1 \}$, so dass $\hdelta_A(q_0, 0^i) = \hdelta_A(q_0, 0^j)$,
also gilt nach Lemma $0^i z \in L \Leftrightarrow 0^j z \in L \smallhspace \forall z \in \wordbool$.
Dies gilt jedoch nicht, weil für jedes $z = 1^i$ zwar jedes $0^i 1^i \in L$ gilt, aber $0^j 1^j \notin L$


Um die Nichtregularität konkreter Sprachen zu beweisen, sucht man nach einfach verifizierbaren Eigenschaften,
denn wenn eine Sprache eine dieser Eigenschaften \textit{nicht} erfüllt, so ist sie nicht regulär.

% TODO: For Kolmogorov complexity elaborate some more, i.e. how to do proofs properly / how to derive a word more easily
% -> TA Slides explain that really well

Eine Methode zum Beweis von Aussagen $L \notin \mathcal{L}_{\text{EA}}$ nennt sich \bi{Pumping} und basiert auf folgender Idee:
Wenn für ein Wort $x$ und einen Zustand $p$ gilt, dass $(p, x) \bigvdash{A}{*} (p, \lambda)$, so gilt auch für alle $i \in \N$, dass $(p, x^i) \bigvdash{A}{*} (p, \lambda)$.
Also kann $A$ nicht zwischen $x$ und $x^i$ unterscheiden, oder in anderen Worten, wie viele $x$ er gelesen hat,
also akzeptiert $A$ entweder alle Wörter der Form $yx^iz$ (für $i \in \N$) oder keines davon

\begin{lemma}[]{Pumping-Lemma für reguläre Sprachen}
    Sei $L$ regulär. Dann existiert ein Wort $w \in \word$
\end{lemma}

Bei der Wahl von den Teilen von $w$ sollte man idealerweise einen Teil bereits gross genug zu wählen, so dass (i) zutrifft, was es nachher einfacher macht.
