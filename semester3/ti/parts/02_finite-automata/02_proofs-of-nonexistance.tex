\subsection{Beweise der Nichtexistenz}
Im Gegensatz zum Beweis, dass eine bestimmte Klasse von Programmen (Algorithmen) ein Problem lösen kann
(was ein einfacher Existenzbeweis ist, bei welchem man eine korrekte Implementation liefern kann),
ist der Beweis, dass diese Klasse von Programmen (Algorithmen) dies nicht tun kann viel schwieriger,
da man (logischerweise) nicht für alle (undendlich vielen) Programme zeigen kann, dass sie das Problem nicht lösen.

In diesem Kurs werden wir aber vorerst nur die Klasse der endlichen Automaten behandlen, welche sehr stark eingeschränkt sind,
was diese Beweise verhältnismässig einfach macht.
Falls also ein EA $A$ für zwei unterschiedliche Wörter $x$ und $y$ im gleichen Zustand endet (also $\hdelta(q_0, x) = \hdelta(q_0, y))$),
so heisst das für uns von jetzt an, dass $A$ nicht zwischen $x$ und $y$ unterscheiden kann:

\begin{lemma}[]{Unterscheidung von Wörtern}
    Sei $A$ ein EA über $\Sigma$ und $x \neq y \in \Sigma^*$ so dass
    \begin{align*}
        (q_0, x) \bigvdash{A}{*} (p, \lambda) \text{ und } (q_0, y) \bigvdash{A}{*} (p, \lambda)
    \end{align*}
    für ein $p \in Q$ (also $\hdelta_A (q_0, x) = \hdelta(q_0, y) = p(x, y \in \class [p])$).
    Dann existiert für jedes $z \in \Sigma^*$ ein $r \in Q$, so dass $xz, yz \in \class[p]$, also gilt insbesondere
    \begin{align*}
        xz \in L(A) \Longleftrightarrow yz \in L(A)
    \end{align*}
\end{lemma}
