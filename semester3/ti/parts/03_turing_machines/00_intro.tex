% Starts on Page 94 (= 108 in PDF)
\subsection{Das Modell der Turingmaschine}
Eine Turingmaschine (oft auch Turing-Maschine geschrieben) besteht informell aus
\begin{enumerate}[label=(\roman*)]
    \item einer endlichen Kontrolle, die das Programm enthält
    \item einem Arbeitsband unendlicher Länge (das es Erlaubt, im Vergleich zum EA, Daten zu speichern)
    \item einem Lese-/Schreibkopf, der sich in beide Richtungen auf dem Band bewegen kann
\end{enumerate}
Formaler:
\begin{definition}[]{Turingmaschine (TM)}
    Eine \bi{Turingmaschine} ist eine $7$-Tupel $M = (Q, \Sigma, \Gamma, \delta, q_0, \qacc, \qrej)$, wobei:
    \begin{enumerate}[label=(\roman*)]
        \item $Q$ ist die \bi{Zustandsmenge}
        \item $\Sigma$ ist das \bi{Eingabealphabet} mit $\cent, \textvisiblespace \notin \Sigma$
        \item $\Gamma$ ist das \bi{Arbeitsalphabet} mit $\Sigma \subseteq \Gamma$, $\cent, \textvisiblespace \in \Gamma$ und $\Gamma \cap Q = \emptyset$
              ($\cent$ = Startmarker, $\textvisiblespace$ = Blanksymbol)
        \item $\delta : (Q - \{ \qacc, \qrej \}) \times \Gamma \longrightarrow Q \times \Gamma \times \{ L, R, N \}$ ist die \bi{Übergangsfunktion von $M$},
              wobei $\{ L, R, N \}$ die möglichen Bewegunsrichtungen des Lese-/Schreibkopfs sind
              und hat die Eigenschaft $\delta(q, \cent) \in Q \times \{ \cent \} \times \{ R, N \}$ für alle $q \in Q$
        \item $q_0$ ist der \bi{Anfangszustand}
        \item $\qacc$ ist der \bi{akzeptierende Zustand} (genau einer in jedem $M$)
        \item $\qrej$ ist der \bi{verwerfende Zustand} (genau einer in jedem $M$)
    \end{enumerate}
    Eine \bi{Konfiguration} $C$ von $M$ ist ein Element aus $\text{Konf}(M) = \{ \cent \} \cdot \Gamma^* \cdot Q \cdot \Gamma^+ \cup Q \cdot \{ \cent \} \Gamma^+$
    (wobei $\cdot$ die Konkatenation ist)

    Eine \bi{Startkonfiguration} für ein Eingabewort $x$ ist $q_0\cent x$

    Ein \bi{Schritt von $M$} ist eine Relation $\bigvdash{M}{}$ auf der Menge der Konfigurationen, also $\bigvdash{M}{} \subseteq \text{Konf}(M) \times \text{Konf}(M)$.

    Der Rest der Definition findet sich auf Seiten 96 - 98 (= Seiten 110 - 112 im PDF)
\end{definition}
Turingmaschinen, die immer halten, repräsentieren Algorithmen, die immer terminieren und die richtige Ausgabe liefern.
Rekursive Sprachen und entscheidbare Entscheidungsprobleme sind algorithmisch erkennbar, respektive lösbar.

Es gibt auch definitionen der TM, die ohne Startmarker $\cent$ auskommen, bei denen ist das Arbeitsband in beide Richtungen unendlich.

Graphisch stellt man Turingmaschinen folgendermassen dar:
Wir haben wieder einen Graphen mit gerichteten Kanten.
Für $\delta(q, a) = (p, b, X)$ mit $q, p \in Q$, $a, b \in \Sigma$ und $X \in \{ L, R, N \}$ werden die Kanten mit folgendem Format beschriftet:
$q \rightarrow a, X$.

Mit einem TM die Sprache $\{ 0^n 1^n \divides n \in \N \}$ erkennen kann man nun, indem man jeweils das linkeste und rechteste Symbol durch ein anderes Symbol ersetzt,
beispielsweise, wenn das Eingabealphabet $\Sigma = \{ 0, 1 \}$, dann könnte man $\Gamma$ das Symbol $2$ hinzufügen, mit dem man jedes bearbeitete Symbol ersetzt.

Im Buch wird als Beispiel auf Seite 99ff (= Seite 114ff im PDF) ein komplexeres Wort gewählt, 
bei welchem ein zeichen $a \in \alphabetbool$ durch $(a, B) \in \alphabetbool \times \{ A, B \}$ ersetzt wird, 
da wir zwei Phasen haben und zwischen denen unterscheiden wollen können.

% 113
