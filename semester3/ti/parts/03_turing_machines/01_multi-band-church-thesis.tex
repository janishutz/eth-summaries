\newsection
\subsection{Mehrband-Turingmaschinen und Church'sche These}
Die Turingmaschinen sind das Standardmodell der Berechenbarkeitstheorie, aber benötigen einige Modifikationen, um wirklich geeignet zu sein
(da das Von-Neumann Modell physisch unterschiedliche CPU, Eingabemedium und Speicher für Programme und Daten fordert, aber die TM ein gemeinsames Eingabemedium und Speicher hat).

Eine $k$-Band-Turingmaschine (für $k \in \N_0$) hat folgende Komponenten:
\begin{itemize}
    \item eine endliche Kontrolle (= Programm)
    \item ein endliches Band mit einem Lesekopf
    \item $k$ Arbeitsbänder, jedes mit eigenem Lese-/Schreibkopf
\end{itemize}
Zu Beginn ist die MTM in folgender Situation:
\begin{itemize}
    \item Das Eingabeband enthält $\cent w \$$, wobei $\cent$ und $\$$ die linke / rechte Seite der Eingabe markieren
    \item Der Lesekopf des Eingabebands zeigt auf $\cent$
    \item Alle Arbeitsbänder beinhalten $\cent \text{\textvisiblespace\textvisiblespace} \ldots$ und deren Lese-/Schreibköpfe zeigen auf $\cent$
    \item Die endliche Kontrolle ist im Anfangszustand $q_0$
\end{itemize}
Alle $k + 1$ Köpfe dürfen sich während der Berechnung in beide Richtungen bewegen (solange das nicht out-of-bounds geht).
Zudem darf der Lesekopf nicht schreiben, also beleibt der Inhalt des Eingabebands gleich.

Gleich wie bei einer TM ist das Arbeitsalphabet der Arbeitsbänder $\Gamma$ und alle Felder der Arbeitsbänder sind von links nach rechts nummeriert, wobei $0$ bei $\cent$ liegt.

Eine Konfiguration einer $k$-Band-TM $M$ ist $(q, w, i, u_1, i_1, u_2, i_2, \ldots, u_k, i_k) \in Q \times \word \times \N \times (\Gamma^* \times \N)^k$,
wobei $q$ der Zustand ist, der Inhalt des Eingabebands ist $\cent w \$$, der Lesekopf zeigt auf das $i$-te Feld,
für $j \in \{ 1, 2, \ldots, k \}$ ist der Inhalt des $j$-ten Bandes $\cent u_k \text{\textvisiblespace} \ldots$ und $i_j \leq |u_j|$ ist die Position des Feldes.

Ein Berechnungsschritt von $M$ kann mit
\rmvspace
\begin{align*}
    \delta: Q \times (\Sigma \cup \{ \cent, \$ \}) \times \Gamma^k \rightarrow Q \times \{ L, R, N \} \times (\Gamma \times \{ L, R, N \})^k
\end{align*}

\drmvspace
dargestellt werden, wobei die Argumente $(q, a, b_1, \ldots, b_k)$ der aktuelle Zustand $q$, das gelesene Eingabesymbol $a$ und die $k$ Symbole $b_i \in \Gamma$,
auf welchen die Köpfe der Arbeitsbänder stehen.

Die Eingabe $w$ wird von $M$ akzeptiert, falls $M$ den Zustand $\qacc$ erreicht und falls $M$ den Zustand $\qrej$ erreicht oder nicht terminiert, wird die Eingabe verworfen.

\vspace{0.3cm}
\hrule
\vspace{0.2cm}

Wir sagen, dass eine Maschine $A$ äquivalent zu einer Maschine $B$ ist, falls für jede Eingabe $x \in \wordbool$ gilt:
$A \text{ $<$property$>$ } x \Longleftrightarrow B \text{ $<$property$>$ } x$ mit
$<$property$> \ \in \{ \text{akzeptiert}, \text{ verwrift}, \text{ arbeitet unendlich lange auf} \}$, also ist $L(A) = L(B)$

\inlinelemma Zu jeder TM $A$ existiert eine zu $A$ äquivalente $1$-Band-TM $B$

\inlinelemma Zu jeder Mehrband-Turingmaschine $A$ existiert eine zu $A$ äquivalente TM $B$

Die Beweise dazu finden sich auf Seite 107, beziehungsweise Seite 109 (= 121 \& 123 im PDF).

In diesem Kurs müssen wir glücklicherweise meist nicht Beweise der Äquivalenz durchführen, wie auch nicht dass die TM die gewünschte Tätigkeit realisiert.

\inlinedef Zwei Maschinenmodelle (Maschinenklassen) $\mathcal{A}$ und $\mathcal{B}$ sind äquivalent wenn beides zutrifft:
\begin{enumerate}[label=(\roman*)]
    \item für jede Maschine $A \in \mathcal{A}$ eine zu $A$ äquivalente Maschine $B \in \mathcal{B}$ existiert
    \item für jede Maschine $C \in \mathcal{B}$ eine zu $C$ äquivalente Maschine $D \in \mathcal{A}$ existiert
\end{enumerate}

\inlinetheorem Die Maschinenmodelle von Turingmaschinen und Mehrband-Turingmaschinen sind äquivalent

\inlineproof Impliziert von Lemmas \ref{lemma:4-1} und \ref{lemma:4-2}

Um zu beweisen, dass Turing-Maschinen äquivalent zu höheren Programmiersprachen sind argumentiert man über die Existenz eines Interpreters für TM.


% ────────────────────────────────────────────────────────────────────
\subsubsection{Church'sche These}
\begin{center}
    \fbox{
        \parbox{16cm}{
            \textit{Die Turingmaschinen sind die Formalisierung des Begriffes ``Algorithmus'', das heisst,
                die Klasse der rekursiven Sprachen (der entscheidbaren Entscheidungsproblem)
                stimmt mit der Klasse der algorithmisch (automatisch) erkennbaren Sprache überein
            }
        }
    }
\end{center}
Die These ist nicht beweisbar, da dazu der Begriff des Algorithmus formalisiert werden müsste, was er bekanntlich nicht ist.

Dies führt zu einer interessanten Situation, in welcher es \textit{theoretisch} möglich wäre, dass jemand ein stärkeres Modell findet, als die TM sind,
eines nämlich, welches Entscheidungsprobleme lösen kann, die die TM nicht kann.

Wir nehmen also (wie in vielen Bereichen der Physik (die Relativitätstheorie ist ein gutes Beispiel) und Mathematik) und postulieren sie als Axiom.

\shade{Orange}{Fun fact} Die Church'sche These ist das einzige informatikspezifische Axiom.
