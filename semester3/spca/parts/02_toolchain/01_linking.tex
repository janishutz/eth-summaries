\subsection{Linking}

Linking is the final step in the compilation pipeline: separately compiled object files are combined into an executable.

The advantages of using Linkers are clear:
\begin{enumerate}
    \item \textbf{Separate Compilation}: Changing one source file requires only recompiling that file.
    \item \textbf{Space Optimization}: Executable code only contains functions (e.g. from libraries) that are actually used.
\end{enumerate}

\subsubsection{Symbol Resolution}

The first step during Linking is Symbol Resolution.

In the context of Linking, all variables and functions are considered \textit{Symbols}. Compilers store all symbol definitions in a \textit{Symbol Table}.
The linker associates symbol references with \textit{exactly one} definition.

\inlinedef \textbf{Symbol types}

\begin{itemize}
    \item \textbf{Global Symbols} can be referenced by other modules (e.g. \texttt{non-static} in \texttt{C})
    \item \textbf{External Symbols} are referenced globals defined elsewhere
    \item \textbf{Local Symbols} are defined and referenced exclusively in one module (e.g. \texttt{static} in \texttt{C})
\end{itemize}

Note: Local linker symbols and local program variables are \textit{not} the same.

\inlinedef \textbf{Symbol strength}

Duplicate symbols either lead to linking errors (\texttt{-fno-common}, the default) or compile (\texttt{-fcommon})

\begin{itemize}
    \item \textbf{Strong Symbols} are procedure names and initialized globals
    \item \textbf{Weak Symbols} are uninitialized globals (on \texttt{-fcommon})
\end{itemize}

in \texttt{C}, function symbols can explicitly be declared weak using:
\begin{minted}{C}
    #pragma weak func
    __attribute__((weak))__ func()
\end{minted}

\content{Duplicate Handling} The linker uses these definitions to handle duplicates:

\begin{enumerate}
    \item Given multiple strong symbols are illegal
    \item Given a strong symbol and multiple weak symbols, pick the strong symbol
    \item Given multiple weak symbols, choose an \textit{arbitrary} one
\end{enumerate}

\subsubsection{Relocation}

The second step during Linking is Relocation.

Code and data sections of separate sources are combined, and symbols are relocated from relative locations (in \texttt{.o} files) to absolute locations (in \texttt{.exe} files)
