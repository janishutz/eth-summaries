The \texttt{GNU Compiler Collection}, or short (which is also it's executable name) \texttt{gcc},
is a \lC\ compiler toolchain that is commonly used to compile \lC\ code on UNIX platforms.

It includes all the necessary tools to compile a \lC\ program:
\begin{itemize}
    \item A preprocessor, called \texttt{cpp}
    \item The actual \lC\ compiler, called \texttt{cc} (though on the slides it states it is called \texttt{cc1}, but at least in the Arch Linux package, \texttt{cc1} is not a thing)
    \item An \texttt{x86} assembler, called \texttt{as}
    \item A static linker, called \texttt{ld}
\end{itemize}
However, the individual parts are usually not called individually, but using the toolchain command \texttt{gcc}
(and that is usually again abstracted away using CMake or Make, which is in turn commonly called via a build system like Meson to automatically build packages for distribution).

\texttt{gcc} has (as of \texttt{GCC 15.2.1 20260103} on Arch Linux) about 1000 CLI arguments that can be passed.
Below is a list of the most important flags that can be passed, as discussed in the lectures:
\begin{table}[h!]
    \begin{tables}{ll}{Flag                    & Description}
              \texttt{-E}                  & Stop after the preprocessor (output is a \texttt{.i} file)                             \\
              \texttt{-S}                  & Stop after the compiler (output is assembly in \texttt{.s} file)                       \\
              \texttt{-c}                  & Stop after the assembler (output is \texttt{.o} file)                                  \\
              \texttt{-o}                  & Specify the executable name                                                            \\
              \texttt{-DNDEBUG}            & Removes all assert statements                                                          \\
              \texttt{-OX}                 & Optimization level where \texttt{X} can be one of \texttt{0, 1, 2, 3}                  \\
              \texttt{-g}                  & Compile with debugging information                                                     \\
              \texttt{-Wall}               & Enable common warnings                                                                 \\
              \texttt{-Wextra}             & Enable more warnings                                                                   \\
              \texttt{-Werror}             & Makes all warnings errors                                                              \\
              \texttt{-march=XXX}          & Optimize for the architecture (can be e.g. \texttt{native}, \texttt{x86-64-v4}, \dots) \\
              \texttt{-fno-tree-vectorize} & Do not vectorize code (\texttt{-O3} commonly enables vectorization)                    \\
    \end{tables}
    \caption{Command line flags for GCC}
    \label{tab:gcc-flags}
\end{table}
