\subsubsection{Synchronization}
As we have outlined, sequential consistency may not be desirable when trying to build a high-performance system.
We thus may want to relax sequential consistency.
A primary reason to this is out-of-order execution giving a massive speed boost, as we do not have to wait for slow memory accesses to finish
and can already compute what we have ready.
Luckily, there are plenty of ways to work around this. For example, we can make later writes bypass earlier writes, later reads bypass earlier writes,
break write atomicity (i.e. the order is not fixed anymore) or we cannot make any ordering guarantees at all.

\texttt{x86} introduces specific instructions for synchronizing, for example \texttt{lfence} (load fence), \texttt{sfence} (store fence), \texttt{mfence} (memory fence) and others.
It is typical for an \texttt{x86-64} processor to implement relaxed sequential consistency and this is sometimes referred to as Total Store Ordering (TSO).

As a general rule, the weaker the consistency model is, the quicker it runs and the cheaper it is in hardware.
However too week a consistency model and some algorithms will simply stop working correctly.

\inlinedef \textbf{Barriers / Fences} are synonyms and are used to stop either the compiler or the CPU from reordering instructions or statements for order-critical operations.

\content{Compiler barriers} If we use \texttt{gcc}, we can use the following compiler intrinsic to stop it from reordering visible loads and stores:
\mint{c}|__asm__ __volatile __ ("" ::: "memory");|

The above intrinsic will also apply a memory barrier, which in terms of assembly code will use the \texttt{mfence} instruction.
This instruction stops the CPU reordering past it, i.e. any instruction before the fence cannot happen after one behind the fence.
However, any instructions past the fence are fair game and can be reordered (i.e. two instructions behind the fence can be reordered).

If we only need this for stores or loads, we can use \texttt{lfence} or \texttt{sfence}, respectively.

\content{TAS} (Test-and-Set)

\content{CAS} (Compare-and-Swap)
