\newpage 
\subsubsection{Operations}

\content{Multiplication} is straightforward, all $3$ parameters can be operated on separately:
$$
    (-1)^{s_1}M_1 \cdot 2^{E_1} \cdot (-1)^{s_2} M_2 \cdot 2^{E_2} \quad = \quad (-1)^{s_1 \oplus s_2} (M_1 \cdot M_2) 2^{E_1 + E_2} 
$$
\textbf{Post-Normalization}: 
\begin{enumerate}
    \item If $M \geq 2$, shift $M$ right and increment $E$
    \item If $E$ out of range, overflow (set to $\infty$)
    \item Round $M$ to fit desired precision.
\end{enumerate}

\content{Addition} is more complicated: (Assumption: $E_1 \geq E_2$)
$$
    (-1)^{s_1}M_1 \cdot 2^{E_1} + (-1)^{s_2} M_2 \cdot 2^{E_2} \quad = \quad (-1)^{s'} M' \cdot 2^{E_1}
$$
$s', M'$ are the result of a signed align \& add.\\
This means $(-1)^{s_1}M_1$ is shifted left by $E_1-E_2$, and then $(-1)^{s_2}M_2$ is added.

\textbf{Post-Normalization}:
\begin{enumerate}
    \item if $M \geq 2$, shift $M$ right, increment $E$
    \item if $M \leq 1$, shift $M$ left $k$, decrement $E$ by $k$
    \item Overflow $E$ if out of range (set to $\infty$)
    \item Round $M$ to desired precision
\end{enumerate}
