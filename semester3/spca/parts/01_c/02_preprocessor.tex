\newpage
\subsection{The C preprocessor}
To have \texttt{gcc} stop compiliation after running through \texttt{cpp}, the \texttt{C preprocessor}, use \texttt{gcc -E <file name>}.

Imports in \lC\ are handled by the preprocessor, that for each \verb|#include <file1.h>|, the preprocessor simply copies the contents of the file recursively into one file.

Depending on if we use \verb|#include <file1.h>| or \verb|#include "file1.h"| the preprocessor will search for the file either in the system headers or in the project directory.
Be wary of including files twice, as the preprocessor will recursively include all files (i.e. it will include files from the files we included)

The \lC\ preprocessor gives us what are called \texttt{preprocessor macros}, which have the format \verb|#define NAME SUBSTITUTION|.
\rmvspace

\inputcodewithfilename{c}{}{code-examples/00_c/01_preprocessor/00_macros.c}

To avoid issues with semicolons at the end of preprocessor macros that wrap statements that cannot end in semicolons, we can use a concept called semicolon swallowing.
For that, we wrap the statements in a \texttt{do \dots\ while(0)} loop, which is removed by the compiler on compile, also taking with it the semicolon.

There are also a number of predefined macros:
\begin{itemize}[noitemsep]
    \item \verb|__FILE__|: Filename of processed file
    \item \verb|__LINE__|: Line number of this usage of macro
    \item \verb|__DATE__|: Date of processing
    \item \verb|__TIME__|: Time of processing
    \item \verb|__STDC__|: Set if ANSI Standard \lC\ compiler is used
    \item \verb|__STDC_VERSION__|: The version of Standard \lC\ being compiled
    \item \dots many more
\end{itemize}
In headers, we typically use \verb|#ifndef __FILENAME_H_| followed by a \verb|#define __FILENAME_H_| or the like to check if the header was already included before
