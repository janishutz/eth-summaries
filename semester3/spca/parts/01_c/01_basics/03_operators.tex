\newpage
\subsubsection{Operators}
The list of operators in \lC\ is similar to the one of \texttt{Java}, etc.
In Table \ref{tab:c-operators}, you can see an overview of the operators, sorted by precedence in descending order.
You may notice that the \verb|&| and \verb|*| operators appear twice. The higher precedence occurrence is the address operator and dereference, respectively,
and the lower precedence is \texttt{bitwise and} and \texttt{multiplication}, respectively.

Additionally, \verb|+| and \verb|-| also appear twice, the higher precedence one being a unary plus or minus, respectively, which is used to denote positive or negative numbers,
and it can also be used to do a sort-of assertion that we have an arithmetic type using the preprocessor macro
\mint{c}|#define CHECK_ARITHMETIC(x) (+(x))|
which will cause a compiler error if \texttt{x} is for example a pointer.
Of course, the lower precedence \verb|+| and \verb|-| is addition and subtraction, respectively.

Very low precedence belongs to boolean operators \verb|&&| and \texttt{||}, as well as the ternary operator and assignment operators
\begin{table}[h!]
    \begin{tables}{ll}{Operator                                  & Associativity}
              \texttt{() [] -> .}                            & Left-to-right                \\
              \verb|! ~ ++ -- + - * & (type) sizeof|         & Right-to-left  \\
              \verb|* / %|                                   & Left-to-right  \\
              \verb|+ -|                                     & Left-to-right  \\
              \verb|<< >>|                                   & Left-to-right  \\
              \verb|< <= >= >|                               & Left-to-right  \\
              \verb|== !=|                                   & Left-to-right  \\
              \verb|&| (logical and)                         & Left-to-right  \\
              \verb|^| (logical xor)                         & Left-to-right  \\
              \texttt{|} (logical or)                        & Left-to-right                \\
              \verb|&&| (boolean and)                        & Left-to-right  \\
              \texttt{||} (boolean or)                       & Left-to-right                \\
              \texttt{? :} (ternary)                         & Right-to-left                \\
              \verb|= += -= *= /= %= &= ^=||\verb|= <<= >>=| & Right-to-left  \\
              \verb|,|                                       & Left-to-right  \\
    \end{tables}
    \caption{\lC\ operators ordered in descending order by precedence}
    \label{tab:c-operators}
\end{table}

\shade{blue}{Associativity} 
\begin{itemize}
    \item Left-to-right: $A + B + C \mapsto (A + B) + C$
    \item Right-to-left: \texttt{A += B += C} $\mapsto$ \texttt{(A += B) += C}
\end{itemize}

As it should be, boolean and, as well as boolean or, support early termination.

The ternary operator works as in other programming languages \verb|result = expr ? res_true : res_false;|

As previously touched on, every statement is also an expression, i.e. the following works
\mint{c}|printf("%s", x = foo(y)); // prints output of foo(y) and x has that value|

Pre-increment (\texttt{++i}, new value returned) and post-increment (\texttt{i++}, old value returned) are also supported by \lC.

\lC\ has an \texttt{assert} statement, but do not use it for error handling. The basic syntax is \texttt{assert( expr );}
