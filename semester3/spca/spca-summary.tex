\documentclass{article}

\input{~/projects/latex/old/janishutz-helpers-v1.tex}

\renewcommand{\authorTitle}{Robin Bacher, Janis Hutz\\\url{https://github.com/janishutz/eth-summaries}}
\renewcommand{\authorHeaders}{Robin Bacher, Janis Hutz}
\usepackage{tikz}
\usetikzlibrary{positioning, arrows.meta, shapes.geometric}

\setup{Systems Programming and Computer Architecture}

\usepackage{lmodern}
\setFontType{sans}

\usepackage{multirow}

\newcommand{\lC}{\texttt{C}}
\newcommand{\content}[1]{\shade{blue}{#1}}
\newcommand{\warn}[1]{\bg{orange}{#1}}
\newcommand{\danger}[1]{\shade{red}{#1}}

\setNumberingStyle{0}

\begin{document}
\startDocument
\usetcolorboxes


\vspace{2cm}
\begin{center}
    \begin{Large}
        Rust and the like have an \texttt{unsafe} block... \lC's equivalent to this is
    \end{Large}
    \begin{code}{c}
        int main( int argc, char *argv[] ) {
            // Unsafe code goes here
        }
    \end{code}
    \begin{Large}
        i.e. \bi{YOU are the one that makes \lC\ code safe!}
    \end{Large}
\end{center}


\vspace{4cm}
\begin{center}
    \begin{Large}
        ``\textit{If you are using CMake to solve the exercises... First off, sorry that you like CMake}``
    \end{Large}

    \hspace{3cm} - Timothy Roscoe, 2025
\end{center}

\vspace{3cm}
\begin{center}
    HS2025, ETHZ\\[0.2cm]
    \begin{Large}
        Summary of the Lectures and Lecture Slides
    \end{Large}\\[0.2cm]
\end{center}


%          ╭────────────────────────────────────────────────╮
%          │                     Quotes                     │
%          ╰────────────────────────────────────────────────╯
\newpage
\hspace{1cm}
\begin{center}
    \begin{Huge}
        \shade{ForestGreen}{Quotes}
    \end{Huge}

    \begin{Large}
        ``\textit{An LLM is a lossy index over human statements}``
    \end{Large}

    \hspace{3cm} - Professor Buhmann, Date unknown

    \begin{Large}
        ``\textit{If you are using CMake to solve the exercises... First off, sorry that you like CMake}``

        ``\textit{You can't have a refrigerator behave like multiple refrigerators}`` % 10:55 16.09: TODO: Get exact wording

        ``\textit{Why is C++ called C++ and not ++C? It's like you don't get any value and then it's incremented, which is true}`` % 11:54 17.09: TODO: Get exact wording
    \end{Large}

    \hspace{3cm} - Timothy Roscoe, 2025
\end{center}



%          ╭────────────────────────────────────────────────╮
%          │               Table of contents                │
%          ╰────────────────────────────────────────────────╯
\newpage
\printtoc{magenta}


%          ╭────────────────────────────────────────────────╮
%          │                    Content                     │
%          ╰────────────────────────────────────────────────╯
% ── Intro to x86 asm ────────────────────────────────────────────────
\newsection
\section{Introduction}
This summary tries to summarize everything that is important to know for this course.
It aims to be a full replacement for the slides, but as with all my summaries, there may be missing or incorrect information in here,
so use at your own risk. You have been warned!

The summary does \textit{not} follow the order the lecture does.
This is to make related information appear more closely to each other than they have in the lecture and the summary assumes you have already seen
the concepts in the lectures or elsewhere (or are willing to be thrown in the deep end).

The target semester for this summary is HS2025, so there might have been changes in your year.
If there are changes and you'd like to update this summary, please open a pull request in the summary's repo at

\begin{center}
    \hlurl{https://github.com/janishutz/eth-summaries}
\end{center}


% ── x86 assembly ────────────────────────────────────────────────────
\newpage
\section{x86 Assembly}
\newsection
\subsection{Trigonometrische Interpolation}
\subsubsection{Von Approximation zur Interpolation}
Wir erinnern uns daran, dass wir die Fourier-Approximation durch den Abbruch der unendlichen Fourier-Reihe erhalten, oder in anderen Worten, wir verkleinern die Limiten der Summe.

\fancyremark{DFT mit $N = 2n$ Koeffizienten an Punkten $\frac{l}{N}$ für $l = 0, 1, \ldots, N - 1$}

Der Shift ist hier gegeben durch (für $k \geq 0$ ist $\gamma_k = \hat{f}_N(k)$ und für $k < 0$ ist $\gamma_k = \hat{f}_N(N + k)$)
\begin{align*}
    f_{N - 1}(x)                 & = \sum_{k = -n}^{n - 1} \gamma_k e^{2 \pi ikx} = \sum_{k = 0}^{n - 1} \gamma_k e^{2\pi ikx} + \sum_{k = -n}^{-1} \gamma_k e^{2\pi ikx} \\
    \Leftrightarrow f_{N - 1}(x) & = \frac{1}{N} \left( \sum_{j = 0}^{N - 1} \left( f\left( \frac{j}{n} \right)
        \sum_{k = -n}^{n - 1} e^{2\pi ik \left( x - \frac{j}{N} \right)} \right) \right)
\end{align*}

\vspace{-1pc}

Wenn wir die Funktion nun an der Stelle $\frac{l}{N}$ auswerten so erhalten wir:
\rmvspace
\begin{align*}
    f_{N - 1}\left( \frac{l}{N} \right) = \ldots = f\left( \frac{l}{N} \right)
\end{align*}

\vspace{-1.8pc}
was aufgrund der Orthogonalität der diskreten Fourier-Vektoren funktioniert, welche besagt, dass $\displaystyle \sum_{k = -n}^{n - 1} \omega_N^{k(j - l)} = 0$, für alle $j \neq l$.
Für $j = l$ ergibt die Summe $N$.

Dies heisst also, dass die Fourier-Approximation die Interpolationsbedingungen an den Punkten $\frac{l}{N}$ erfüllt,
also können wir die Lösung der Interpolationsaufgabe $p_{N - 1} \left( \frac{l}{N} \right) = f\left( \frac{l}{N} \right)$ f $l = 0, 1, \ldots, N - 1$ im Raum
\rmvspace
\begin{align*}
    \mathcal{T}_N = \text{span}\{ e^{2\pi ijt} \divides j = - \floor{\frac{N - 1}{2}}, \ldots, \floor{\frac{N}{2}} \}
\end{align*}

\rmvspace\rmvspace
folgendermassen finden können: 
\begin{enumerate}[label=(\arabic*)]
    \item Mittels Gleichungssystem $\sum_{j} \gamma_j e^{2\pi ijt_l} = f(t_l)$ für $l = 0, \ldots, N - 1$. Operationen: $\tco{N^3}$
    \item Mittels FFT in $\tco{N \log(N)}$ Operationen, aber nur falls die Punkte äquidistant sind, also $t_l = \frac{l}{N}$.
        Dann ist die Matrix des obigen Gleichungssystems $F^{-1}_N$
\end{enumerate}

\vspace{0.2cm}

Unten findet sich Python code der mit den unterschiedlichen Methoden die Koeffizienten des Trigonometrischen Polynoms bestimmt.
\rmvspace
\begin{code}{python}
    def get_coeff_trig_poly(t: np.ndarray, y: np.ndarray):
    N = y.shape[0]
    if N % 2 == 1:
    n = (N - 1.0) / 2.0
    M = np.exp(2 * np.pi * 1j * np.outer(t, np.arange(-n, n + 1)))
    else:
    n = N / 2.0
    M = np.exp(2 * np.pi * 1j * np.outer(t, np.arange(-n, n)))
    c = np.linalg.solve(M, y)
    return c

    N = 2**12
    t = np.linspace(0, 1, N, endpoint=False)
    y = np.random.rand(N)
    direct = get_coeff_trig_poly(t, y)
    using_fft = np.fft.fftshift(np.fft.fft(y) / N)
    using_ifft = np.conj(np.fft.fftshift(np.fft.ifft(y)))
\end{code}

\newsection
\subsection{Trigonometrische Interpolation}
\subsubsection{Von Approximation zur Interpolation}
Wir erinnern uns daran, dass wir die Fourier-Approximation durch den Abbruch der unendlichen Fourier-Reihe erhalten, oder in anderen Worten, wir verkleinern die Limiten der Summe.

\fancyremark{DFT mit $N = 2n$ Koeffizienten an Punkten $\frac{l}{N}$ für $l = 0, 1, \ldots, N - 1$}

Der Shift ist hier gegeben durch (für $k \geq 0$ ist $\gamma_k = \hat{f}_N(k)$ und für $k < 0$ ist $\gamma_k = \hat{f}_N(N + k)$)
\begin{align*}
    f_{N - 1}(x)                 & = \sum_{k = -n}^{n - 1} \gamma_k e^{2 \pi ikx} = \sum_{k = 0}^{n - 1} \gamma_k e^{2\pi ikx} + \sum_{k = -n}^{-1} \gamma_k e^{2\pi ikx} \\
    \Leftrightarrow f_{N - 1}(x) & = \frac{1}{N} \left( \sum_{j = 0}^{N - 1} \left( f\left( \frac{j}{n} \right)
        \sum_{k = -n}^{n - 1} e^{2\pi ik \left( x - \frac{j}{N} \right)} \right) \right)
\end{align*}

\vspace{-1pc}

Wenn wir die Funktion nun an der Stelle $\frac{l}{N}$ auswerten so erhalten wir:
\rmvspace
\begin{align*}
    f_{N - 1}\left( \frac{l}{N} \right) = \ldots = f\left( \frac{l}{N} \right)
\end{align*}

\vspace{-1.8pc}
was aufgrund der Orthogonalität der diskreten Fourier-Vektoren funktioniert, welche besagt, dass $\displaystyle \sum_{k = -n}^{n - 1} \omega_N^{k(j - l)} = 0$, für alle $j \neq l$.
Für $j = l$ ergibt die Summe $N$.

Dies heisst also, dass die Fourier-Approximation die Interpolationsbedingungen an den Punkten $\frac{l}{N}$ erfüllt,
also können wir die Lösung der Interpolationsaufgabe $p_{N - 1} \left( \frac{l}{N} \right) = f\left( \frac{l}{N} \right)$ f $l = 0, 1, \ldots, N - 1$ im Raum
\rmvspace
\begin{align*}
    \mathcal{T}_N = \text{span}\{ e^{2\pi ijt} \divides j = - \floor{\frac{N - 1}{2}}, \ldots, \floor{\frac{N}{2}} \}
\end{align*}

\rmvspace\rmvspace
folgendermassen finden können: 
\begin{enumerate}[label=(\arabic*)]
    \item Mittels Gleichungssystem $\sum_{j} \gamma_j e^{2\pi ijt_l} = f(t_l)$ für $l = 0, \ldots, N - 1$. Operationen: $\tco{N^3}$
    \item Mittels FFT in $\tco{N \log(N)}$ Operationen, aber nur falls die Punkte äquidistant sind, also $t_l = \frac{l}{N}$.
        Dann ist die Matrix des obigen Gleichungssystems $F^{-1}_N$
\end{enumerate}

\vspace{0.2cm}

Unten findet sich Python code der mit den unterschiedlichen Methoden die Koeffizienten des Trigonometrischen Polynoms bestimmt.
\rmvspace
\begin{code}{python}
    def get_coeff_trig_poly(t: np.ndarray, y: np.ndarray):
    N = y.shape[0]
    if N % 2 == 1:
    n = (N - 1.0) / 2.0
    M = np.exp(2 * np.pi * 1j * np.outer(t, np.arange(-n, n + 1)))
    else:
    n = N / 2.0
    M = np.exp(2 * np.pi * 1j * np.outer(t, np.arange(-n, n)))
    c = np.linalg.solve(M, y)
    return c

    N = 2**12
    t = np.linspace(0, 1, N, endpoint=False)
    y = np.random.rand(N)
    direct = get_coeff_trig_poly(t, y)
    using_fft = np.fft.fftshift(np.fft.fft(y) / N)
    using_ifft = np.conj(np.fft.fftshift(np.fft.ifft(y)))
\end{code}

\subsubsection{Registers}
\texttt{x86} assembly is a bit particular with register naming (register names all start in \%).
The initial 16-bit version of \texttt{x86} had the following registers (sub registers are registers that can be used to access the high
(\texttt{h} suffix) or low (\texttt{l} suffix) half of the register. Only registers ending in \texttt{x} feature these sub registers.
They, as well as \texttt{\%si} and \texttt{\%di} are general purpose):
\begin{tables}{lll}{Name    & Sub-registers                & Description}
              \texttt{\%ax} & \texttt{\%ah}, \texttt{\%al} & accumulate          \\
              \texttt{\%cx} & \texttt{\%ch}, \texttt{\%cl} & counter             \\
              \texttt{\%dx} & \texttt{\%dh}, \texttt{\%dl} & data                \\
              \texttt{\%bx} & \texttt{\%bh}, \texttt{\%bl} & base                \\
              \texttt{\%si} & -                            & Source index        \\
              \texttt{\%di} & -                            & Destination index   \\
              \hline
              \texttt{\%sp} & -                            & Stack pointer       \\
              \texttt{\%bp} & -                            & Base pointer        \\
              \texttt{\%ip} & -                            & Instruction pointer \\
              \texttt{\%sr} & -                            & Status (flags)      \\
\end{tables}
When the architecture was extended to 32-bit, all registers previously available were retained and a 32 bit version of each was introduced with the prefix \texttt{e}.
In other words, any 16 bit code would still work as previously, as e.g. the \texttt{\%ax} register was simply now the lower 16 bits of the \texttt{\%eax} register.

The same happened again when extending to 64-bit, only this time the \texttt{r} prefix was used.
So, the register \texttt{\$eax} was now the lower 32 bits of \texttt{\%rax}.
Additionally, the following registers are also available, with \texttt{X} to be substituted with 8 through 15: \texttt{\%rX} and the lower 32 bits \texttt{\%rXd}

\subsubsection{Instructions}
Instructions usually have a 3 letter \texttt{mnemonic} with a one letter postfix that indicates the number of bytes.
The following postfixes are available: \texttt{b} (byte, 1 byte), \texttt{w} (word, 2 bytes), \texttt{l} (long word, 4 bytes) and \texttt{q} (quad, 8 bytes).

The following options can be passed for source and destination: Registers,

\content{Immediates} To use a constant value (aka Immediate) in an instruction, we prefix the number with \texttt{\$} (following number is decimal).
To use hex, we can use \texttt{\$0x}, for binary, we can use \texttt{\$0b}, etc.

\content{Memory addresses} To treat a register as a memory address, use parenthesis, e.g. \texttt{(\%rax)} interprets the value of \texttt{\%rax} as a memory address.
The instruction will then read the number of bytes, as specified by the postfix of the instruction.

The full syntax for memory address modes is \texttt{D(Rb, Ri, S)}, where
\begin{itemize}[noitemsep]
    \item \texttt{D}: Displacement (constant offset), can be 0, 1, 2 or 4 bytes (not bits, if you are confused as I was)
    \item \texttt{Rb}: Base register (to which offsets, etc are added). Can be any of the 16 integer registers
    \item \texttt{Ri}: Index register: Any, except for \texttt{\%rsp} (and \texttt{\%rbp} is also rarely used)
    \item \texttt{S}: Scale factor (1, 2, 4 or 8, to correct offsets)
\end{itemize}
The computation that happens is the following: \texttt{Mem[ Reg[Rb] + S * Reg[Ri] + D ]}.
Using the \texttt{lea src, dest} instruction, we can get the address computed into the dest register.
Can be abused for similar arithmetic expressions.

\subsection{Data types}
Assembly supports the following integer types (where GAS stands for GNU Assembly).
If they are signed or unsigned does not matter (as we will see in section \ref{sec:c-integers}), so it's up to you to interpret them as one or the other
\begin{tables}{llll}{Intel & GAS        & Bytes & \lC}
              byte         & \texttt{b} & 1     & \texttt{[unsigned] char}  \\
              word         & \texttt{w} & 2     & \texttt{[unsigned] short} \\
              double word  & \texttt{l} & 4     & \texttt{[unsigned] int}   \\
              quad word    & \texttt{q} & 8     & \texttt{[unsigned] long}  \\
\end{tables}
These integer types are also used for pointer addresses.
Assembly also supports floating point numbers.
They are stored and operated on in floating point registers.
\begin{tables}{llll}{Intel & GAS        & Bytes & \lC}
              single       & \texttt{s} & 4     & \texttt{float}  \\
              double       & \texttt{l} & 8     & \texttt{double} \\
              extended     & \texttt{t} & 16     & \texttt{long double} \\
\end{tables}
Assembly does not support any aggregate types (such as arrays, structs, etc) natively. You can however (obviously) make your own.
In the following section we will cover how \lC\ datatypes are compiled into assembly.
Do note that the \texttt{sizeof} function in \lC\ returns the number of bytes.

\subsubsection{Arrays}
Arrays of type \texttt{T} and length \texttt{L} are allocated as a contiguous region of memory with size \texttt{L * sizeof(T)} bytes.
We then also store a reference / identifier \texttt{A} to the array (i.e. similar to variable name in \lC), that holds the address of the first
element of the array and can then be used in conjunction with ``assembly pointer arithmetic''.

Array loops that are written as for-loops in code are usually transformed into do-while loops by the compiler to save one condition check in the beginning,
except of course, it might be possible that the loop is never executed.

\subsubsection{Structures}
We again allocate a contiguous region of memory.
Only now, the number of bytes required isn't as straight forward to compute anymore, it is still not hard:
We simply sum up the sizes of all members and that will be our required sizes, so for the $n$ members $x_i$ of struct \texttt{my\_struct}, we have
$\texttt{sizeof(my\_struct)} = \sum_{i = 0}^{n - 1} \texttt{sizeof}(x_i)$.

However, the size of a struct may be different to fulfill alignment requirements set forth by the ISA or operating system.
This could mean that the struct takes \texttt{n * K} bytes, where \texttt{K} is the alignment of the largest element

For alignment on \texttt{x86-64} we have:
\mrmvspace
\begin{multicols}{2}
    \begin{itemize}[noitemsep]
        \item 1 byte (no restrictions)
        \item 2 bytes (LSB must be 0)
        \item 4 bytes (2 LSB must be 00)
        \item 8 bytes (3 LSB must be 000)
        \item 16 bytes (4 LSB must be 0000)
    \end{itemize}
\end{multicols}

\dhrmvspace
Another issue is accessing members. The solution to this is however easy and efficient, as at compile time, the offsets are pre-determined
and compiled into the setter and/or getter code for the struct.

\newpage
\subsubsection{Nested / Multidimensional arrays, Struct arrays}
All of these arrays have similar underlying concepts in the way they are allocated, yet all are a bit different

\content{Common ideas} Each of the array's elements are allocated in contiguous regions of memory, with the elements also in contiguous regions of memory.
(Imagine it as lining up all elements on a band, i.e. as going through the array in a nested loop and printing all the elements into a single line.)
The size of the array is determined by \texttt{n * sizeof(T)}, where \texttt{T} is the type of the elements of the array (or outer array).
This is what is different for the lot (as well as accessing elements):

\content{Nested array} \texttt{T} is another array. We thus have a recursive definition, where \texttt{sizeof(T)} resolves to \texttt{n * sizeof(T1)}, etc.
Accessing element $i$, $j$, $k$ is handled as follows: $o = i * \texttt{sizeof(T)} + j * \texttt{sizeof(T1)} + k * \texttt{sizeof(T2)}$,
with \texttt{T1} and \texttt{T2} the types of the nested arrays

\content{Struct arrays} \texttt{T} is a struct.


\subsubsection{Multi-Level arrays}
In comparison to multidimensional arrays, we have arrays of pointers that contain either more arrays of pointers, (normal) arrays or pointers to other data types.
The size of such a Multi-Level array is determined by:
\texttt{n * sizeof(ptr)}, where \texttt{sizeof(ptr)} is the platform-specific size of a pointer and \texttt{n} is the number of elements in the array

To do an access, we need to do two (or more) memory reads, which we can again do using address computations.

The benefit of these kinds of arrays is that we can store arbitrary data types together in an array, giving us more flexibility.

\subsubsection{Unions}
Since unions can hold any of the elements listed (but only one at a time), we allocate based on the size of the largest element.

\newsection
\subsection{Trigonometrische Interpolation}
\subsubsection{Von Approximation zur Interpolation}
Wir erinnern uns daran, dass wir die Fourier-Approximation durch den Abbruch der unendlichen Fourier-Reihe erhalten, oder in anderen Worten, wir verkleinern die Limiten der Summe.

\fancyremark{DFT mit $N = 2n$ Koeffizienten an Punkten $\frac{l}{N}$ für $l = 0, 1, \ldots, N - 1$}

Der Shift ist hier gegeben durch (für $k \geq 0$ ist $\gamma_k = \hat{f}_N(k)$ und für $k < 0$ ist $\gamma_k = \hat{f}_N(N + k)$)
\begin{align*}
    f_{N - 1}(x)                 & = \sum_{k = -n}^{n - 1} \gamma_k e^{2 \pi ikx} = \sum_{k = 0}^{n - 1} \gamma_k e^{2\pi ikx} + \sum_{k = -n}^{-1} \gamma_k e^{2\pi ikx} \\
    \Leftrightarrow f_{N - 1}(x) & = \frac{1}{N} \left( \sum_{j = 0}^{N - 1} \left( f\left( \frac{j}{n} \right)
        \sum_{k = -n}^{n - 1} e^{2\pi ik \left( x - \frac{j}{N} \right)} \right) \right)
\end{align*}

\vspace{-1pc}

Wenn wir die Funktion nun an der Stelle $\frac{l}{N}$ auswerten so erhalten wir:
\rmvspace
\begin{align*}
    f_{N - 1}\left( \frac{l}{N} \right) = \ldots = f\left( \frac{l}{N} \right)
\end{align*}

\vspace{-1.8pc}
was aufgrund der Orthogonalität der diskreten Fourier-Vektoren funktioniert, welche besagt, dass $\displaystyle \sum_{k = -n}^{n - 1} \omega_N^{k(j - l)} = 0$, für alle $j \neq l$.
Für $j = l$ ergibt die Summe $N$.

Dies heisst also, dass die Fourier-Approximation die Interpolationsbedingungen an den Punkten $\frac{l}{N}$ erfüllt,
also können wir die Lösung der Interpolationsaufgabe $p_{N - 1} \left( \frac{l}{N} \right) = f\left( \frac{l}{N} \right)$ f $l = 0, 1, \ldots, N - 1$ im Raum
\rmvspace
\begin{align*}
    \mathcal{T}_N = \text{span}\{ e^{2\pi ijt} \divides j = - \floor{\frac{N - 1}{2}}, \ldots, \floor{\frac{N}{2}} \}
\end{align*}

\rmvspace\rmvspace
folgendermassen finden können: 
\begin{enumerate}[label=(\arabic*)]
    \item Mittels Gleichungssystem $\sum_{j} \gamma_j e^{2\pi ijt_l} = f(t_l)$ für $l = 0, \ldots, N - 1$. Operationen: $\tco{N^3}$
    \item Mittels FFT in $\tco{N \log(N)}$ Operationen, aber nur falls die Punkte äquidistant sind, also $t_l = \frac{l}{N}$.
        Dann ist die Matrix des obigen Gleichungssystems $F^{-1}_N$
\end{enumerate}

\vspace{0.2cm}

Unten findet sich Python code der mit den unterschiedlichen Methoden die Koeffizienten des Trigonometrischen Polynoms bestimmt.
\rmvspace
\begin{code}{python}
    def get_coeff_trig_poly(t: np.ndarray, y: np.ndarray):
    N = y.shape[0]
    if N % 2 == 1:
    n = (N - 1.0) / 2.0
    M = np.exp(2 * np.pi * 1j * np.outer(t, np.arange(-n, n + 1)))
    else:
    n = N / 2.0
    M = np.exp(2 * np.pi * 1j * np.outer(t, np.arange(-n, n)))
    c = np.linalg.solve(M, y)
    return c

    N = 2**12
    t = np.linspace(0, 1, N, endpoint=False)
    y = np.random.rand(N)
    direct = get_coeff_trig_poly(t, y)
    using_fft = np.fft.fftshift(np.fft.fft(y) / N)
    using_ifft = np.conj(np.fft.fftshift(np.fft.ifft(y)))
\end{code}

\subsubsection{Arithmetic Operations}
Arithmetic / logic operations need a size postfix (replace \texttt{X} below with \texttt{b} (1B), \texttt{w} (2B), \texttt{l} (4B) or \texttt{q} (8B)).
\begin{tables}{lll}{Mnemonic & Format             & Computation}
              \texttt{addX}  & \texttt{Src, Dest} & \texttt{Dest} $\gets$ \texttt{Dest + Src}               \\
              \texttt{subX}  & \texttt{Src, Dest} & \texttt{Dest} $\gets$ \texttt{Dest - Src}               \\
              \texttt{imulX} & \texttt{Src, Dest} & \texttt{Dest} $\gets$ \texttt{Dest * Src}               \\
              \texttt{salX}  & \texttt{Src, Dest} & \texttt{Dest} $\gets$ \texttt{Dest << Src}              \\
              \texttt{sarX}  & \texttt{Src, Dest} & \texttt{Dest} $\gets$ \texttt{Dest >> Src} (arithmetic) \\
              \texttt{shrX}  & \texttt{Src, Dest} & \texttt{Dest} $\gets$ \texttt{Dest >> Src} (logical)    \\
              \texttt{xorX}  & \texttt{Src, Dest} & \texttt{Dest} $\gets$ \texttt{Dest \string^ Src}        \\
              \texttt{andX}  & \texttt{Src, Dest} & \texttt{Dest} $\gets$ \texttt{Dest \& Src}              \\
              \texttt{orX}   & \texttt{Src, Dest} & \texttt{Dest} $\gets$ \texttt{Dest | Src}               \\
              \texttt{incX}  & \texttt{Dest}      & \texttt{Dest} $\gets$ \texttt{Dest + 1}                 \\
              \texttt{decX}  & \texttt{Dest}      & \texttt{Dest} $\gets$ \texttt{Dest - 1}                 \\
              \texttt{negX}  & \texttt{Dest}      & \texttt{Dest} $\gets$ \texttt{-Dest}                    \\
              \texttt{notX}  & \texttt{Dest}      & \texttt{Dest} $\gets$ \texttt{\string~Dest}             \\
\end{tables}

\subsubsection{Condition Codes}
Any arithmetic operation (that is truly part of the arithmetic operations group, so not including \texttt{lea} for example) implicitly sets the \bi{condition codes}.
The following condition codes were covered in the lecture (operation: \texttt{t = a + b}):
\begin{itemize}
    \item \texttt{CF} (Carry Flag): Set if carry out from MSB (unsigned overflow)
    \item \texttt{ZF} (Zero Flag): Set if \texttt{t == 0}
    \item \texttt{SF} (Sign Flag): Set if \texttt{(a - b) < 0} (for signed)
    \item \texttt{OF} (Overflow Flag): Set if two's complement overflow (i.e. \verb+(a>0 && b>0 && t<0) || (a<0 && b<0 && t>=0)+)
\end{itemize}

\content{Explicit computation}
In the below explanations, we always assume \texttt{src2 = b} and \texttt{src1 = a}

To explicitly compute them, we can use the \texttt{cmpX src2, src1} (i.e. for easier understanding, \texttt{cmpX b, a}) instruction (with X again any of the size postfixes),
that essentially computes $(a - b)$ without setting a destination register.

When we execute that instruction, \texttt{CF} is set if \texttt{a < b} (unsigned), \texttt{ZF} is set if \texttt{a == b}, \texttt{SF} is set if \texttt{a < b} (signed)
and \texttt{OF} is set as above, where \texttt{t = a - b}.

Another instruction that is used is \texttt{testX src2, src1} (X again a size postfix, easier: \texttt{testX b, a}), and acts like computing \texttt{a \& b} and where \texttt{ZF} is set if \texttt{a \& b == 0}
and \texttt{SF} is set if \texttt{a \& b < 0}.

\content{Zeroing register} We can use a move instruction, but that is less efficient than using \texttt{xorl reg, reg},
where \texttt{reg} is the 32-bit version of the reg we want to zero. This works because on 32-bit operations,
the upper 32 bit of the 64 bit register will be zeroed.

\content{Reading condition codes} To read condition codes, we can use the \texttt{setC} instructions,
where the \texttt{C} is to be substituted by an element of table \ref{tab:condition-codes}

\subsubsection{Jumping}
To jump, use \texttt{jmp <label>} (unconditional jump) or the \texttt{jC <label>} instructions, with \texttt{C} from table \ref{tab:condition-codes}

\begin{table}[h!]
    \begin{tables}{lll}{\texttt{setC} & Condition           & Description}
              \texttt{e}          & \verb|ZF|           & Equal / Zero              \\
              \texttt{ne}         & \verb+~ZF+          & Not Equal / Not Zero      \\
              \texttt{s}          & \verb|SF|           & Negative                  \\
              \texttt{ns}         & \verb+~SF+          & Nonnegative               \\
              \texttt{g}          & \verb+~(SF^OF)&~ZF+ & Greater (signed)          \\
              \texttt{ge}         & \verb+~(SF^OF)+     & Greater or equal (signed) \\
              \texttt{l}          & \verb+SF^OF+        & Less (signed)             \\
              \texttt{le}         & \verb+(SF^OF)|ZF+   & Less or equal (signed)    \\
              \texttt{a}          & \verb+~CF&~ZF+      & Above (unsigned)          \\
              \texttt{b}          & \verb|CF|           & Below (unsigned)          \\
    \end{tables}
    \caption{Condition code postfixes for jump and set instructions}
    \label{tab:condition-codes}
\end{table}

\content{Conditional Moves}

Similar to \texttt{jC}, the same postfixes can be applied to \texttt{cmovC}, for example:

\begin{minted}{gas}
    cmpl     %eax, %edx # computes (edx - eax) without overwriting edx
    cmovle   %edx, %eax # only copies edx into eax if edx <= eax
\end{minted}

Due to the computed condition flags, this will move \verb|%edx| into \verb|%eax|, only if \verb|%edx| is less than or equal (\verb|le|) to \verb|%eax|.

This can be used to, for example, compile ternary expressions from \verb|C| to Assembly.
However, this requires evaluating both possible expressions, which may lead to a (significant) performance overhead.

\newpage
\subsection{Control Flow}

Control flow structures from \verb|C| like \verb|if/else| or \verb|for| are compiled into assembly mainly using jumps and conditional \verb|move|.

By the nature of Assembly and thanks to compilers optimizing aggressively, there is no \textit{single} definitive translation of the \verb|C| control structures: The compiler may translate it very differently depending on the context of the program.

\subsubsection{Conditional statements}
A function using an \verb|if/else| construct to choose the maximum of $2$ numbers might compile like this:

\inputcodewithfilename{gas}{code-examples/01_asm/}{02_max.s}

A function computing the absolute difference $|x-y|$ using an \verb|if/else| construct, might use a conditional \verb|move| instead:

\inputcodewithfilename{gas}{code-examples/01_asm/}{03_absdiff.s}

\subsubsection{While Loops}

A recursive factorial function using a \verb|do while| loop may be compiled like this:

\inputcodewithfilename{gas}{code-examples/01_asm/}{04_factorial.s}

\newpage

The same function, using a \verb|while| loop instead may lead to this:

\inputcodewithfilename{gas}{code-examples/01_asm/}{05_factorial.s}

\subsubsection{For Loops \& Switch}

\verb|for| loops follow the same idea as \verb|while| loops, albeit with a few more jumps.

\verb|switch| statements are implemented differently depending on size and case values: Sparse switch statements are compiled as decision trees, whereas large switch statements may become \textit{jump tables}.

\newpage
\subsection{The Stack}
In the below two, we can do this with $x = 1, 2, 4, 8$, each corresponding to a size prefix that is set with \texttt{X}

\bi{Stack push} \texttt{pushX src}: Fetch operand at \texttt{src}, decrement \texttt{\%rsp} by $x$, then writes the operand at address of \texttt{\%rsp}

\bi{Stack pop} \texttt{popX dest}: Does the opposite, reads operand at \texttt{\%rsp}, increments it by $x$, then writes the operand into \texttt{dest}

\content{Procedure call / return} Use \texttt{call LABEL}. This pushes return label to the stack and jumps to the LABEL.
After this instruction, we also may use the \texttt{pushX} instruction to store further registers.
Just remember to pop in the correct order with the correct size again!

The \texttt{ret} instruction is the return instruction and it will jump back to the caller and execution will continue there.

\subsubsection{Calling Conventions}
The callee is the function that is called and the caller is the code / function that calls the function.
\begin{itemize}
    \item \texttt{\%rax} and \texttt{\%eax} can be used without first saving (usually used as return)
    \item Argument registers are caller saved (or not if not needed again)
    \item \texttt{\%rsp} should not be modified anyway
    \item \texttt{\%rbp} is callee saved and is used as frame pointer (usually set to equal \texttt{\%rsp} at start of procedure and can be used to access elements of the frame
          (as it should not change during execution of the function and should always point to the start of the frame))
\end{itemize}

\begin{multicols}{2}
    \begin{tables}{ll}{Name      & Description}
              \texttt{\%rax} & Return value, \#variable args \\
              \texttt{\%rbx} & Base pointer, Callee saved    \\
              \texttt{\%rcx} & Argument 4                    \\
              \texttt{\%rdx} & Argument 3 (and return 2)     \\
              \texttt{\%rsi} & Argument 2                    \\
              \texttt{\%rdi} & Argument 1                    \\
              \texttt{\%rsp} & Stack pointer                 \\
              \texttt{\%rbp} & Frame pointer, Callee saved   \\
    \end{tables}
    \begin{tables}{ll}{Name      & Description}
              \texttt{\%r8}  & Argument 5                \\
              \texttt{\%r9}  & Argument 6                \\
              \texttt{\%r10} & Static chain pointer      \\
              \texttt{\%r11} & Temporary                 \\
              \texttt{\%r12} & Callee saved              \\
              \texttt{\%r13} & Callee saved              \\
              \texttt{\%r14} & Callee saved              \\
              \texttt{\%r15} & GOT pointer, callee saved \\
    \end{tables}
\end{multicols}
If we have more than 6 arguments to be passed, we can use the stack for this.
If we can do all accesses to the stack relative to the stack pointer, we do not need to
update \texttt{\%rbp} and not even \texttt{\%rbx}, or we can use it for other purposes.

We can also allocate the entire stack frame immediately by incrementing the stack pointer to the final position and then store data relative to it.
To deallocate a stack frame, simply increment the stack pointer

\newpage
\subsection{Unorthodox Control Flow}
In \lC, the \texttt{setjmp.h} header file can be included, which gives us access to \texttt{setjmp} and \texttt{longjmp}.

To use them, we first need to declare a \texttt{jmp\_buf} somewhere, usually as a static variable.

The \texttt{setjmp( jmp\_buf env } function stores the current stack / environment in the \texttt{jmp\_buf} and returns 0.

The \texttt{longjmp( jmp\_buf env, int val )} function causes a second return, which returns \texttt{val},
to the \texttt{setjmp} invocation and jumps back to that place.

\inputcodewithfilename{c}{}{code-examples/01_asm/08_unorthodox-controlflow.c}

What the above code outputs is: \texttt{second} followed by \texttt{main}.

\newpage
They are implemented in Assembly as follows. Nothing really surprising for the implementation there.
The assembly code is from the Musl \lC\ library
\inputcodewithfilename{gas}{}{code-examples/01_asm/09_setjmp-longjmp.s}

\newpage
\subsection{Coroutines}
Coroutines are functions that call each other when they are done or they need new data to work on.
An example is a decompresser that calls a parser when it has finished compressing parts of the file and that parser then again calls the decompresser when it has finished parsing.

We can implement that either by rewriting the functions into a single function, which often is a bit clumsy.
A way around this is to use \bi{Continuations}, where the first function saves its state and the context is switched to the other function.
That function can then load its state and continue where it left off, runs until it finishes its task, then saves its state and the context switches back to the original function.
\inputcodewithfilename{c}{}{code-examples/01_asm/10_coroutine.h}
\newpage
\inputcodewithfilename{c}{}{code-examples/01_asm/10_coroutine.c}

As you can see, the \texttt{setjmp.h} functions are the foundation of all concurrent programming.



% ── Intro to C ──────────────────────────────────────────────────────
\newpage
\section{The C Programming Language}
\newsection
\subsection{Trigonometrische Interpolation}
\subsubsection{Von Approximation zur Interpolation}
Wir erinnern uns daran, dass wir die Fourier-Approximation durch den Abbruch der unendlichen Fourier-Reihe erhalten, oder in anderen Worten, wir verkleinern die Limiten der Summe.

\fancyremark{DFT mit $N = 2n$ Koeffizienten an Punkten $\frac{l}{N}$ für $l = 0, 1, \ldots, N - 1$}

Der Shift ist hier gegeben durch (für $k \geq 0$ ist $\gamma_k = \hat{f}_N(k)$ und für $k < 0$ ist $\gamma_k = \hat{f}_N(N + k)$)
\begin{align*}
    f_{N - 1}(x)                 & = \sum_{k = -n}^{n - 1} \gamma_k e^{2 \pi ikx} = \sum_{k = 0}^{n - 1} \gamma_k e^{2\pi ikx} + \sum_{k = -n}^{-1} \gamma_k e^{2\pi ikx} \\
    \Leftrightarrow f_{N - 1}(x) & = \frac{1}{N} \left( \sum_{j = 0}^{N - 1} \left( f\left( \frac{j}{n} \right)
        \sum_{k = -n}^{n - 1} e^{2\pi ik \left( x - \frac{j}{N} \right)} \right) \right)
\end{align*}

\vspace{-1pc}

Wenn wir die Funktion nun an der Stelle $\frac{l}{N}$ auswerten so erhalten wir:
\rmvspace
\begin{align*}
    f_{N - 1}\left( \frac{l}{N} \right) = \ldots = f\left( \frac{l}{N} \right)
\end{align*}

\vspace{-1.8pc}
was aufgrund der Orthogonalität der diskreten Fourier-Vektoren funktioniert, welche besagt, dass $\displaystyle \sum_{k = -n}^{n - 1} \omega_N^{k(j - l)} = 0$, für alle $j \neq l$.
Für $j = l$ ergibt die Summe $N$.

Dies heisst also, dass die Fourier-Approximation die Interpolationsbedingungen an den Punkten $\frac{l}{N}$ erfüllt,
also können wir die Lösung der Interpolationsaufgabe $p_{N - 1} \left( \frac{l}{N} \right) = f\left( \frac{l}{N} \right)$ f $l = 0, 1, \ldots, N - 1$ im Raum
\rmvspace
\begin{align*}
    \mathcal{T}_N = \text{span}\{ e^{2\pi ijt} \divides j = - \floor{\frac{N - 1}{2}}, \ldots, \floor{\frac{N}{2}} \}
\end{align*}

\rmvspace\rmvspace
folgendermassen finden können: 
\begin{enumerate}[label=(\arabic*)]
    \item Mittels Gleichungssystem $\sum_{j} \gamma_j e^{2\pi ijt_l} = f(t_l)$ für $l = 0, \ldots, N - 1$. Operationen: $\tco{N^3}$
    \item Mittels FFT in $\tco{N \log(N)}$ Operationen, aber nur falls die Punkte äquidistant sind, also $t_l = \frac{l}{N}$.
        Dann ist die Matrix des obigen Gleichungssystems $F^{-1}_N$
\end{enumerate}

\vspace{0.2cm}

Unten findet sich Python code der mit den unterschiedlichen Methoden die Koeffizienten des Trigonometrischen Polynoms bestimmt.
\rmvspace
\begin{code}{python}
    def get_coeff_trig_poly(t: np.ndarray, y: np.ndarray):
    N = y.shape[0]
    if N % 2 == 1:
    n = (N - 1.0) / 2.0
    M = np.exp(2 * np.pi * 1j * np.outer(t, np.arange(-n, n + 1)))
    else:
    n = N / 2.0
    M = np.exp(2 * np.pi * 1j * np.outer(t, np.arange(-n, n)))
    c = np.linalg.solve(M, y)
    return c

    N = 2**12
    t = np.linspace(0, 1, N, endpoint=False)
    y = np.random.rand(N)
    direct = get_coeff_trig_poly(t, y)
    using_fft = np.fft.fftshift(np.fft.fft(y) / N)
    using_ifft = np.conj(np.fft.fftshift(np.fft.ifft(y)))
\end{code}

\newsection
\subsection{Trigonometrische Interpolation}
\subsubsection{Von Approximation zur Interpolation}
Wir erinnern uns daran, dass wir die Fourier-Approximation durch den Abbruch der unendlichen Fourier-Reihe erhalten, oder in anderen Worten, wir verkleinern die Limiten der Summe.

\fancyremark{DFT mit $N = 2n$ Koeffizienten an Punkten $\frac{l}{N}$ für $l = 0, 1, \ldots, N - 1$}

Der Shift ist hier gegeben durch (für $k \geq 0$ ist $\gamma_k = \hat{f}_N(k)$ und für $k < 0$ ist $\gamma_k = \hat{f}_N(N + k)$)
\begin{align*}
    f_{N - 1}(x)                 & = \sum_{k = -n}^{n - 1} \gamma_k e^{2 \pi ikx} = \sum_{k = 0}^{n - 1} \gamma_k e^{2\pi ikx} + \sum_{k = -n}^{-1} \gamma_k e^{2\pi ikx} \\
    \Leftrightarrow f_{N - 1}(x) & = \frac{1}{N} \left( \sum_{j = 0}^{N - 1} \left( f\left( \frac{j}{n} \right)
        \sum_{k = -n}^{n - 1} e^{2\pi ik \left( x - \frac{j}{N} \right)} \right) \right)
\end{align*}

\vspace{-1pc}

Wenn wir die Funktion nun an der Stelle $\frac{l}{N}$ auswerten so erhalten wir:
\rmvspace
\begin{align*}
    f_{N - 1}\left( \frac{l}{N} \right) = \ldots = f\left( \frac{l}{N} \right)
\end{align*}

\vspace{-1.8pc}
was aufgrund der Orthogonalität der diskreten Fourier-Vektoren funktioniert, welche besagt, dass $\displaystyle \sum_{k = -n}^{n - 1} \omega_N^{k(j - l)} = 0$, für alle $j \neq l$.
Für $j = l$ ergibt die Summe $N$.

Dies heisst also, dass die Fourier-Approximation die Interpolationsbedingungen an den Punkten $\frac{l}{N}$ erfüllt,
also können wir die Lösung der Interpolationsaufgabe $p_{N - 1} \left( \frac{l}{N} \right) = f\left( \frac{l}{N} \right)$ f $l = 0, 1, \ldots, N - 1$ im Raum
\rmvspace
\begin{align*}
    \mathcal{T}_N = \text{span}\{ e^{2\pi ijt} \divides j = - \floor{\frac{N - 1}{2}}, \ldots, \floor{\frac{N}{2}} \}
\end{align*}

\rmvspace\rmvspace
folgendermassen finden können: 
\begin{enumerate}[label=(\arabic*)]
    \item Mittels Gleichungssystem $\sum_{j} \gamma_j e^{2\pi ijt_l} = f(t_l)$ für $l = 0, \ldots, N - 1$. Operationen: $\tco{N^3}$
    \item Mittels FFT in $\tco{N \log(N)}$ Operationen, aber nur falls die Punkte äquidistant sind, also $t_l = \frac{l}{N}$.
        Dann ist die Matrix des obigen Gleichungssystems $F^{-1}_N$
\end{enumerate}

\vspace{0.2cm}

Unten findet sich Python code der mit den unterschiedlichen Methoden die Koeffizienten des Trigonometrischen Polynoms bestimmt.
\rmvspace
\begin{code}{python}
    def get_coeff_trig_poly(t: np.ndarray, y: np.ndarray):
    N = y.shape[0]
    if N % 2 == 1:
    n = (N - 1.0) / 2.0
    M = np.exp(2 * np.pi * 1j * np.outer(t, np.arange(-n, n + 1)))
    else:
    n = N / 2.0
    M = np.exp(2 * np.pi * 1j * np.outer(t, np.arange(-n, n)))
    c = np.linalg.solve(M, y)
    return c

    N = 2**12
    t = np.linspace(0, 1, N, endpoint=False)
    y = np.random.rand(N)
    direct = get_coeff_trig_poly(t, y)
    using_fft = np.fft.fftshift(np.fft.fft(y) / N)
    using_ifft = np.conj(np.fft.fftshift(np.fft.ifft(y)))
\end{code}

\newpage
\subsubsection{Control Flow}
Many of the control-flow structures of \texttt{C} can be found in the below code snippet.
A note of caution when using goto: It is almost never a good idea (can lead to unexpected behaviour, is hard to maintain, etc).
Where it however is very handy is for error recovery (and cleanup functions) and early termination of multiple loops (jumping out of a loop).
So, for example, if you have to run multiple functions to set something up and one of them fails,
you can jump to a label and have all cleanup code execute that you have specified there.
And because the labels are (as in Assembly) skipped over during execution, you can make very nice cleanup code.
We can also use \texttt{continue} and \texttt{break} statements similarly to \texttt{Java}, they do not however accept labels.
(Reminder: \texttt{continue} skips the loop body and goes to the next iteration)

\inputcodewithfilename{c}{}{code-examples/00_c/00_basics/01_func.c}



\newpage
\subsubsection{Declarations}
We have already seen a few examples for how \texttt{C} handles declarations.
In concept they are similar (and scoping works the same) to most other \texttt{C}-like programming languages, including \texttt{Java}.

\inputcodewithfilename{c}{}{code-examples/00_c/00_basics/02_declarations.c}

\newpage
A peculiarity of \texttt{C} is that the bit-count is not defined by the language, but rather the hardware it is compiled for.
\rmvspace

\begin{fullTable}{llll}{\texttt{C} data type & typical 32-bit & ia32  & x86-64}{Comparison of byte-sizes for each datatype on different architectures}
    \texttt{char}               & 1              & 1     & 1       \\
    \texttt{short}              & 2              & 2     & 2       \\
    \texttt{int}                & 4              & 4     & 4       \\
    \texttt{long}               & 4              & 4     & 8       \\
    \texttt{long long}          & 8              & 8     & 8       \\
    \texttt{float}              & 4              & 4     & 4       \\
    \texttt{double}             & 4              & 8     & 8       \\
    \texttt{long double}        & 8              & 10/12 & 16      \\
\end{fullTable}

\drmvspace
\warn{Type format} Be however aware that this table uses the \texttt{LP64} format for the x86-64 sizes
and this is the format all UNIX-Systems use (i.e. Linux, BSD, Darwin (the Mac Kernel)).
64 bit Windows however uses \texttt{LLP64}, i.e. \texttt{int} and \texttt{long} have the same size (32) and \texttt{long long} and pointers are 64 bit.


\content{Integers} By default, integers in \lC\ are \texttt{signed}, to declare an unsigned integer, use \texttt{unsigned int}.
Since it is hard and annoying to remember the number of bytes that are in each data type, \texttt{C99} has introduced the extended integer types,
which can be imported from \texttt{stdint.h} and are of form \texttt{int<bit count>\_t} and \texttt{uint<bit count>\_t},
where we substitute the \texttt{<bit count>} with the number of bits (have to correspond to a valid type of course).


\content{Booleans} Another notable difference of \texttt{C} compared to other languages is that \texttt{C} doesn't natively have a \texttt{boolean} type,
by convention a \texttt{short} is used to represent it, where any non-zero value means \texttt{true} and \texttt{0} means \texttt{false}.
Since boolean types are quite handy, the \texttt{!} syntax for negation turns any non-zero value of any integer type into zero and vice-versa.
\texttt{C99} has added support for a bool type via \texttt{stdbool.h}, which however is still an integer.


\content{Implicit casts} Notably, \texttt{C} doesn't have a very rigid type system and lower bit-count types are implicitly cast to higher bit-count data types, i.e.
if you add a \texttt{short} and an \texttt{int}, the \texttt{short} is cast to \texttt{short} (bits 16-31 are set to $0$) and the two are added.
Explicit casting between almost all types is also supported.
Some will force a change of bit representation, but most won't (notably, when casting to and from \texttt{float}-like types, minus to \texttt{void})


\content{Expressions} Every \lC\ statement is also an expression, see above code block for example.


\content{Void} The \texttt{void} type has \bi{no} value and is used for untyped pointers and declaring functions with no return value


\content{Structs} Are like classes in OOP, but they contain no logic.
We can assign copy a struct by assignment and they behave just like everything else in \texttt{C} when used as an argument for functions
in that they are passed by value and not by reference.
You can of course pass it also by reference (like any other data type) by setting the argument to type \texttt{struct mystruct * name} and then calling the function using
\texttt{func(\&test)} assuming \texttt{test} is the name of your struct


\content{Typedef} To define a custom type using \texttt{typedef <type it represents> <name of the new type>}.

You may also use \texttt{typedef} on structs using \texttt{typedef struct <struct tag> <name of the new alias>},
you can thus instead of e.g. \verb|struct list_el my_list;| write \verb|list my_list;|, if you have used \verb|typedef struct list_el list;| before.
It is even possible to do this:
\drmvspace
\begin{code}{c}
    typedef struct list_el {
        unsigned long val;
        struct list_el *next;
    } list_el;

    struct list_el my_list;
    list_el my_other_list;
\end{code}
\rmvspace

\content{Namespaces}
\lC\ has a few different namespaces, i.e. you can have the one of the same name in each namespace (i.e. you can have \texttt{struct a}, \texttt{int a}, etc).
The following namespaces were covered:
\rmvspace
\begin{itemize}[noitemsep]
    \item Label names (used for \texttt{goto})
    \item Tags (for \texttt{struct}, \texttt{union} and \texttt{enum})
    \item Member names one namespace for each \texttt{struct}, \texttt{union} and \texttt{enum}
    \item Everything else mostly (types, variable names, etc, including typedef)
\end{itemize}

\newpage
\subsubsection{Operators}
The list of operators in \lC\ is similar to the one of \texttt{Java}, etc.
In Table \ref{tab:c-operators}, you can see an overview of the operators, sorted by precedence in descending order.
You may notice that the \verb|&| and \verb|*| operators appear twice. The higher precedence occurrence is the address operator and dereference, respectively,
and the lower precedence is \texttt{bitwise and} and \texttt{multiplication}, respectively.

Additionally, \verb|+| and \verb|-| also appear twice, the higher precedence one being a unary plus or minus, respectively, which is used to denote positive or negative numbers,
and it can also be used to do a sort-of assertion that we have an arithmetic type using the preprocessor macro
\mint{c}|#define CHECK_ARITHMETIC(x) (+(x))|
which will cause a compiler error if \texttt{x} is for example a pointer.
Of course, the lower precedence \verb|+| and \verb|-| is addition and subtraction, respectively.

Very low precedence belongs to boolean operators \verb|&&| and \texttt{||}, as well as the ternary operator and assignment operators
\begin{table}[h!]
    \begin{tables}{ll}{Operator                                  & Associativity}
              \texttt{() [] -> .}                            & Left-to-right                \\
              \verb|! ~ ++ -- + - * & (type) sizeof|         & Right-to-left  \\
              \verb|* / %|                                   & Left-to-right  \\
              \verb|+ -|                                     & Left-to-right  \\
              \verb|<< >>|                                   & Left-to-right  \\
              \verb|< <= >= >|                               & Left-to-right  \\
              \verb|== !=|                                   & Left-to-right  \\
              \verb|&| (logical and)                         & Left-to-right  \\
              \verb|^| (logical xor)                         & Left-to-right  \\
              \texttt{|} (logical or)                        & Left-to-right                \\
              \verb|&&| (boolean and)                        & Left-to-right  \\
              \texttt{||} (boolean or)                       & Left-to-right                \\
              \texttt{? :} (ternary)                         & Right-to-left                \\
              \verb|= += -= *= /= %= &= ^=||\verb|= <<= >>=| & Right-to-left  \\
              \verb|,|                                       & Left-to-right  \\
    \end{tables}
    \caption{\lC\ operators ordered in descending order by precedence}
    \label{tab:c-operators}
\end{table}

\shade{blue}{Associativity} 
\begin{itemize}
    \item Left-to-right: $A + B + C \mapsto (A + B) + C$
    \item Right-to-left: \texttt{A += B += C} $\mapsto$ \texttt{(A += B) += C}
\end{itemize}

As it should be, boolean and, as well as boolean or, support early termination.

The ternary operator works as in other programming languages \verb|result = expr ? res_true : res_false;|

As previously touched on, every statement is also an expression, i.e. the following works
\mint{c}|printf("%s", x = foo(y)); // prints output of foo(y) and x has that value|

Pre-increment (\texttt{++i}, new value returned) and post-increment (\texttt{i++}, old value returned) are also supported by \lC.

\lC\ has an \texttt{assert} statement, but do not use it for error handling. The basic syntax is \texttt{assert( expr );}

\newpage
\subsubsection{Arrays}
\label{sec:c-arrays}
\lC\ compiler does not do any array bound checks! Thus, always check array bounds.
Unlike some other programming languages, arrays are \bi{not} dynamic length.

The below snippet includes already some pointer arithmetic tricks. The variable \texttt{data} is a pointer to the first element of the array.
\inputcodewithfilename{c}{code-examples/00_c/00_basics/}{03_arrays.c}

\subsubsection{Strings}
\lC\ doesn't have a \texttt{string} data type, but rather, strings are represented (when using \texttt{ASCII}) as \texttt{char} arrays,
with length of the array $n + 1$ (where $n$ is the number of characters of the string).
The extra element is the termination character, called the \texttt{null character}, denoted \verb|\0|.
To determine the actual length of the string (as it may be padded), we can use \verb|strnlen(str, maxlen)| from \texttt{string.h}
\inputcodewithfilename{c}{code-examples/00_c/00_basics/}{04_strings.c}

\subsubsection{Integers in C}
As a reminder, integers are encoded as follows in big endian notation, with $x_i$ being the $i$-th bit and $w$ being the number of bits used to represent the number:
\begin{itemize}[noitemsep]
    \item \bi{Unsigned}: $\displaystyle \sum_{i = 0}^{w - 1} x_i \cdot 2^i$
    \item \bi{Signed}: $\displaystyle -x_{w - 1} \cdot 2^{w - 1} + \sum_{i = 0}^{w - 1} x_i \cdot 2^i$ (two's complement notation, with $x_{w - 1}$ being the sign-bit)
\end{itemize}
The minimum number representable is $0$ and $-2^{w - 1}$, respectively, whereas the maximum number representable is $2^w - 1$ and $2^{w - 1} - 1$

We can use the shift operators to multiply and divide by two. Shift operations are usually \textit{much} cheaper than multiplication and division.

\newpage
\subsubsection{Pointers}
On loading of a program, the OS creates the virtual address space for the process, inspects the executable and loads the data to the right places in the address space,
before other preparations like final linking and relocation are done.

Stack-based languages (supporting recursion) allocate stack in frames that contain local variables, return information and temporary space.
When a procedure is entered, a stack frame is allocated and executes any necessary setup code (like moving the stack pointer, see \ref{sec:asm-stack}).
When a procedure returns, the stack frame is deallocated and any necessary cleanup code is executed, before execution of the previous frame continues.

\bi{In \lC\ a pointer is a variable whose value is the memory address of another variable}

Of note is that if you simply declare a pointer using \texttt{type * p;} you will get different memory addresses every time.
The (Linux)-Kernel randomizes the address space to prevent some common exploits.
\inputcodewithfilename{c}{}{code-examples/00_c/00_basics/05_pointers.c}

\newpage
\begin{scriptsize}
    Some pointer arithmetic has already appeared in section \ref{sec:c-arrays}, but same kind of content with better explanation can be found here
\end{scriptsize}

\content{Pointer Arithmetic} Note that when doing pointer arithmetic, adding $1$ will move the pointer by \texttt{sizeof(type)} bits.
Pointer arithmetic with a \texttt{void} pointer is thus not allowed by standard \lC, as the compiler does not know the size of the data type.
However, \texttt{gcc} does allow it and assumes the size of \texttt{void} is \texttt{1 byte}.

You may use pointer arithmetic on whatever pointer you'd like (as long as it's not a null pointer).
This means, you \textit{can} make an array wherever in memory you'd like.
The issue is just that you are likely to overwrite something, and that something might be something critical (like a stack pointer),
thus you will get \bi{undefined} behaviour! (This is by the way a common concept in \lC, if something isn't easy to make more flexible
(example for \texttt{malloc}, if you pass a pointer to memory that is not the start of the \texttt{malloc}'d section, you get undefined behaviour),
in the docs mention that one gets undefined behaviour if you do not do as it says, so\dots RTFM!)

As already seen in the section arrays (section \ref{sec:c-arrays}), we can use pointer arithmetic for accessing array elements.
The array name is treated as a pointer to the first element of the array, except when:
\begin{itemize}[noitemsep]
    \item it is operand of \texttt{sizeof} (return value is $n \cdot \texttt{sizeof(type)}$ with $n$ the number of elements)
    \item its address is taken (then \texttt{\&a == a})
    \item it is a string literal initializer. If we modify a pointer \texttt{char *b = "String";} to string literal in code,
          the \texttt{"String"} is stored in the code segment and if we modify the pointer, we get undefined behaviour
\end{itemize}
\shade{purple}{Fun fact}: \texttt{A[i]} is always rewritten \texttt{*(A + i)} by compiler.

\content{Function arguments} Another important aspect is passing by value or by reference.
You can pass every data type by reference, you can not however pass an array by value (as an array is treated as a pointer, see above).

\content{Body-less loops}
\rmvspace
\begin{code}{c}
    int x = 0;
    while ( x++ < 10 ); // This is (of course) not a useful snippet, but shows the concept
\end{code}

\content{Function pointers}
A function can be passed as an argument to another function using the typical address syntax with the \verb|&| symbol is annotated as argument using
\verb|type (* name)(type arg1, ...)|
and is called using \verb|(*func)(arg1, ...)|.

\newpage
\subsection{The C preprocessor}
To have \texttt{gcc} stop compiliation after running through \texttt{cpp}, the \texttt{C preprocessor}, use \texttt{gcc -E <file name>}.

Imports in \lC\ are handled by the preprocessor, that for each \verb|#include <file1.h>|, the preprocessor simply copies the contents of the file recursively into one file.

Depending on if we use \verb|#include <file1.h>| or \verb|#include "file1.h"| the preprocessor will search for the file either in the system headers or in the project directory.
Be wary of including files twice, as the preprocessor will recursively include all files (i.e. it will include files from the files we included)

The \lC\ preprocessor gives us what are called \texttt{preprocessor macros}, which have the format \verb|#define NAME SUBSTITUTION|.
\rmvspace

\inputcodewithfilename{c}{}{code-examples/00_c/01_preprocessor/00_macros.c}

To avoid issues with semicolons at the end of preprocessor macros that wrap statements that cannot end in semicolons, we can use a concept called semicolon swallowing.
For that, we wrap the statements in a \texttt{do \dots\ while(0)} loop, which is removed by the compiler on compile, also taking with it the semicolon.

There are also a number of predefined macros:
\begin{itemize}[noitemsep]
    \item \verb|__FILE__|: Filename of processed file
    \item \verb|__LINE__|: Line number of this usage of macro
    \item \verb|__DATE__|: Date of processing
    \item \verb|__TIME__|: Time of processing
    \item \verb|__STDC__|: Set if ANSI Standard \lC\ compiler is used
    \item \verb|__STDC_VERSION__|: The version of Standard \lC\ being compiled
    \item \dots many more
\end{itemize}
In headers, we typically use \verb|#ifndef __FILENAME_H_| followed by a \verb|#define __FILENAME_H_| or the like to check if the header was already included before

\newsection
\subsection{Trigonometrische Interpolation}
\subsubsection{Von Approximation zur Interpolation}
Wir erinnern uns daran, dass wir die Fourier-Approximation durch den Abbruch der unendlichen Fourier-Reihe erhalten, oder in anderen Worten, wir verkleinern die Limiten der Summe.

\fancyremark{DFT mit $N = 2n$ Koeffizienten an Punkten $\frac{l}{N}$ für $l = 0, 1, \ldots, N - 1$}

Der Shift ist hier gegeben durch (für $k \geq 0$ ist $\gamma_k = \hat{f}_N(k)$ und für $k < 0$ ist $\gamma_k = \hat{f}_N(N + k)$)
\begin{align*}
    f_{N - 1}(x)                 & = \sum_{k = -n}^{n - 1} \gamma_k e^{2 \pi ikx} = \sum_{k = 0}^{n - 1} \gamma_k e^{2\pi ikx} + \sum_{k = -n}^{-1} \gamma_k e^{2\pi ikx} \\
    \Leftrightarrow f_{N - 1}(x) & = \frac{1}{N} \left( \sum_{j = 0}^{N - 1} \left( f\left( \frac{j}{n} \right)
        \sum_{k = -n}^{n - 1} e^{2\pi ik \left( x - \frac{j}{N} \right)} \right) \right)
\end{align*}

\vspace{-1pc}

Wenn wir die Funktion nun an der Stelle $\frac{l}{N}$ auswerten so erhalten wir:
\rmvspace
\begin{align*}
    f_{N - 1}\left( \frac{l}{N} \right) = \ldots = f\left( \frac{l}{N} \right)
\end{align*}

\vspace{-1.8pc}
was aufgrund der Orthogonalität der diskreten Fourier-Vektoren funktioniert, welche besagt, dass $\displaystyle \sum_{k = -n}^{n - 1} \omega_N^{k(j - l)} = 0$, für alle $j \neq l$.
Für $j = l$ ergibt die Summe $N$.

Dies heisst also, dass die Fourier-Approximation die Interpolationsbedingungen an den Punkten $\frac{l}{N}$ erfüllt,
also können wir die Lösung der Interpolationsaufgabe $p_{N - 1} \left( \frac{l}{N} \right) = f\left( \frac{l}{N} \right)$ f $l = 0, 1, \ldots, N - 1$ im Raum
\rmvspace
\begin{align*}
    \mathcal{T}_N = \text{span}\{ e^{2\pi ijt} \divides j = - \floor{\frac{N - 1}{2}}, \ldots, \floor{\frac{N}{2}} \}
\end{align*}

\rmvspace\rmvspace
folgendermassen finden können: 
\begin{enumerate}[label=(\arabic*)]
    \item Mittels Gleichungssystem $\sum_{j} \gamma_j e^{2\pi ijt_l} = f(t_l)$ für $l = 0, \ldots, N - 1$. Operationen: $\tco{N^3}$
    \item Mittels FFT in $\tco{N \log(N)}$ Operationen, aber nur falls die Punkte äquidistant sind, also $t_l = \frac{l}{N}$.
        Dann ist die Matrix des obigen Gleichungssystems $F^{-1}_N$
\end{enumerate}

\vspace{0.2cm}

Unten findet sich Python code der mit den unterschiedlichen Methoden die Koeffizienten des Trigonometrischen Polynoms bestimmt.
\rmvspace
\begin{code}{python}
    def get_coeff_trig_poly(t: np.ndarray, y: np.ndarray):
    N = y.shape[0]
    if N % 2 == 1:
    n = (N - 1.0) / 2.0
    M = np.exp(2 * np.pi * 1j * np.outer(t, np.arange(-n, n + 1)))
    else:
    n = N / 2.0
    M = np.exp(2 * np.pi * 1j * np.outer(t, np.arange(-n, n)))
    c = np.linalg.solve(M, y)
    return c

    N = 2**12
    t = np.linspace(0, 1, N, endpoint=False)
    y = np.random.rand(N)
    direct = get_coeff_trig_poly(t, y)
    using_fft = np.fft.fftshift(np.fft.fft(y) / N)
    using_ifft = np.conj(np.fft.fftshift(np.fft.ifft(y)))
\end{code}

\subsubsection{Dynamic Memory Allocation}
Memory allocated with \texttt{malloc} is typically $8$- or $16$-byte aligned.

\content{Explicit vs. Implicit} In explicit memory management, the application does both the allocation \textit{and} deallocation memory,
whereas in implicit memory management, the application allocates the memory, but usually a \textit{Garbage Collector} (GC) frees it.

For some languages, like Rust, one would assume that it does implicit allocation, but Rust is a language using explicit management,
it's just that the \textit{compiler} and not the programmer decides when to allocate and when to deallocate.

\warn{Assumptions in this course} We assume that memory is \bi{word} addressed (= 8 Bytes).

\content{Goals} The allocation should have the highest possible throughput and at the same time the best (i.e. lowest) possible memory utilization.
This however is usually conflicting, so we have to balance the two.

\numberingOff
\inlinedef \bi{Aggregate payload} $P_k$: All \texttt{malloc}'d stuff minus all \texttt{free}'d stuff

\inlinedef \bi{Current heap size} $H_k$: Monotonically non-decreasing. Grows when \texttt{sbrk} system call is issued.

\inlinedef \bi{Peak memory utilization} $U_k = (\max_{i < k} P_i) / H_k$


A bit problem for the \texttt{free} function is to know how much memory to free without knowing the size of the to be freed block.
This is just one of many other implementation issues:
\begin{itemize}
    \item How do we keep track of the free blocks? I.e. where and how large are they?
    \item What do we do with the extra space of a block when allocating a smaller block?
    \item How do we pick a block?
    \item How do we reinsert a freed block into the heap?
\end{itemize}
This all leads to an issue known as \bi{fragmentation}

\inlinedef \bi{Internal Fragmentation}: If for a given block the payload (i.e. the requested size) is smaller than the block size.
This depends on the pattern of previous requests and is thus easy to measure

\inlinedef \bi{External Fragmentation}: There is enough aggregate heap memory, but there isn't a single large enough free block available
This depends on the pattern of future requests and is thus hard to measure

\subsubsection{Garbage Collection}

The memory manager must somehow be able to tell what memory can be freed. In general, we cannot know if memory is going to be used or not,
except if there exists no pointer to it anymore.
Garbage collectors use graphs to track pointer availability.
In other words, a block is reachable if there exists a path from a root node to it.

An easy GC algorithm is called \bi{Mark and Sweep}. It has an extra bit in the header called the \textit{mark bit} and can be built on top of malloc/free.
The concept is to use malloc until we ``run out of space'' and to then run these steps:
\begin{itemize}
    \item \bi{Mark}: Starts at each root node and sets a mark bit on each reachable block.
    \item \bi{Sweep}: Scan all blocks and free all blocks that are unmarked.
\end{itemize}

\subsubsection{Common pitfalls}
\begin{itemize}[noitemsep]
    \item Dereferencing bad pointers (e.g. passing an \texttt{int} to a function expecting a pointer)
    \item Reading uninitialized memory (memory allocated with \texttt{malloc} should be considered garbage)
    \item Overwriting memory (if you mess up pointer arithmetic or don't do boundary checks)
    \item Referencing nonexistent variables (variables go out of scope on function returns, except \texttt{static})
    \item Freeing blocks multiple times (can corrupt the heap)
    \item Referencing freed blocks (always \texttt{NULL} pointers after using \texttt{free()})
    \item Failing to free blocks (memory leak incoming and make sure to free the ENTIRE data structure!)
\end{itemize}
Some of these bugs (especially bad references) can usually be found using a debugger.

Substitute \texttt{malloc} with a \texttt{malloc} that has extra checking code (like \texttt{UToronto CSRI malloc} to detect memory leaks)

Another option is using \texttt{valgrind} (a memory debugger). Or, simply don't bother with \lC\ and use \texttt{Rust}.

\newpage
\subsection{Variadic functions}

Variadic functions take a variable number of arguments and use the \texttt{...} syntax in \lC.
A notable example of such a function is \texttt{printf}
\inputcodewithfilename{c}{}{code-examples/00_c/03_others/00_variadic.c}

\newpage

\subsection{Code Vulnerabilities}

A brief interjection on some code vulnerabilities.

\content{System-level protections}

\begin{itemize}
    \item \textbf{Compiler-inserted checks} on functions
    \item \textbf{Randomized stack offsets}: Allocate \textit{random} amount on stack before running the program
    \item \textbf{Nonexecutable segments}: Memory needs a special \textit{execute} permission
\end{itemize}

\subsubsection{Buffer overflow}

Buffer overflows are a method for code injection on vulnerable code with specific buffer size-checking deficiencies.\\
There are 2 ways to do this:
\begin{enumerate}
    \item Change a function call or return address
    \item Push malicious assembly onto the stack
\end{enumerate}

For example, consider this code:

\inputcodewithfilename{c}{code-examples/00_c/05_vulnerabilities/}{01_buffer_overflow_echo.c}

This is a problem, since \texttt{echo} may be compiled to something similar to this:

\inputcodewithfilename{gas}{code-examples/00_c/05_vulnerabilities/}{02_buffer_overflow_echo_asm.s}

Since \texttt{buf} is on the stack, and there is no size-enforcement when writing to \texttt{buf}, malicious input can write \textit{before} \texttt{\%rsp}, since the Stack grows downwards. 
This means stack memory that the program is intending to use again can be modified.

However, inserting exectuable assembly like this usually does not work, since the stack may not be executable due to missing system permission.

The vulnerability above could be fixed by using \texttt{fgets(buf, 4, stdin)} instead, which checks the size.

\content{Heap overflow} On the heap, buffer overflows work differently, as the heap contains no return addresses. However, the heap stores function pointers, which can be modified. Further, sophisticated attacks can use buffer overflow to potentialy modify pointers in dynamically allocated memory.

\subsubsection{Return-oriented Programming}

Return-oriented Programming is a more sophisticated exploit, which does not rely on injecting any new code.

The key idea is: Overwrite return addresses and jump to \textit{specific} machine instruction sequences \textit{already present} in process memory.

% This is only covered in the attack-lab exercise, not the slides.
\newsection
\subsection{Trigonometrische Interpolation}
\subsubsection{Von Approximation zur Interpolation}
Wir erinnern uns daran, dass wir die Fourier-Approximation durch den Abbruch der unendlichen Fourier-Reihe erhalten, oder in anderen Worten, wir verkleinern die Limiten der Summe.

\fancyremark{DFT mit $N = 2n$ Koeffizienten an Punkten $\frac{l}{N}$ für $l = 0, 1, \ldots, N - 1$}

Der Shift ist hier gegeben durch (für $k \geq 0$ ist $\gamma_k = \hat{f}_N(k)$ und für $k < 0$ ist $\gamma_k = \hat{f}_N(N + k)$)
\begin{align*}
    f_{N - 1}(x)                 & = \sum_{k = -n}^{n - 1} \gamma_k e^{2 \pi ikx} = \sum_{k = 0}^{n - 1} \gamma_k e^{2\pi ikx} + \sum_{k = -n}^{-1} \gamma_k e^{2\pi ikx} \\
    \Leftrightarrow f_{N - 1}(x) & = \frac{1}{N} \left( \sum_{j = 0}^{N - 1} \left( f\left( \frac{j}{n} \right)
        \sum_{k = -n}^{n - 1} e^{2\pi ik \left( x - \frac{j}{N} \right)} \right) \right)
\end{align*}

\vspace{-1pc}

Wenn wir die Funktion nun an der Stelle $\frac{l}{N}$ auswerten so erhalten wir:
\rmvspace
\begin{align*}
    f_{N - 1}\left( \frac{l}{N} \right) = \ldots = f\left( \frac{l}{N} \right)
\end{align*}

\vspace{-1.8pc}
was aufgrund der Orthogonalität der diskreten Fourier-Vektoren funktioniert, welche besagt, dass $\displaystyle \sum_{k = -n}^{n - 1} \omega_N^{k(j - l)} = 0$, für alle $j \neq l$.
Für $j = l$ ergibt die Summe $N$.

Dies heisst also, dass die Fourier-Approximation die Interpolationsbedingungen an den Punkten $\frac{l}{N}$ erfüllt,
also können wir die Lösung der Interpolationsaufgabe $p_{N - 1} \left( \frac{l}{N} \right) = f\left( \frac{l}{N} \right)$ f $l = 0, 1, \ldots, N - 1$ im Raum
\rmvspace
\begin{align*}
    \mathcal{T}_N = \text{span}\{ e^{2\pi ijt} \divides j = - \floor{\frac{N - 1}{2}}, \ldots, \floor{\frac{N}{2}} \}
\end{align*}

\rmvspace\rmvspace
folgendermassen finden können: 
\begin{enumerate}[label=(\arabic*)]
    \item Mittels Gleichungssystem $\sum_{j} \gamma_j e^{2\pi ijt_l} = f(t_l)$ für $l = 0, \ldots, N - 1$. Operationen: $\tco{N^3}$
    \item Mittels FFT in $\tco{N \log(N)}$ Operationen, aber nur falls die Punkte äquidistant sind, also $t_l = \frac{l}{N}$.
        Dann ist die Matrix des obigen Gleichungssystems $F^{-1}_N$
\end{enumerate}

\vspace{0.2cm}

Unten findet sich Python code der mit den unterschiedlichen Methoden die Koeffizienten des Trigonometrischen Polynoms bestimmt.
\rmvspace
\begin{code}{python}
    def get_coeff_trig_poly(t: np.ndarray, y: np.ndarray):
    N = y.shape[0]
    if N % 2 == 1:
    n = (N - 1.0) / 2.0
    M = np.exp(2 * np.pi * 1j * np.outer(t, np.arange(-n, n + 1)))
    else:
    n = N / 2.0
    M = np.exp(2 * np.pi * 1j * np.outer(t, np.arange(-n, n)))
    c = np.linalg.solve(M, y)
    return c

    N = 2**12
    t = np.linspace(0, 1, N, endpoint=False)
    y = np.random.rand(N)
    direct = get_coeff_trig_poly(t, y)
    using_fft = np.fft.fftshift(np.fft.fft(y) / N)
    using_ifft = np.conj(np.fft.fftshift(np.fft.ifft(y)))
\end{code}

\subsubsection{Fractional Binary Numbers}

We can represent any real number (with a finite decimal representation) as:
$$
    d=\sum_{i=-n}^{m}10^i\cdot d_i \qquad\qquad \underbrace{d_m d_{m-1} \cdots d_1 d_0\ .\ d_{-1} d_{-2} \cdots d_{-(n-1)} d_{-n}}_{d_i \text{ is the } i \text{-th digit of } d \text{ (neg. indices indicate decimals)}}
$$
We can use the same idea for Base $2$ as well:
$$
    b=\sum_{i=-n}^{m} 2^i \cdot b_i \qquad\qquad b_m b_{m-1} \cdots b_1 b_0\ .\ b_{-1} b_{-2} \cdots b_{-(n-1)} b_{-n}
$$
To get an intuition for this representation, looking at some examples is helpful:
\begin{multicols}{2}

A few observations:
\begin{enumerate}
    \item Shifting the dot right: Division by $2$
    \item Shifting the dot left: Multiply by $2$
    \item Numbers of the form $0.111\ldots$ are just below $1.0$
    \item Some numbers representable in finite Base $10$ are infinite in Base $2$, e.g. $\frac{1}{5} = 0.20_{10}$
\end{enumerate}

\newcolumn

\renewcommand{\arraystretch}{1.2}
\begin{center}
    \begin{tabular}{lcl}
        \textbf{Binary} & \textbf{Fraction} & \textbf{Decimal} \\
        \hline
        $0.0$           & $\frac{0}{2}$     & $0.0$         \\
        $0.01$          & $\frac{1}{4}$     & $0.25$        \\
        $0.010$         & $\frac{2}{8}$     & $0.25$        \\
        $0.0011$        & $\frac{3}{16}$    & $0.1875$      \\
        $0.00110$       & $\frac{6}{32}$    & $0.1875$      \\
        $0.001101$      & $\frac{13}{64}$   & $0.203125$    \\
        $0.0011010$     & $\frac{26}{128}$  & $0.203125$    \\
        $0.00110101$    & $\frac{51}{256}$  & $0.19921875$  \\
    \end{tabular}
\end{center}
\renewcommand{\arraystretch}{1.0}

\end{multicols}

A major issue with this representation is that very large (respectively very small) numbers require a large representation.\\
E.g $a_{10} = 5 \cdot 2^{100}$ has the representation $a_2 = 101\underbrace{000000000000000\ldots}_{100 \text{ Zeros}}\ $. Floating Point is designed to address this.

\subsubsection{Floating Point Representation}
Floating point numbers instead use the representation:
$$
    a = \underbrace{(-1)^s}_\text{Sign} \cdot \underbrace{M}_\text{Mantissa} \cdot \underbrace{2^E}_\text{Exponent}
$$

Single precision and Double precision floating point numbers store the $3$ parameters in separate bit fields $s, e, m$:

\begin{center}
    Single Precision:
    \begin{tabular}{|c|c|c|}
        \hline
        $31$: Sign & $30-23$: Exponent & $22-0$: Mantissa \\
        \hline
    \end{tabular} \\
    Bias: $127$, Exponent range: $[-126, 127]$
\end{center}
\begin{center}
    Double Precision:
    \begin{tabular}{|c|c|c|}
        \hline
        $63$: Sign & $62-52$: Exponent & $51-0$: Mantissa \\
        \hline
    \end{tabular}\\
    Bias: $1023$, Exponent range: $[-1022, 1023]$
\end{center}

Most of the extra precision in $64$b floating point numbers is associated to the mantissa. Note how double precision is necessary to represent all $32$b signed Integers, and not all $64$b signed Integers can be represented in either format.

\newpage

The way these bitfields are interpretd \textit{differs} based on the exponent field $e$:

\begin{enumerate}
    \item \textbf{Normalized Values}: Exponent bit field $e$ is neither all $1$s nor all $0$s.\\
          In this case, $E$ is read in \textit{biased} form: $E = e - b$. The bias is $b=2^{k-1}-1$, where $k$ is the amount of bits reserved for $e$. This produces the exponent ranges $E \in [-(b-1), b]$.\\
          The mantissa field $m$ is interpreted as $M = 0.m_{n-1}\ldots m_1 m_0 + 1$, where $n$ is the amount of bits reserved for $m$
    \item \textbf{Denormalized Values}: Exponent bit field $e$ is all $0$s.\\
          In this case, $E$ is read in \textit{biased} form $E = 1 - b$. (Instead of $E = e - b$)\\
          The mantissa field $m$ is interpreted as $M = 0.m_{n-1}\ldots m_1 m_0$ (without adding $1$)
    \item \textbf{Special Values}: Exponent bit field $e$ is all $1$s.\\
          $m = 0$ represents infinitiy, which is signed using $s$.\\
          $m \neq 0$ is \verb|NaN|, regardless of what is in $m$ or $s$.
\end{enumerate}

\content{Why is the Bias chosen this way?} It allows smooth transitions between normalized and denormalized values.

\subsubsection{Properties}

The advantage of having denormalized values is that 0 can be represented as the bit-field with all $0$s. Further, this enforces equidistant points for values close to $0$, whereas normalized values increase in distance as they move further from $0$.

\content{Example} $8$b Floating Point table to visualize the different cases.
$$
    8\text{b precision Floating Point:}\quad \underbrace{0}_s \underbrace{0000}_e \underbrace{000}_m
$$

\renewcommand{\arraystretch}{1.2}
\begin{center}
    \begin{tabular}{llllll}
        \hline
        Case & $s$ & $e$ & $m$ & $E$ & Value \\ 
        \hline
        \multirow{6}{*}{Denormalized} 
        & 0 & 0000 & 000 & $-6$ & $0$ \\
        & 0 & 0000 & 001 & $-6$ & $\frac{1}{8}\cdot\frac{1}{64}=\frac{1}{512}$ \\
        & 0 & 0000 & 010 & $-6$ & $\frac{2}{8}\cdot\frac{1}{64}=\frac{2}{512}$ \\
        &   &      & $\vdots$ &      & $\vdots$ \\
        & 0 & 0000 & 110 & $-6$ & $\frac{6}{8}\cdot\frac{1}{64}=\frac{6}{512}$ \\
        & 0 & 0000 & 111 & $-6$ & $\frac{7}{8}\cdot\frac{1}{64}=\frac{7}{512}$ \\
        \hline
        \multirow{9}{*}{Normalized}
        & 0 & 0001 & 000 & $-6$ & $\frac{8}{8}\cdot\frac{1}{64}=\frac{8}{512}$ \\
        & 0 & 0001 & 001 & $-6$ & $\frac{9}{8}\cdot\frac{1}{64}=\frac{9}{512}$ \\
        &   &      & $\vdots$ &      & $\vdots$ \\
        & 0 & 0110 & 110 & $-1$ & $\frac{14}{8}\cdot\frac{1}{2}=\frac{14}{16}$ \\
        & 0 & 0110 & 111 & $-1$ & $\frac{15}{8}\cdot\frac{1}{2}=\frac{15}{16}$ \\
        & 0 & 0111 & 000 & $0$ & $\frac{8}{8}\cdot 1 = 1$ \\
        & 0 & 0111 & 001 & $0$ & $\frac{9}{8}\cdot 1 = \frac{9}{8}$   \\
        & 0 & 0111 & 010 & $0$ & $\frac{10}{8}\cdot 1 = \frac{10}{8}$ \\
        &   &      & $\vdots$ &      & $\vdots$ \\
        & 0 & 1110 & 110 & $7$ & $\frac{14}{8}\cdot 128 = 224$ \\
        & 0 & 1110 & 111 & $7$ & $\frac{15}{8}\cdot 128 = 240$ \\
        \hline
        Special
        & 0 & 1111 & 000 & n/a & $\infty$ \\
        \hline
    \end{tabular}
\end{center}
\renewcommand{\arraystretch}{1.0}

\newpage
\subsubsection{Rounding}

The basic idea of Floating Point operations is:
\begin{enumerate}
    \item Compute exact result
    \item Round, so it fits the desired precision
\end{enumerate}

\textit{IEEE Standard 754} specifies $4$ rounding modes: \textit{Towards Zero, Round Down, Round Up, Nearest Even}.

The default used is \textit{Nearest Even}\footnote{Changing the rounding mode is usually hard to do without using Assembly.}, which rounds up/down depending on which number is closer, like regular rounding, but picks the nearest even number if it's exactly in the middle.

Rounding can be defined using 3 different bits from the \textit{exact} number: $G, R, S$
$$
    a = 1.BB\ldots BB\underbrace{G}_\text{Guard}\underbrace{R}_\text{Round}\underbrace{XX\ldots XX}_\text{Sticky}
$$

\begin{enumerate}
    \item \textbf{Guard Bit} $G$ is the least significant bit of the (rounded) result
    \item \textbf{Round Bit} $R$ is the $1$st bit cut off after rounding
    \item \textbf{Sticky Bit} $S$ is the logical OR of all remaining cut off bits.
\end{enumerate}

Based on these bits the rounding can be decided:

$$
    R \land S \implies \text{ Round up} \qquad\qquad
    G \land R \land \lnot S \implies \text{ Round to even}
$$

\content{Example} Rounding $8$b precise results to $8$b precision floating point ($4$b mantissa):

\renewcommand{\arraystretch}{1.2}
\begin{center}
    \begin{tabular}{|c|c|c|c|c|}
        \hline
        \textbf{Value} & \textbf{Fraction} & \textbf{GRS} & \textbf{Incr?} & \textbf{Rounded} \\
        \hline
        $128$          & $1.000|0000$      & $000$        & N              & $1.000$          \\
        $13$           & $1.101|0000$      & $100$        & N              & $1.101$          \\
        $17$           & $1.000|1000$      & $010$        & N              & $1.000$          \\
        $19$           & $1.001|1000$      & $110$        & Y              & $1.010$          \\
        $138$          & $1.000|1010$      & $011$        & Y              & $1.001$          \\
        $63$           & $1.111|1100$      & $111$        & Y              & $10.000$         \\
        \hline
    \end{tabular}
\end{center}
\renewcommand{\arraystretch}{1.0}


\textbf{Post-Normalization}: Rounding may cause overflow. In this case: Shift right once and increment exponent.

\newpage 
\subsubsection{Operations}

\content{Multiplication} is straightforward, all $3$ parameters can be operated on separately:
$$
    (-1)^{s_1}M_1 \cdot 2^{E_1} \cdot (-1)^{s_2} M_2 \cdot 2^{E_2} \quad = \quad (-1)^{s_1 \oplus s_2} (M_1 \cdot M_2) 2^{E_1 + E_2} 
$$
\textbf{Post-Normalization}: 
\begin{enumerate}
    \item If $M \geq 2$, shift $M$ right and increment $E$
    \item If $E$ out of range, overflow (set to $\infty$)
    \item Round $M$ to fit desired precision.
\end{enumerate}

\content{Addition} is more complicated: (Assumption: $E_1 \geq E_2$)
$$
    (-1)^{s_1}M_1 \cdot 2^{E_1} + (-1)^{s_2} M_2 \cdot 2^{E_2} \quad = \quad (-1)^{s'} M' \cdot 2^{E_1}
$$
$s', M'$ are the result of a signed align \& add.\\
This means $(-1)^{s_1}M_1$ is shifted left by $E_1-E_2$, and then $(-1)^{s_2}M_2$ is added.

\textbf{Post-Normalization}:
\begin{enumerate}
    \item if $M \geq 2$, shift $M$ right, increment $E$
    \item if $M \leq 1$, shift $M$ left $k$, decrement $E$ by $k$
    \item Overflow $E$ if out of range (set to $\infty$)
    \item Round $M$ to desired precision
\end{enumerate}

\subsubsection{Mathematical Properties}

Floating point is \textit{almost} an Abelian Group.
\begin{itemize}
    \item \textbf{Closed} under Addition, Multiplication (But may generate \verb|NaN|, $\pm \infty$)
    \item \textbf{Commutative}
    \item \textbf{Not Associative} (Overflow \& Rounding)
    \item $0, 1$ \textbf{are Identity}
    \item \textbf{Additive Inverse} (Except $\pm \infty$ and \verb|NaN|)
    \item \textbf{Monotonicity} (Except for $\pm \infty$ and \verb|NaN|)
    \item \textbf{Not Distributive} (Overflow \& Rounding)
\end{itemize}

\subsubsection{Floating Point in C}

C99 guarantees \verb|float| and \verb|double|, and \verb|long double| is usually interpreted as quadruple precision.

Casting/Conversion between Integer types and \verb|float|, \verb|double| \textit{changes} the bit representation in most cases (e.g. $0$ stays the same)



% ── GCC ─────────────────────────────────────────────────────────────
\newpage
\section{The gcc toolchain}
\newsection
\subsection{Trigonometrische Interpolation}
\subsubsection{Von Approximation zur Interpolation}
Wir erinnern uns daran, dass wir die Fourier-Approximation durch den Abbruch der unendlichen Fourier-Reihe erhalten, oder in anderen Worten, wir verkleinern die Limiten der Summe.

\fancyremark{DFT mit $N = 2n$ Koeffizienten an Punkten $\frac{l}{N}$ für $l = 0, 1, \ldots, N - 1$}

Der Shift ist hier gegeben durch (für $k \geq 0$ ist $\gamma_k = \hat{f}_N(k)$ und für $k < 0$ ist $\gamma_k = \hat{f}_N(N + k)$)
\begin{align*}
    f_{N - 1}(x)                 & = \sum_{k = -n}^{n - 1} \gamma_k e^{2 \pi ikx} = \sum_{k = 0}^{n - 1} \gamma_k e^{2\pi ikx} + \sum_{k = -n}^{-1} \gamma_k e^{2\pi ikx} \\
    \Leftrightarrow f_{N - 1}(x) & = \frac{1}{N} \left( \sum_{j = 0}^{N - 1} \left( f\left( \frac{j}{n} \right)
        \sum_{k = -n}^{n - 1} e^{2\pi ik \left( x - \frac{j}{N} \right)} \right) \right)
\end{align*}

\vspace{-1pc}

Wenn wir die Funktion nun an der Stelle $\frac{l}{N}$ auswerten so erhalten wir:
\rmvspace
\begin{align*}
    f_{N - 1}\left( \frac{l}{N} \right) = \ldots = f\left( \frac{l}{N} \right)
\end{align*}

\vspace{-1.8pc}
was aufgrund der Orthogonalität der diskreten Fourier-Vektoren funktioniert, welche besagt, dass $\displaystyle \sum_{k = -n}^{n - 1} \omega_N^{k(j - l)} = 0$, für alle $j \neq l$.
Für $j = l$ ergibt die Summe $N$.

Dies heisst also, dass die Fourier-Approximation die Interpolationsbedingungen an den Punkten $\frac{l}{N}$ erfüllt,
also können wir die Lösung der Interpolationsaufgabe $p_{N - 1} \left( \frac{l}{N} \right) = f\left( \frac{l}{N} \right)$ f $l = 0, 1, \ldots, N - 1$ im Raum
\rmvspace
\begin{align*}
    \mathcal{T}_N = \text{span}\{ e^{2\pi ijt} \divides j = - \floor{\frac{N - 1}{2}}, \ldots, \floor{\frac{N}{2}} \}
\end{align*}

\rmvspace\rmvspace
folgendermassen finden können: 
\begin{enumerate}[label=(\arabic*)]
    \item Mittels Gleichungssystem $\sum_{j} \gamma_j e^{2\pi ijt_l} = f(t_l)$ für $l = 0, \ldots, N - 1$. Operationen: $\tco{N^3}$
    \item Mittels FFT in $\tco{N \log(N)}$ Operationen, aber nur falls die Punkte äquidistant sind, also $t_l = \frac{l}{N}$.
        Dann ist die Matrix des obigen Gleichungssystems $F^{-1}_N$
\end{enumerate}

\vspace{0.2cm}

Unten findet sich Python code der mit den unterschiedlichen Methoden die Koeffizienten des Trigonometrischen Polynoms bestimmt.
\rmvspace
\begin{code}{python}
    def get_coeff_trig_poly(t: np.ndarray, y: np.ndarray):
    N = y.shape[0]
    if N % 2 == 1:
    n = (N - 1.0) / 2.0
    M = np.exp(2 * np.pi * 1j * np.outer(t, np.arange(-n, n + 1)))
    else:
    n = N / 2.0
    M = np.exp(2 * np.pi * 1j * np.outer(t, np.arange(-n, n)))
    c = np.linalg.solve(M, y)
    return c

    N = 2**12
    t = np.linspace(0, 1, N, endpoint=False)
    y = np.random.rand(N)
    direct = get_coeff_trig_poly(t, y)
    using_fft = np.fft.fftshift(np.fft.fft(y) / N)
    using_ifft = np.conj(np.fft.fftshift(np.fft.ifft(y)))
\end{code}

\newpage
\subsection{Compiler optimizations}
While the compiler can do quite a bit to speed up code, it can't rework the core logic, as it has to guarantee that the executable does do what was specified in the code.

So, it is really important to not only consider asymptotic runtime (as \texttt{100n} and \texttt{5n} are both $\tco{n}$, but oviously the latter is 20 times faster).
We thus need to optimize the algorithms, data representations, loops, etc and for that, we need to properly understand how programs are compiled, executed and how the hadware works.

When using \texttt{gcc}, it is usually a good idea to compile a final build with the \texttt{-O2} or \texttt{-O3} flags.

The \texttt{-march} flag was already mentioned in table \ref{tab:gcc-flags} and can be used if you want to go above and beyond, as it will optimize for the specific hardware.
The values that can be passed to \texttt{-march} are listed \hlhref{https://gcc.gnu.org/onlinedocs/gcc/x86-Options.html}{here} and even include a specific CPU microarchitecture.
For example, to compile for Intel Alderlake (12000 series), you can specify \texttt{-march=alderlake}

To understand what you need to optimize, you need to understand what the compiler is good at:
\begin{itemize}
    \item Register allocation
    \item Scheduling (i.e. code selection and ordering)
    \item Dead code elimination
    \item Eliminating minor (!) inefficiencies
\end{itemize}
and what it is not good at:
\begin{itemize}
    \item Improving Asymptotic efficiency (compiler can't turn BubbleSort into e.g. QuickSort)
    \item Improving the constant factor (if your implementation is slow, it likely won't magically become faster, though some bad practices can be eliminated)
    \item Overcoming other optimization blockers such as memory aliasing and procedure side-effects
\end{itemize}

\content{Code motion} is a compiler technique, where it moves certain computations out of loops that always produce the same result.
However: Always remember that the compiler will be \bi{conservative}, i.e. it will always err on the side of caution.

\content{Strength reduction} is a compiler technique, where e.g. sequences of products are turned into cheaper additions in each iteration.
An example is that if you have an operation such as \texttt{n * i} in a loop,
the compiler might replace that with a variable \texttt{ni} that is incremented by \texttt{n} in each iteration.
Similarly, it might replace \texttt{16 * x}, or even worse still, \texttt{x / 16} with \texttt{x << 4} or \texttt{x >> 4}, respectively

\content{Common sub-expressions} can be extracted into pre-computations and then only use cheaper operations on the individual steps.
A good example is if you are using similar multiplications that then only require one addition or subtraction to get to a result close by.

\subsubsection{Optimization blockers}
A sure-fire way to make your code slow is by using a large number of procedure calls.
They are among the slowest operations in \lC.
And, the compiler cannot safely extract the function in a for loop like this:
\begin{code}{c}
    int i;
    for (i = 0; i < strlen(s); i++) {
        if (s[i] >= 'A' && s[i] <= 'Z') {
            s[i] -= ('A' - 'a');
        }
    }
\end{code}
The compiler can't safely remove \texttt{strlen(s)} from the loop, as it may have side-effects,
i.e. may modify other program content other than simply returning a value.
Thus, only ever call functions in the loop condition when you need the side-effects and otherwise, pre-compute it and simply use a variable to check against.
\begin{scriptsize}
    You can declare a function \textit{side-effect free} using \verb|__attribute__((pure))| in the function declaration.
    The compiler may then extract \texttt{strlen(s)} from the loop.
\end{scriptsize}

Another common blocker is memory aliasing. This happens when two pointers point to the same address and of course,
since we can do pointer arithmetic, it is very easy to do that in \lC.
The easiest way to prevent this from happening is to use local variables where possible,
such that they do not need to be passed in using a pointer.

Normally the compiler assumes there can be another pointer that accesses the memory pointed to by this pointer.
If you use the \texttt{restrict} keyword on the variable (i.e. in a function declaration, we have \texttt{void test(double restrict *a)}),
the compiler will assume that for the lifetime of this pointer, there are no other pointers that will be used to access the memory to which it points.

Another technique to improve throughput for something like matrix multiplications is to do it in blocks due to the way caching works.
Since the compiler doesn't \textit{understand} your code, it can't do this for you (as it assumes associativity of the operation)

\subsection{Linking}

Linking is the final step in the compilation pipeline: separately compiled object files are combined into an executable.

The advantages of using Linkers are clear:
\begin{enumerate}
    \item \textbf{Separate Compilation}: Changing one source file requires only recompiling that file.
    \item \textbf{Space Optimization}: Executable code only contains functions (e.g. from libraries) that are actually used.
\end{enumerate}

\subsubsection{Symbol Resolution}

The first step during Linking is Symbol Resolution.

In the context of Linking, all variables and functions are considered \textit{Symbols}. Compilers store all symbol definitions in a \textit{Symbol Table}.
The linker associates symbol references with \textit{exactly one} definition.

\inlinedef \textbf{Symbol types}

\begin{itemize}
    \item \textbf{Global Symbols} can be referenced by other modules (e.g. \texttt{non-static} in \texttt{C})
    \item \textbf{External Symbols} are referenced globals defined elsewhere
    \item \textbf{Local Symbols} are defined and referenced exclusively in one module (e.g. \texttt{static} in \texttt{C})
\end{itemize}

Note: Local linker symbols and local program variables are \textit{not} the same.

\inlinedef \textbf{Symbol strength}

Duplicate symbols either lead to linking errors (\texttt{-fno-common}, the default) or compile (\texttt{-fcommon})

\begin{itemize}
    \item \textbf{Strong Symbols} are procedure names and initialized globals
    \item \textbf{Weak Symbols} are uninitialized globals (on \texttt{-fcommon})
\end{itemize}

in \texttt{C}, function symbols can explicitly be declared weak using:
\begin{minted}{C}
    #pragma weak func
    __attribute__((weak))__ func()
\end{minted}

\content{Duplicate Handling} The linker uses these definitions to handle duplicates:

\begin{enumerate}
    \item Given multiple strong symbols are illegal
    \item Given a strong symbol and multiple weak symbols, pick the strong symbol
    \item Given multiple weak symbols, choose an \textit{arbitrary} one
\end{enumerate}

\subsubsection{Relocation}

The second step during Linking is Relocation.

Code and data sections of separate sources are combined, and symbols are relocated from relative locations (in \texttt{.o} files) to absolute locations (in executable files)

\textbf{Command line order matters} for this, since the Linker will scan \texttt{.o} and \texttt{.a} files in the order they are appear in the CLI arguments.
In general, libraries should therefore be linked \textit{last}.

\newpage
\subsubsection{Packaging Libraries}

Using just the Linker, there are only 2 inconvenient ways to package libraries:

\begin{enumerate}
    \item All functions into 1 file $\mapsto$ linking unnecessarily big objects.
    \item One function per file $\mapsto$ Requires linking a lot of files, annoying for programmer.
\end{enumerate}

\textbf{Static Libraries} solve this: The linker looks for functions inside the static library, and only links matching archive \textit{members} into the executable. However, these come with issues too:

\begin{enumerate}
    \item Duplication in stored executables (e.g. \texttt{libc.a} functions)
    \item Duplication in running executables
    \item Any fix in a library requires importing applications to explicitly relink
\end{enumerate}

\textbf{Shared Libraries} solve this: These are linked at load-time or during run-time. Another advantage is that \textit{multiple} processes can use the same shared library simultaneously. This is how, for example, \texttt{libc} is packaged.

During runtime, shared libraries can be loaded using \texttt{dlopen}:

\inputcodewithfilename{c}{}{code-examples/00_c/04_toolchain/01_dynamic_linking.c}

\newpage

\subsection{File types}

The most common file types used during compilation:

\begin{itemize}
    \item \textbf{Source Code File} (\texttt{.c}) Uncompiled source code in \texttt{C}.
    \item \textbf{Relocatable Object File} (\texttt{.o}) Code \& Data in a format ready for linking.
    \item \textbf{Shared Object File} (\texttt{.so}) Special object file, can be loaded \& linked dynamically: at load or run time.
    \item \textbf{Exectuable File} Code \& Data in a format that can be directly copied into memory \& run.
    \item \textbf{Archive files} (\texttt{.a}) concatenate related \texttt{.o} files into one \texttt{.a}, with an index.
\end{itemize}

\content{Alternate names} On Windows, \texttt{.dll} files are used instead of \texttt{.so} and are called \textit{Dynamic Link Libraries}.

\subsubsection{Executable and Linkable Format (ELF)}

The standard unified format for all object files (\texttt{.exe}, \texttt{.o}, \texttt{.so}) in use since UNIX.

\begin{center}
    \begin{tabular}{l|l}
        \textbf{Section}        & \textbf{Content} \\
        \hline
        ELF header              & contains basic information: Word size, byte ordering, file type, machine type \\
        Segment header table    & page size, virtual address memory segments, segment sizes \\
        \texttt{.text}          & actual code \\
        \texttt{.rodata}        & (Read-only data) for example, jump tables \\
        \texttt{.data}          & initialized global variables \\
        \texttt{.bss}           & uninitialized global variables \\
        \texttt{.symtab}        & symbol table, procedure and static variable names, section names \& locations. \\
        \texttt{.rel.text}      & relocation info for \texttt{.text}, e.g. addresses of instructions that need modifying \\
        \texttt{.rel.data}      & relocation info for \texttt{.data}, e.g. addresses of pointers that need modifying \\
        \texttt{.debug}         & info for symbolic debugging (\texttt{gcc -g})
    \end{tabular}
\end{center}

% Compact table with only the segment names

% \begin{center}
%     \begin{tabular}{|c|}
%         \hline
%         ELF header \\
%         \hline 
%         Segment header table \\
%         \hline
%         \texttt{.text} \\
%         \hline 
%         \texttt{.rodata} \\
%         \hline
%         \texttt{.data} \\
%         \hline
%         \texttt{.bss} \\
%         \hline
%         \texttt{.symtab} \\
%         \hline
%         \texttt{.rel.txt} \\
%         \hline
%         \texttt{.rel.data} \\
%         \hline
%         \texttt{.debug} \\
%         \hline
%         Section header table \\
%         \hline
%     \end{tabular}
% \end{center}



% ── Hardware recap ──────────────────────────────────────────────────
\newsection
\section{Hardware}
\newsection
\subsection{Trigonometrische Interpolation}
\subsubsection{Von Approximation zur Interpolation}
Wir erinnern uns daran, dass wir die Fourier-Approximation durch den Abbruch der unendlichen Fourier-Reihe erhalten, oder in anderen Worten, wir verkleinern die Limiten der Summe.

\fancyremark{DFT mit $N = 2n$ Koeffizienten an Punkten $\frac{l}{N}$ für $l = 0, 1, \ldots, N - 1$}

Der Shift ist hier gegeben durch (für $k \geq 0$ ist $\gamma_k = \hat{f}_N(k)$ und für $k < 0$ ist $\gamma_k = \hat{f}_N(N + k)$)
\begin{align*}
    f_{N - 1}(x)                 & = \sum_{k = -n}^{n - 1} \gamma_k e^{2 \pi ikx} = \sum_{k = 0}^{n - 1} \gamma_k e^{2\pi ikx} + \sum_{k = -n}^{-1} \gamma_k e^{2\pi ikx} \\
    \Leftrightarrow f_{N - 1}(x) & = \frac{1}{N} \left( \sum_{j = 0}^{N - 1} \left( f\left( \frac{j}{n} \right)
        \sum_{k = -n}^{n - 1} e^{2\pi ik \left( x - \frac{j}{N} \right)} \right) \right)
\end{align*}

\vspace{-1pc}

Wenn wir die Funktion nun an der Stelle $\frac{l}{N}$ auswerten so erhalten wir:
\rmvspace
\begin{align*}
    f_{N - 1}\left( \frac{l}{N} \right) = \ldots = f\left( \frac{l}{N} \right)
\end{align*}

\vspace{-1.8pc}
was aufgrund der Orthogonalität der diskreten Fourier-Vektoren funktioniert, welche besagt, dass $\displaystyle \sum_{k = -n}^{n - 1} \omega_N^{k(j - l)} = 0$, für alle $j \neq l$.
Für $j = l$ ergibt die Summe $N$.

Dies heisst also, dass die Fourier-Approximation die Interpolationsbedingungen an den Punkten $\frac{l}{N}$ erfüllt,
also können wir die Lösung der Interpolationsaufgabe $p_{N - 1} \left( \frac{l}{N} \right) = f\left( \frac{l}{N} \right)$ f $l = 0, 1, \ldots, N - 1$ im Raum
\rmvspace
\begin{align*}
    \mathcal{T}_N = \text{span}\{ e^{2\pi ijt} \divides j = - \floor{\frac{N - 1}{2}}, \ldots, \floor{\frac{N}{2}} \}
\end{align*}

\rmvspace\rmvspace
folgendermassen finden können: 
\begin{enumerate}[label=(\arabic*)]
    \item Mittels Gleichungssystem $\sum_{j} \gamma_j e^{2\pi ijt_l} = f(t_l)$ für $l = 0, \ldots, N - 1$. Operationen: $\tco{N^3}$
    \item Mittels FFT in $\tco{N \log(N)}$ Operationen, aber nur falls die Punkte äquidistant sind, also $t_l = \frac{l}{N}$.
        Dann ist die Matrix des obigen Gleichungssystems $F^{-1}_N$
\end{enumerate}

\vspace{0.2cm}

Unten findet sich Python code der mit den unterschiedlichen Methoden die Koeffizienten des Trigonometrischen Polynoms bestimmt.
\rmvspace
\begin{code}{python}
    def get_coeff_trig_poly(t: np.ndarray, y: np.ndarray):
    N = y.shape[0]
    if N % 2 == 1:
    n = (N - 1.0) / 2.0
    M = np.exp(2 * np.pi * 1j * np.outer(t, np.arange(-n, n + 1)))
    else:
    n = N / 2.0
    M = np.exp(2 * np.pi * 1j * np.outer(t, np.arange(-n, n)))
    c = np.linalg.solve(M, y)
    return c

    N = 2**12
    t = np.linspace(0, 1, N, endpoint=False)
    y = np.random.rand(N)
    direct = get_coeff_trig_poly(t, y)
    using_fft = np.fft.fftshift(np.fft.fft(y) / N)
    using_ifft = np.conj(np.fft.fftshift(np.fft.ifft(y)))
\end{code}

\subsection{Code Optimizations}

By understanding the underlying hardware, simple code optimizations can be made to improve performance significantly.

To analyze program performance, a solid benchmark and performance metrics need to be chosen.

Common metrics are:
\begin{enumerate}
    \item \textbf{Latency} i.e. how long does a request take?
    \item \textbf{Throughput} i.e. how many requests (per second) can be processed?
\end{enumerate}

For pipelined processors, throughput is more tricky to specify. Usually, the steady-state throughput is considered.

\inlinedef \textbf{Cycles per Element} (CPE) is used as a performance metric for operations on vectors/lists. \\
The execution time $t$ can then be formulated as: $t = \text{CPE} \cdot n + \text{Overhead}$. ($n$ is the vector/list size)

\inlinedef \textbf{Cycles per Instruction} (CPI) is the inverse of \textit{Instructions per Cycle}

CPI can be further divided into $\text{CPI} = \text{CPI}_\text{Base} + \text{CPI}_\text{Stall}$ assuming a pipelined processor, which may stall for many reasons: Data hazards, control hazards, memory latency ...

\inlinedef \textbf{Clock Cycle Time} (CCT) is the inverse of \textit{Clock Frequency}

Using these we can define Program Execution Time: $t = n \cdot \text{CPI} \cdot \text{CCT}$ ($n$ is the instruction count)

\inlinedef \textbf{Instruction Level Parallelism} (ILP) is the natural parallelism occuring through independent instructions.

\textit{Superscalar} Processors capitalize on ILP by executing multiple instructions per cycle. The instructions are (usually) scheduled dynamically. This is particularly great since it requires no effort from the programmer.

Different instructions generally may take vastly different amounts of time, though this can be reduced via pipelining and multiple execution units. 

\content{Example} This table is specific to an Intel Haswell CPU.

\begin{center}
    \begin{tabular}{l|c|c}
        \textbf{Instruction} & \textbf{Latency} & \textbf{Cycles/issue} \\
        \hline
        Load/store            & 4     & 1     \\
        Int Add               & 1     & 1     \\
        Int Multiply          & 3     & 1     \\
        Int/Long Divide       & 3--30 & 3--30 \\
        FP Multiply           & 5     & 1     \\
        FP Add                & 3     & 1     \\
        FP Divide             & 3--15 & 3--15 \\
    \end{tabular}    
\end{center}

Performance of these operations is \textbf{Latency bound} if they are sequential, \textbf{Throughput bound} if they can run parallel.

\content{Example} \textbf{Loop unrolling} can increase performance for latency bound operations. (Optimization flags \textit{should} do this)

Unrolling can be done to an arbitrary degree, but there are diminishing returns at some point. An ideal unrolling factor must be found experimentally.

\content{Example} \textbf{Reassociation} can also improve performance for parallelizable operations, if it breaks some sequential dependency.

\content{Example} \textbf{Separate Accumulators} can improve performance for latency bound operations too, as separate load/store units may be used.

\newpage
\subsection{Vector Operations}

Extreme performance gains beyond the results of the previous section can be gained using hardware vector registers on supported CPUs.

\content{Example} In Intel AVX2, $256$b vector registers like \verb|%ymm0|, \verb|%ymm1| can be used to perform component-wise single/double precision FP operations.
\begin{minted}{gas}
    vaddsd  %ymm0, %ymm1, %ymm1     # Comp.-wise 32b FP add
    vaddsd  %ymm0, %ymm1, %ymm1     # Comp.-wise 64b FP add
\end{minted}

\newsection
\subsection{Trigonometrische Interpolation}
\subsubsection{Von Approximation zur Interpolation}
Wir erinnern uns daran, dass wir die Fourier-Approximation durch den Abbruch der unendlichen Fourier-Reihe erhalten, oder in anderen Worten, wir verkleinern die Limiten der Summe.

\fancyremark{DFT mit $N = 2n$ Koeffizienten an Punkten $\frac{l}{N}$ für $l = 0, 1, \ldots, N - 1$}

Der Shift ist hier gegeben durch (für $k \geq 0$ ist $\gamma_k = \hat{f}_N(k)$ und für $k < 0$ ist $\gamma_k = \hat{f}_N(N + k)$)
\begin{align*}
    f_{N - 1}(x)                 & = \sum_{k = -n}^{n - 1} \gamma_k e^{2 \pi ikx} = \sum_{k = 0}^{n - 1} \gamma_k e^{2\pi ikx} + \sum_{k = -n}^{-1} \gamma_k e^{2\pi ikx} \\
    \Leftrightarrow f_{N - 1}(x) & = \frac{1}{N} \left( \sum_{j = 0}^{N - 1} \left( f\left( \frac{j}{n} \right)
        \sum_{k = -n}^{n - 1} e^{2\pi ik \left( x - \frac{j}{N} \right)} \right) \right)
\end{align*}

\vspace{-1pc}

Wenn wir die Funktion nun an der Stelle $\frac{l}{N}$ auswerten so erhalten wir:
\rmvspace
\begin{align*}
    f_{N - 1}\left( \frac{l}{N} \right) = \ldots = f\left( \frac{l}{N} \right)
\end{align*}

\vspace{-1.8pc}
was aufgrund der Orthogonalität der diskreten Fourier-Vektoren funktioniert, welche besagt, dass $\displaystyle \sum_{k = -n}^{n - 1} \omega_N^{k(j - l)} = 0$, für alle $j \neq l$.
Für $j = l$ ergibt die Summe $N$.

Dies heisst also, dass die Fourier-Approximation die Interpolationsbedingungen an den Punkten $\frac{l}{N}$ erfüllt,
also können wir die Lösung der Interpolationsaufgabe $p_{N - 1} \left( \frac{l}{N} \right) = f\left( \frac{l}{N} \right)$ f $l = 0, 1, \ldots, N - 1$ im Raum
\rmvspace
\begin{align*}
    \mathcal{T}_N = \text{span}\{ e^{2\pi ijt} \divides j = - \floor{\frac{N - 1}{2}}, \ldots, \floor{\frac{N}{2}} \}
\end{align*}

\rmvspace\rmvspace
folgendermassen finden können: 
\begin{enumerate}[label=(\arabic*)]
    \item Mittels Gleichungssystem $\sum_{j} \gamma_j e^{2\pi ijt_l} = f(t_l)$ für $l = 0, \ldots, N - 1$. Operationen: $\tco{N^3}$
    \item Mittels FFT in $\tco{N \log(N)}$ Operationen, aber nur falls die Punkte äquidistant sind, also $t_l = \frac{l}{N}$.
        Dann ist die Matrix des obigen Gleichungssystems $F^{-1}_N$
\end{enumerate}

\vspace{0.2cm}

Unten findet sich Python code der mit den unterschiedlichen Methoden die Koeffizienten des Trigonometrischen Polynoms bestimmt.
\rmvspace
\begin{code}{python}
    def get_coeff_trig_poly(t: np.ndarray, y: np.ndarray):
    N = y.shape[0]
    if N % 2 == 1:
    n = (N - 1.0) / 2.0
    M = np.exp(2 * np.pi * 1j * np.outer(t, np.arange(-n, n + 1)))
    else:
    n = N / 2.0
    M = np.exp(2 * np.pi * 1j * np.outer(t, np.arange(-n, n)))
    c = np.linalg.solve(M, y)
    return c

    N = 2**12
    t = np.linspace(0, 1, N, endpoint=False)
    y = np.random.rand(N)
    direct = get_coeff_trig_poly(t, y)
    using_fft = np.fft.fftshift(np.fft.fft(y) / N)
    using_ifft = np.conj(np.fft.fftshift(np.fft.ifft(y)))
\end{code}

\subsubsection{Performance Metrics}
\begin{itemize}[noitemsep]
    \item \bi{Miss Rate} is the fraction of memory references that are not found in the cache. Defined as $\displaystyle \frac{\text{misses}}{\text{accesses}} = 1 - \text{hit rate}$
          and is typically 3-10\% in L1 caches and less than 1\% in L2 caches.
    \item \bi{Hit Time} is the time required to deliver a line in the cache to the processor and typically is 1-2 cycles for L1 cache and 5-20 cycles for L2 caches.
    \item \bi{Miss penalty} is the additional time required when a cache miss occurs. Typically around 50-200 cycles
\end{itemize}
Judging by these numbers, it makes a huge difference if we hit the cache and the speed difference can easily exceed a factor of 100x.
Of note is as well that a $99\%$ hit rate is twice as good as a $97\%$ hit rate with a miss penalty of 100 cycles and a cache hit time of 1 cycle:
\begin{itemize}[noitemsep]
    \item \bi{97\% hits:} $1 \text{ cycle} + 0.03 \cdot 100 \text{ cycles} = 4\text{ cycles}$
    \item \bi{99\% hits:} $1 \text{ cycle} + 0.01 \cdot 100 \text{ cycles} = 2\text{ cycles}$
\end{itemize}
Thus, we always use \textit{miss rate} instead of hit rate.

For a multi-level cache, we start with the last level cache and compute its miss penalty and combine that with the next higher level and so on (example with 2 level cache)
\rmvspace
\begin{align*}
    \text{MissPenaltyL2} & = \text{DRAMaccessTime} + \frac{\text{BlockSize}}{\text{Bandwidth}} \\
    \text{MissPenaltyL1} & = \text{HitTimeL2} + \text{MissRateL2} \cdot \text{MissPenaltyL2}\\
    \text{AverageMemoryAccessTime} &= \text{HitTimeL1} + \text{MissRateL1} \cdot \text{MissPenaltyL1}
\end{align*}

\newpage
\subsubsection{Cache misses}
\begin{itemize}[noitemsep]
    \item \bi{Compulsory / Cold miss} Occurs on the first access of a block (there can't be any data there yet)
    \item \bi{Conflict miss} The cache may be large enough, but multiple lines may map to the current block,
          e.g. referencing blocks 0, 8, 0, 8, \dots would miss every time if they are both mapped to the same cache line.
          This is the typical behaviour for most caches.
    \item \bi{Capacity miss} The number of active cache blocks is larger than the cache
    \item \bi{Coherency miss} See in section \ref{sec:hw-multicore}.
          They happen if cache lines have to be invalidated due in multiprocessor scenarios to preserve sequential consistency, etc
\end{itemize}

\subsubsection{Cache organization}
\content{Structure} Caches can be defined using $S, E, B$ such that $S \cdot E \cdot B = \text{Cache Size}$.\\
$S = 2^s$ is the set count, $E = 2^e$ is the lines per set, and $B=2^b$ is the number of bytes per cache block / cache line\footnote{
    The terms \textit{cache block} and \textit{cache line} will be used interchangeably and sometimes shortened to just \textit{block} or \textit{line}}.

Each block has the following structure where \texttt{v} is the valid bit:
\begin{center}
    Cache line:
    \begin{tabular}{|c|c|c|}
        \hline
        v & tag & data ($B = 2^b$ bytes per block) \\
        \hline
    \end{tabular}
\end{center}

\content{Address} The cache address can be separated into fields which dictate the cache location:
\begin{center}
    Address:
    \begin{tabular}{|c|c|c|}
        \hline
        tag & set index & block offset \\
        \hline
    \end{tabular}
\end{center}
Since we have $S=2^s$ sets and $B=2^b$ bytes per block, we need $s$ bits for the set index, $b$ bits for block offset.
The remaining part (tag) is stored with the cache block and needs to match for a cache hit.

Do note that the cache address is the same as the physical memory address, we just refer to it as cache address when talking about the interpretation the cache uses for it.
See section \ref{sec:hw-cache-set} for more information


Higher cache associativity helps reduce the number of conflict misses (often) by increasing the number of lines available for each block.
It however, at the same cache size, reduces the number of available cache sets, which in turn may increase the number of conflict misses.

\inlinedef \textbf{Direct-mapped} i.e. $E = 1$ (1 cache line per set only).

\inlinedef \textbf{$N$-way Set-Associative} i.e. $E = N$ ($N$ cache lines per set, in this course we primarily covered $N = 2$).


% ────────────────────────────────────────────────────────────────────

\subsubsection{Determining cache set from memory address}
\label{sec:hw-cache-set}
The cache set a memory address is mapped to does not depend (directly) on the associativity of the cache\footnote{
    The only dependence it has on the associativity is that with the same cache size, the number of sets is reduced if the associativity is increased}
To be able to compute the cache block that a memory request maps to, we need to know the number of sets and the line size.

We can then compute the set it maps to using \texttt{x \% S}, where \texttt{x = addr >> b} is the block number in the memory.

Thus, in the cache address, the tag corresponds to \texttt{y = x / S}, so it indicates which actual memory location is stored in the cache
(as multiple different memory block numbers will map to the same cache set).

\newpage
\subsubsection{Memory reads}
\content{In Direct-Mapped caches}
\begin{enumerate}[noitemsep]
    \item We find the cache set and check if the valid bit is set. If not, skip to step 3, false
    \item Check if the tag is equal to the requested tag.
    \item If true, return the block at the correct offset, else evict the line and fetch the correct line (will be $B$ bytes) and return the block at the correct offset.
\end{enumerate}


\content{In $2$-way Set-Associative caches}
\begin{enumerate}[noitemsep]
    \item Find the corresponding cache set and compare the tag to both blocks.
    \item If one matches, check its valid bit. If none match, go to step 5
    \item If valid, return the block at the correct offset.
    \item If invalid, evict the line, fetch the correct one. Return to step 3
    \item If no match, evict one of the two (choose using a replacement policy like LRU (Least Recently Used) or randomly if the other block is not invalid),
          fetch requested line, go to step 3.
\end{enumerate}

\subsubsection{Memory writes}
Memory writes are just as slow, if not sometimes slower than memory reads.
Thus, they also have to be cached to improve performance.
Again, there are a few options to handle write caching and we will cover the two most prevalent ones:
\begin{itemize}
    \item \bi{Write-through} Here, the data is immediately written to main memory.
          The obvious benefit is that the data is always up-to-date in the memory as well, but we do not gain any speed from doing that, it is thus very slow.
    \item \bi{Write-back} We defer the write until the line is replaced (or sometimes other conditions or an explicit cache flush).
          The obvious benefit is the increased speed, as we can benefit from the low cache access latency.
          We do however need a \textit{dirty bit} to indicate that the cache line differs from main memory.
          This introduces additional complexity, especially in multi-core situations, which is why in that case,
          often a write-through mode is enabled for the variables that need atomicity.
\end{itemize}

Another question that arises is what to do on a \textit{write-miss}?
\begin{itemize}
    \item \bi{Write-allocate} The data is loaded into the cache and is beneficial if more writes to the location follow suite.
          It is however harder to implement and it may evict an existing value from the cache.
          This is commonly seen with write-back caches.
    \item \bi{No-write-allocate} This writes to the memory immediately, is easier to implement, but again slower, especially if the value is later re-read.
          This is commonly seen with write-through caches.
\end{itemize}

\subsubsection{Writing fast code}
Improving the speed of the code can be done by optimizing the code's locality (both temporal and spatial locality).

Computing a matrix product or iterating over a list can be much faster depending on the way you access the array and how it is stored.

A common tactic to improve the throughput for these kinds of operations is to make sure that the order of operations is correct
(i.e. for a row-major matrix to iterate over the elements of a row in the inner loop),
or to do block multiplication where you multiply small blocks at once.

\subsubsection{Cache Addressing Schemes}
\label{sec:hw-addressing-schemes}

The cache can see either the virtual or physical address, and the tag and index do \textit{not} need to both use the physical/virtual address.
If this seems confusing, first read and understand section \ref{sec:hw-virt-mem}

\begin{center}
    \begin{tabular}{c|c|c}
        \hline
        \textbf{Indexing}  & \textbf{Tagging}  & \textbf{Code} \\
        \hline
        Virtually Indexed  & Virtually Tagged  & VV            \\
        Virtually Indexed  & Physically Tagged & VP            \\
        Physically Indexed & Virtually Tagged  & PV            \\
        Physically Indexed & Physically Tagged & PP
    \end{tabular}
\end{center}

\newsection
\subsection{Trigonometrische Interpolation}
\subsubsection{Von Approximation zur Interpolation}
Wir erinnern uns daran, dass wir die Fourier-Approximation durch den Abbruch der unendlichen Fourier-Reihe erhalten, oder in anderen Worten, wir verkleinern die Limiten der Summe.

\fancyremark{DFT mit $N = 2n$ Koeffizienten an Punkten $\frac{l}{N}$ für $l = 0, 1, \ldots, N - 1$}

Der Shift ist hier gegeben durch (für $k \geq 0$ ist $\gamma_k = \hat{f}_N(k)$ und für $k < 0$ ist $\gamma_k = \hat{f}_N(N + k)$)
\begin{align*}
    f_{N - 1}(x)                 & = \sum_{k = -n}^{n - 1} \gamma_k e^{2 \pi ikx} = \sum_{k = 0}^{n - 1} \gamma_k e^{2\pi ikx} + \sum_{k = -n}^{-1} \gamma_k e^{2\pi ikx} \\
    \Leftrightarrow f_{N - 1}(x) & = \frac{1}{N} \left( \sum_{j = 0}^{N - 1} \left( f\left( \frac{j}{n} \right)
        \sum_{k = -n}^{n - 1} e^{2\pi ik \left( x - \frac{j}{N} \right)} \right) \right)
\end{align*}

\vspace{-1pc}

Wenn wir die Funktion nun an der Stelle $\frac{l}{N}$ auswerten so erhalten wir:
\rmvspace
\begin{align*}
    f_{N - 1}\left( \frac{l}{N} \right) = \ldots = f\left( \frac{l}{N} \right)
\end{align*}

\vspace{-1.8pc}
was aufgrund der Orthogonalität der diskreten Fourier-Vektoren funktioniert, welche besagt, dass $\displaystyle \sum_{k = -n}^{n - 1} \omega_N^{k(j - l)} = 0$, für alle $j \neq l$.
Für $j = l$ ergibt die Summe $N$.

Dies heisst also, dass die Fourier-Approximation die Interpolationsbedingungen an den Punkten $\frac{l}{N}$ erfüllt,
also können wir die Lösung der Interpolationsaufgabe $p_{N - 1} \left( \frac{l}{N} \right) = f\left( \frac{l}{N} \right)$ f $l = 0, 1, \ldots, N - 1$ im Raum
\rmvspace
\begin{align*}
    \mathcal{T}_N = \text{span}\{ e^{2\pi ijt} \divides j = - \floor{\frac{N - 1}{2}}, \ldots, \floor{\frac{N}{2}} \}
\end{align*}

\rmvspace\rmvspace
folgendermassen finden können: 
\begin{enumerate}[label=(\arabic*)]
    \item Mittels Gleichungssystem $\sum_{j} \gamma_j e^{2\pi ijt_l} = f(t_l)$ für $l = 0, \ldots, N - 1$. Operationen: $\tco{N^3}$
    \item Mittels FFT in $\tco{N \log(N)}$ Operationen, aber nur falls die Punkte äquidistant sind, also $t_l = \frac{l}{N}$.
        Dann ist die Matrix des obigen Gleichungssystems $F^{-1}_N$
\end{enumerate}

\vspace{0.2cm}

Unten findet sich Python code der mit den unterschiedlichen Methoden die Koeffizienten des Trigonometrischen Polynoms bestimmt.
\rmvspace
\begin{code}{python}
    def get_coeff_trig_poly(t: np.ndarray, y: np.ndarray):
    N = y.shape[0]
    if N % 2 == 1:
    n = (N - 1.0) / 2.0
    M = np.exp(2 * np.pi * 1j * np.outer(t, np.arange(-n, n + 1)))
    else:
    n = N / 2.0
    M = np.exp(2 * np.pi * 1j * np.outer(t, np.arange(-n, n)))
    c = np.linalg.solve(M, y)
    return c

    N = 2**12
    t = np.linspace(0, 1, N, endpoint=False)
    y = np.random.rand(N)
    direct = get_coeff_trig_poly(t, y)
    using_fft = np.fft.fftshift(np.fft.fft(y) / N)
    using_ifft = np.conj(np.fft.fftshift(np.fft.ifft(y)))
\end{code}

\subsubsection{Address Translation}

Address translation happens in a dedicated hardware component: The Memory Management Unit (MMU).
Thus, the CPU can use the virtual memory addresses and does not have to worry about translating to the physical addresses.

Virtual and Physical Addresses share the same structure, but the VPN (Virtual Page Number) is usually far longer than the PPN (Physical Page Number),
since the virtual space is far bigger. The offsets match however.

\begin{multicols}{2}
    \begin{center}
        Virtual:
        \begin{tabular}{|c|c|}
            \hline
            V. Page Number & V. Page Offset \\
            \hline
        \end{tabular}
    \end{center}
    \newcolumn
    \begin{center}
        Physical:
        \begin{tabular}{|c|c|}
            \hline
            P. Page Number & P. Page Offset \\
            \hline
        \end{tabular}
    \end{center}
\end{multicols}

The Page Table (PT) (Located at a special Page Table Base Register (PTBR)) contains the mapping $\text{VPN} \mapsto \text{PPN}$.
Page Table Entries (PTE) are cached in the L1 cache like any other memory word.

The Translation Lookaside Buffer (TLB) is a small hardware cache inside the MMU that is used to accelerate the PT lookups, which is typically faster than an L1 cache hit.
The PT is usually stored in memory\footnote{In practice, most address translations actually hit the TLB.} and contains for each address the corresponding physical address,
as well as a valid bit, which indicates if the page is in memory.

If a page is not in memory a page fault is triggered, which transfers control to the OS, which then loads the page into memory,
updates the page table and returns control back to the process. This and the inverse (i.e. unloading pages from memory onto the disk) is often referred to as \textit{swapping}.

Due to the slowness of disks, page sizes are usually fairly large (4-8KB) typically, in some cases up to 4MB.
The replacement policy algorithms are highly sophisticated and too complicated to be implemented in hardware and are thus usually handled by the operating system.

\content{Address Translation with page hit} The CPU requests a virtual memory address from the MMU.
It fetches the PTE from memory and sends the physical address to the memory system, which sends the data to the CPU.

\content{Address Translation with page fault} During the check of the valid bit of the page table, the MMU find that it is not set.
It thus triggers a page fault exception, which is then handled and a victim page (if necessary) is then picked and evicted (and if the dirty flag is set, it is paged out to disk).
The handler then loads the new page into memory and updates the PT and the original instruction is then restarted on the CPU and the address translation will then succeed.

As already touched on, the TLB can be used to speed up translation.
\content{Address Translation with TLB Hit} When checking the TLB for the entry, it find the entry and we save one memory access.
With the PPN retrieved, the memory system sends the data to the CPU.

\content{Address Translation with TLB Miss} This works similar to the case when there is no TLB, as the TLB returns a miss signal for the request.
Only that the PTE that is returned is inserted into the TLB via replacement policy (if applicable). The data is then fetched from the physical address and sent to the CPU.


\content{Example} We consider $N=14$ bit virtual addresses and $M=12$ bit physical addresses. The offset takes $6$ bits.\footnote{The images in this example are from the SPCA lecture notes for FS25.}

If we assume a TLB with $16$ entries, and $4$ way associativity, the VPN translates like this: \scriptsize(where \texttt{TLBT = Tag} and \texttt{TLBI = Set})\normalsize

\begin{center}
    \includegraphics[width=0.7\linewidth]{images/VPN-to-TLB.png}
\end{center}

Similarly, if we assume a direct-mapped $16$ line cache with $4$ byte blocks: \scriptsize(where \texttt{CT = Tag}, \texttt{CI = Set} and \texttt{CO = Offset})\normalsize

\begin{center}
    \includegraphics[width=0.65\linewidth]{images/PPN-to-Cache.png}
\end{center}

\subsubsection{Multilevel Page Tables}
\content{Motivation} For a 48-bit Virtual Address Space with $4$KB ($= 2^{12}$ bytes) page size, the number of bits required is $2^{48} / 2^{12} \cdot 2^3 = 2^{39}$ bytes
(that is 512 GB). The $2^{3}$ bytes is the size of the page table entry (8 bytes).

Multi-Level page tables add further steps to this process: Instead of a PT we have a Page Directory Table (PDE) which contains the addresses of separate Page Tables.
The top of the VPN is used to index into each of these, which technically allows any depth of page tables.

\subsubsection{x86 Virtual Memory}

In \verb|x86-64| Virtual Addresses are $48$ bits long, yielding an address space of $256$TB.\\
Physical Addresses are $52$ bits, with $40$ bit PPNs, yielding a page size of $4KB$ (we thus have $64$ bit PTEs).

On the slides, they are again using (as far as we can tell) a Skylake CPU (Core 6000 series, could also be Kaby Lake, Core 7000 series).

On that architecture, the TLB contained the 40 bit PPN, a 32 bit TLB Tag, as well as
\begin{itemize}
    \item a valid bit (\texttt{V})
    \item a global bit (\texttt{G}, coped from PDE / PTE and prevents eviction)
    \item a supervisor-only bit (\texttt{S}, i.e. only accessible to OS, copied from PDE / PTE)
    \item a writable bit (\texttt{W}, page is writable, copied from PDE / PTE)
    \item a dirty bit (\texttt{D}, PTE has been marked dirty (i.e. modified vs memory))
\end{itemize}
There are a number of flags set and there are also a significant number of bits available for systems programmers to use on \texttt{x86}.
Since they are highly unlikely to be exam-relevant, we will only point out that there are a lot of them (including setting supervisor mode, read/write mode, dirty, etc).
To view them all, find them in the lecture slides of lecture 20, pages 87 through 90.

For many years, cache sizes have been stagnant, this was due to the limited number of bits that could be used to efficiently determine if a line is available in the cache.
Today, there are techniques to overcome that limitation and we have seen fairly substantial increases in cache sizes since
(primarily from Team Red, starting with the AMD Ryzen 7 5800X3D).

This was caused by the fact that only 6 bits from the PPO were used to determine the set and not more to improve performance,
as the cache indexing could occur during address translation.

\content{Addressing Schemes Revisited}
Returning to the Addressing Schemes from section \ref{sec:hw-addressing-schemes}, it becomes evident that this is the key to solving the issue just touched on.
If we virtually tag and virtually index the address, we have the issue that there may exist multiple multiple PA for each VA (i.e. it is context dependent).
To circumvent that issue, an ASID (Address Space Identifier) is added to the tag.

The Virtually Indexed, physically tagged scheme is what we have just seen and is commonly used for L1 caches.

The Physically Tagged, Physically Indexed is the solution to the cache size restriction.
It however suffers from slower access times, as the address translation has to complete before the cache line can be identified.

\content{Write buffers} It is also common to have write buffers (which act like FIFO queue).
It enables slower cache operations to complete that are typically associated with writing.

\content{Large pages} Simply a lower number of bits remain in the PPN and some bits are ignored (we increased the ``offset-portion'' to 21 bits = 2MB).
We can increase that to a page size of 1GB, if we increase the ``offset-portion'' further, to 30 bits to be precise.

% \input{parts/03_hw/04_virtual-memory/}
\newpage
\subsection{Exceptions}

Control flow is mainly dictated by program state, and manipulated using jumps, branches, calls and returns. 
To react to changes in system state, exceptional control flow is used instead.

\textbf{Low level mechanisms}
\begin{itemize}
    \item Hardware Exceptions
    \item Exceptions via combination of Hardware and OS software
\end{itemize}

\textbf{High level mechanisms}
\begin{itemize}
    \item Process context switch
    \item Signals
    \item Nonlocal jumps
    \item Language-level exceptions (e.g. Java)
\end{itemize}

Generally, on an exception, control is transferred to a handler specific to the type of exception, which investigates the situation and returns control upon success.
Mostly, this is handled via a \textit{Exception Table} which is allocated on boot. On exception, this table is indexed depending on the type of exception to locate the corresponding handler. This causes a switch to Kernel Mode.

\inlinedef \textbf{Exception}: A control transfer to the OS in reponse to an event
\begin{itemize}
    \item \textbf{Synchronous}: result of executing some instruction
    \item \textbf{Asynchronous}: result of an event external to the processor
\end{itemize}

\begin{center}
    \begin{tabular}{l|l|l}
        \textbf{Type of exception} & \textbf{Cause} & \textbf{Async/Sync} \\
        \hline
        Interrupt & Signal from I/O device          & Async     \\
        Trap      & Intentional exception           & Sync      \\
        Fault     & Potentially recoverable error   & Sync      \\
        Abort     & Nonrecoverable error            & Sync      \\
    \end{tabular}
\end{center}

\subsubsection{Synchronous Exceptions}

\inlinedef \textbf{Trap} is an intentional exception that transfers control back to the next instruction

For example, opening a file in \verb|C| executes a trap via a system call.

\inlinedef \textbf{Fault} is an unintentional, possibly recoverable exception. Either re-executes faulty instruction or aborts

For example, page faults, protection faults, floating point exceptions

\inlinedef \textbf{Abort} is unintentional and unrecoverable. Always aborts the program.

For example, a machine error.

\subsubsection{Asynchronous Exceptions}

Asynchronous Exceptions are indicated by setting the processor's (physical) interrupt pin.

For example, 
\begin{itemize}
    \item \textbf{Interrupts} are actions like network data arrival or hitting a key on the keyboard
    \item \textbf{Hard Reset Interrupts} are executed by hitting the system reset button
    \item \textbf{Soft Reset Interrupts} are caused by, for example, hitting \verb|CTRL|+\verb|ALR|+\verb|DEL| (on Windows)
\end{itemize}

\newpage
\subsection{Multi-Core}
\subsubsection{Background}
In the early days of computer hardware it was fairly easy to get higher performance due to the rapid advances in transistor technology.
However, today, what is known as Moore's Law (i.e. that the transistor count of integrated circuits doubles every two years).
However, due to power constraints and the slowing down of advances in transistor technology,
the transistor count growth has slowed down quite a bit since the beginning of the century and is predicted to further stagnate as time goes on.

This leads to various issues, among others, the performance of CPUs isn't going up as quickly anymore as it used to.
Additionally, due to power constraints, building faster and faster single-core CPUs is not possible and the advances in that field have slowed to a crawl.

To mitigate and offset these issues, manufacturers started to add multiple cores to parallelize operations.
This however brings a whole host of new issues with it, for example, how do you make sure that no data races occur,
how do you schedule, etc? These questions have mostly been answered in the course Parallel Programming, so we will not cover that here.

The only reason transistor count is still growing at a seemingly constant rate today is that manufacturers manage to cram more and more cores into a CPU.
But even that has slowed down in recent years.

While in 2019 a highest core count AMD EPYC CPU (i.e. the EPYC 7742 from the ROME family) had 64 Zen 2 cores,
in 2025 the highest core count EPYC CPU (i.e. the EPYC 9965 from the EPYC Turin Dense Family) had 192 Zen 5c cores,
where the highest full core CPU was the EPYC 9755 (from the EPYC Turin family), which had 128 Zen 5 cores.

The way they manage this while not hitting the power wall is by making the CPUs physically larger.
While a consumer Ryzen 9 9950X3D (the fastest consumer CPU at the time of writing) easily fits into the palm of even a small hand,
an EPYC Turin CPU is so large that it covers most of even a big hand.

\subsubsection{Limitations}
\content{The Power Wall} More and more transistors need more and more power, thus leading to power delivery and dissipation becoming and issue.
To compute the power dissipation, use the formula $P_{diss} = P_{dyn} + P_{leak} + P_{short}$,
where $P_{dyn} = C V^2 f$ (with $C$ the capacitance, $V$ the supply voltage and $f$ the processor frequency) is the dynamic power,
$P_{leak}$ the leakage power (see DDCA) and $P_{short}$ the short circuit power while switching.

At some point the chip becomes almost impossible to cool. A great example of a CPU series that suffers from this is the Intel Rocket Lake CPUs.
The Intel Core i9-14900K is notoriously hot-running, using almost 300 watts for a very small chip and thus runs very hot.

Thus, to further increase performance, chip designers are trying to make the hardware more efficient, which allows them to further boost performance with extra power headroom.

\content{The Memory Wall} Between 1985 and 2005, CPU performance has increased on average by 55\% a year, whereas memory throughput has only increased by roughly 10\% a year.
Thus, performance has more and more become limited by memory performance rather than pure CPU performance and to this day is the largest overhead in most applications.

\content{The ILP Wall} While it is possible to improve single core performance using instruction-level parallelism,
this has been thoroughly exhausted and is not a feasible way to significantly improve CPU performance.

Around 2003, all of these walls were hit simultaneously, as they hit a power wall and thus could not clock the processors any higher,
the memory access times were the limiting factors and ILP was almost completely exhausted, as not enough parallel instructions existed in code.

Current trends are a reduction in clock frequency in favour of more parallelism in the hardware, e.g. by providing more cores, or better caching, branch prediction, etc.

\subsubsection{Coherency and Consistency}
\fancydef{Coherency} The values in cache all match each other and the processors all see a coherent view of the memory

\fancydef{Consistency} The order in which changes are seen by different processors is consistent

Most modern system's CPU cores are caches coherent, i.e. it behaves as if all cores access a single memory array.
This leads to one big advantage: It is easy to program, however is hard to implement in hardware and memory is also slower as a result.

Memory consistency on the other hand is not standardized across companies

\input{parts/03_hw/06_multicore/03_relaxing-seq-consistency.tex}
\newpage
\subsubsection{Multicore synchronization}
There are two main ways to synchronize, which are:
\begin{enumerate}
    \item \bi{Atomic operations} such as \texttt{TAS}, \texttt{CAS}, etc.
          It does still have ordering constraints specified in the memory model
    \item \bi{Interprocessor interrupts} (IPIs) This invokes the interrupt handler on remote CPU,
          but is VERY slow (500+ cycles) and thus often avoided, except in the OS
\end{enumerate}

\content{TAS} (Test-and-Set) We can only set to the memory location using TAS if said location is $0$.
It can thus be used for a mutex, with a simple spinlock, which is simple to implement and often the fastest if the lock isn't held for long.
Since we most commonly do not read a value of \texttt{0} in the lock memory location, we can use a TATAS (Test And Test-and-Set) lock to reduce the performance overhead.

\inputcodewithfilename{c}{}{code-examples/03_hw/01_tas.c}

A word of caution: Do not use TAS to check if a value has changed outside a lock.
It will most likely not not work in \lC\ and almost certainly not in \texttt{Java} or any higher level languages


\content{CAS} (Compare-and-Swap)

\subsubsection{Symmetric Multiprocessing}
SMP allows multiple cores to access the same memory. 
It does still allow each core to have a separate cache, which is a de-facto requirement for it to work, as otherwise the memory becomes an even more serious bottleneck.

However, even with all the cache optimizations, the memory is still the bottleneck in SMP.
The MOESI protocol can alleviate some of the slowness by enabling reads to be serviced by other caches,
but that can again slow that cache down.
Additionally, memory accesses can stall the current processor, as well as other processors while accessing data.

So, to reduce idle times, we would like to issue instructions to the Functional Units (FUs), but Instruction Level Parallelism (ILP) is limited due to data dependencies.

This is where SMT (Simultaneous Multithreading) comes into play. As of 2025, only AMD's consumer hardware features SMT, Intel has abandoned it with Arrow Lake (Core Ultra series).
On most platforms, especially Intel platforms up to there, SMT was often referred to as Hyperthreading.
It uses the fact that there are likely operations that can be served by cache in other threads and we can keep the FUs busy.

There are three main ways of achieving SMT:
\begin{itemize}
    \item \bi{Thread IDs} There are multiple independent instruction streams with each having their own thread IDs for later reassignment.
    \item \bi{Fine-grained MT} On every instruction dispatch, a thread to dispatch from is picked
    \item \bi{Coarse-grained MT} When a memory stall is encountered, switch to another thread
\end{itemize}
Of note is that CPU cores with multithreading appear to the OS as multiple cores, and in the common case of what AMD and Intel (up to Arrow Lake) do (did),
they would appear as two cores, which is also where the ``Thread'' count comes from on CPU spec sheets.

As nice and good of an idea as this sounds, it isn't necessarily faster, as threads might be competing for cache, which is why Intel has abandoned SMT altogether
and they are seeing quite competitive performance.
But then again, it is very cheap to implement in terms of extra transistors used, but the performance gain is in the 10-20\% range typically, if even.

Finally, the performance gain strongly depends on the workload. Scientific computing rarely benefits from it, as the compute is often the limiting factor,
however, web servers for example strongly benefit from it, as they are commonly severely memory constrained.

\subsubsection{Non-Uniform Memory Access}
In a typical early SMP architecture, more cores were provided, but not necessarily more cache, or even completely shared cache between the cores.

This is where the NUMA concept comes in, restricting each core, or group of cores to a subset of the memory.
This specifically advantageous in large data centers, where there might be hundreds of CPUs with Terabytes of memory.

Initially, NUMA was constricted to data center uses, but with the Zen microarchitecture, AMD has brought the concept to the consumer market,
where larger and more performant CPUs with larger core counts are split into multiple CCXs (Core CompleXes), each with cache per core and cache per CCX.
Then, there is the Infinity Fabric, which is a CCX interconnect, allowing the cores from the other CCX(s) to access the data in the cache of the current CCX.

However, that is not a full NUMA implementation, as they still share one memory bus.
In a multi CPU deployment (i.e. with multiple CPU sockets containing a CPU), they often have their own memory and memory controller, as well as an interconnect,
allowing them to communicate with the other sockets to access their data in the memory.
In such a deployment, a CPU socket with its own memory is referred to as a NUMA Node.

Let's look at an example with two AMD EPYC 7742 CPUs, each having 64 cores. Each of these CPUs features 128 PCIe Gen 4 lanes,
64 of which could be used for the Infinity Fabric CPU to CPU interconnect (which uses the PCIe Interface).
It is comparatively slow, maxing out approximately at a theoretical 128 GB/s in one direction.
Compare that to the roughly 205 GB/s bandwidth of DDR4 3200 MT/s

\content{Cache coherence} We can no longer snoop on the bus, as it is no bus anymore.
A solution is to emulate a bus, which enables something akin to snooping, but without the shared bus.
Each node sends a message to all other nodes and waits for a reply from all the other nodes before proceeding.
This is the way AMD's Infinity Fabric works (or used to work)

Another option to circumvent the issue is to use a Cache Directory, where we store the data, the node ID of which one is the one it originated from,
plus one bit per node indicating if the line is present in that node.
It is primarily efficient if the lines are not widely shared or if there are lots of NUMA nodes.

\subsubsection{Performance and Optimization}
Depending on the interconnect layout, accessing data from certain nodes is \textit{much} slower than from others.
This especially happens if there is no direct path to the other node from the current node and the access needs to pass through other nodes.

It is also of utmost importance to keep threads on the same CPU to improve locality, which improves cache coverage and try to pack data into a single cache line.
Otherwise, the cache ``ping-pongs''.

\content{MCS locks} There are many ways to improve performance on multiprocessor systems. The MCS lock is one of the best locking system for multiprocessors.
It tries to solve the issue that if a cache line contains a lock, it is continuously invalidated and that dominates the intercom traffic.
The solution to the problem is that a processor enqueues itself on a list of waiting processors and then spins on its \bi{own entry} in the list.
\inputcodewithfilename{c}{}{code-examples/03_hw/02_mcs.c}

\subsection{Devices}

From a programmer's perspective a Device can be seen as:
\begin{itemize}
    \item Hardware assessible via software
    \item Hardware occupying some bus location
    \item Hardware mapping to some set of registers
    \item Source of interrupts
    \item Source of direct memory transfers
\end{itemize}

\subsubsection{Device Registers}

Sometimes devices expose register: The CPU can load from these to obtain e.g. status info or inpur data. 
The CPU can store to device registers to e.g. set device state or write output.

Device registers can be addressed in 2 different ways:
\begin{enumerate}
    \item \textbf{Memory Mapped}: Registers \textit{appear} as memory locations, access via \texttt{movX}
    \item \textbf{IO Instructions}: Special ISA instructions to work with devices
\end{enumerate}

It's important to note: despite \textit{appearing} as memory, device registers behave differently: State may change without CPU manipulation and writes may trigger actions.
The specific way this interaction works is device-specific.

\content{Example} A very simple device driver in \texttt{C} may look like this:

\inputcodewithfilename{c}{code-examples/03_hw/}{00_driver.c}

Of course, a proper driver would also include error handling, initialization and wouldn't spin to wait. To avoid waiting, interrupts are usually used.

\content{Caches} For Device Registers, the cache must be bypassed. Memory mapped IO causes a lot of issues for caching: 
\begin{enumerate}
    \item Reads can't be cached (Value may change independent of CPU)
    \item Write-back doesn't work exactly (Device controls when the write happens)
    \item Reads and writes can't be combined into 1 cache-line
\end{enumerate}

\newpage
\subsubsection{Direct Memory Access}

Direct Memory Access (DMA) requires a dedicated DMA controller, which is generally built-in nowadays. 
DMA allows bypassing the CPU entirely: Data is transferred directly between IO device and memory.
This is especially useful for large transfers, but not small transfers due to the induced overhead.

The key advantage is that data transfer and processing are decoupled.\\ 
The CPU never needs to deal with copying between device and memory, and the CPU cache is never polluted. 

The key disadvantage is that Memory is inconsistent with the CPU cache. This is addressed in various ways:
\begin{enumerate}
    \item CPU may mark DMA buffers as \textit{non-cacheable}
    \item Cache can \textit{snoop} DMA bus transactions (Doesn't scale well, only for small systems)
    \item OS can explicitly flush/invalidate cache regions. (Usually done by a device driver)
\end{enumerate}

Another issue is that DMA addresses are \textit{Physical}. The OS (via device drivers) must \textit{manually} translate these to virtual addresses. 
Some systems also contain a dedicated component called IOMMU that deals with this.

\subsubsection{Device Drivers}

Device drivers are programs used by the OS to communicate with devices. In a nutshell, the driver is the only program that \textit{directly} interacts with the device, any other program talks to the device \textit{through} the driver (which ideally abstracts away a lot of the process).

Intuitively, both the driver and device can be thought of as state machines, which affect eachother.

\inlinedef \textbf{Descriptor Ring} is a type of buffer commonly used to interact with devices. The datastructure is a looped queue (ring): Device reads from the head, OS writes at the tail. The space beyond is then "owned" by the device/OS.

This can either be implemented as contiguous memory or using pointers (which is mainly what is done in practice, for flexibility).
Overruns (Device has no buffers for received packets) and Underruns (CPU has read all received packets) are usually handled sensibly: i.e. the CPU waits for an interrupt or the device will simply wait.

\content{Parallel Programming}: These are producer/consumer queues! But these use messages instead of mutexes and monitors.

% The slides contained a lot of examples and gave an intro to how PCI(e) works, but I don't think it's very relevant



\end{document}
