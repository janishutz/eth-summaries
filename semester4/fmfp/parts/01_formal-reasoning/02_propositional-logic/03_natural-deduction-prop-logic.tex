\subsubsection{Natural deduction for propositional formulas}
\inlinedefinition[Sequent] Is an assertion of form $A_1, \ldots, A_n \vdash A$, with $A, A_1, \ldots, A_n$ being propositional formulas.

\inlineintuition $A$ follows from the $A_i$ and if the system is sound, the $A_i$ semantically entail $A$.

\inlinedefinition[Axiom] is the starting point (usually the leaves) of the derivation trees and are usually of the form
\[
    \begin{prooftree}
        \infer0[axiom]{\ldots, A, \ldots \vdash A}
    \end{prooftree}
\]
i.e. when coming up with a derivation tree for a \bi{proof}, we want to reach a leaf where $A$ is contained in $\Gamma$.

\inlinedefinition[Proof] of $A$ is a derivation tree with \texttt{root} $\vdash A$. If a deductive system is \textit{sound}, then $A$ is a tautology.

There are two kinds of rules, \bi{introduce} and \bi{eliminate} connectives. If you are confused about the order when applying them when coming up with a deduction tree,
they are oriented top-down, so e.g. the introduction rule is inverted when coming up with the deduction tree.

If all rules are sound (i.e. they preserve semantic entailment), then the logic is sound.
