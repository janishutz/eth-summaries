\definition \textbf{Grundraum} $\Omega\qquad$ \textbf{Elementarereignis} $\omega \in \Omega$

\definition \textbf{$\sigma$-Algebra} $\quad \mathcal{F} \subseteq \mathcal{P}(\Omega)\qquad$ \textbf{Ereignis} $A \in \mathcal{F}$

\begin{tabular}{lll}
    (i)     & $\Omega \in \mathcal{F}$ \\
    (ii)    & $A \in \mathcal{F}$               & $\implies A^\comp \in \mathcal{F}$ \\
    (iii)   & $A_1,\cdots,A_n \in \mathcal{F}$  & $\implies \displaystyle\underset{i \leq n}{\bigcup} A_i \in \mathcal{F}$
\end{tabular}

\lemma \textbf{Abgeschlossenheit} der $\sigma$-Algebra $\F$

\begin{tabular}{ll}
    (i)     & $\emptyset \in \F$ \\
    (ii)    & $A_1,\cdots,A_n \in \F \implies \displaystyle\overunderset{\infty}{i=1}{\bigcap} A_i \in \F$ \\
    (iii)   & $A, B \in \F \implies A \cup B \in \F$ \\
    (iv)    & $A, B \in \F \implies A \cap B \in \F$
\end{tabular}

\definition \textbf{Wahrscheinlichkeitsmass} auf $(\Omega, \F): \P$
$$
    \P: \F \to [0,1] \qquad \text{s.d.} \qquad A \mapsto \P[A]
$$
\begin{tabular}{ll}
    (i)     & $\P[\Omega] = 1$ \\
    (ii)    & $\P[A] = \displaystyle\sum_{i=1}^{\infty} A_i \iff A = \overunderset{\infty}{i=1}{\bigcup}A_i \quad \text{s.d.} \quad \overunderset{\infty}{i=1}{\bigcap}A_i = \emptyset$
\end{tabular}

\lemma \textbf{Eigenschaften} von $\P$

\begin{tabular}{ll}
    (i)     & $\P[\emptyset] = 0$ \\
    (ii)    & $\displaystyle\overunderset{k}{i=1}{\bigcap} A_i = \emptyset \implies \P\Biggl[ \overunderset{k}{i=1}{\bigcup} A_i \Biggr] = \sum_{i=1}^{k} \P[A_i]$ \\
    (iii)   & $\P[A^\comp] = 1 - \P[A]$ \\
    (iv)    & $\P[A \cup B] = \P[A] + \P[B] - \P[A \cap B]$
\end{tabular}

\definition \textbf{Wahrscheinlichkeitsraum} $(\Omega, \F, \P)$
\subtext{$A \in \mathcal{F}, \quad \omega \in \Omega$}

\begin{tabular}{lll}
    $A$ tritt ein           &$\iffdef$& $\omega \in A$ \\
    $A$ tritt nicht ein    &$\iffdef$& $\omega \notin A$ 
\end{tabular}
\footnotesize
\begin{tabular}{ll}
    (i)     & $\emptyset$ tritt nie ein \\
    (ii)    & $\Omega$ tritt immer ein
\end{tabular}
\normalsize

\newpage

\definition \textbf{Laplace Modell} $(\Omega, \F, \P)$
\subtext{$\Omega$ endlich.}

\begin{tabular}{ll}
    (i)     & $\F = \mathcal{P}(\Omega)$ \\
    (ii)    & $\forall A \in \F:\quad \P[A] = \displaystyle\frac{|A|}{|\Omega|}$
\end{tabular}

\lemma \textbf{Nützliche Ungleichungen}

\begin{tabular}{lll}
    (i)     & $A \subseteq B \implies \P[A] \leq \P[B]$     & \subtext{(Monotonie)} \\
    (ii)    & $\P \Biggl[ \displaystyle\overunderset{\infty}{i=1}{\bigcup} A_i \Biggr] \leq \sum_{\infty}^{i=1}\P[A_i]$ & \subtext{(Union Bound)}
\end{tabular}

\subtext{$A_1,A_2,\ldots$ müssen \textit{nicht} disjunkt sein.}

\lemma \textbf{Stetigkeit} von $\P$ gegen $\infty$\\
\begin{tabular}{ll}
    (i)     & $\forall n: A_n \subseteq A_{n+1} \implies \limn \P[A_n] = \P \Biggl[ \displaystyle\overunderset{\infty}{n=1}{\bigcup} A_n \Biggr]$ \\
    (ii)    & $\forall n: B_n \supseteq B_{n+1} \implies \limn \P[B_n] = \P \Biggl[ \displaystyle\overunderset{\infty}{n=1}{\bigcap} B_n \Biggr]$
\end{tabular}

\subtext{$(A_n), (B_n)$ sind monotone Folgen von Ereignissen}

\definition \textbf{Bedingte Wahrscheinlichkeit}
$$
    \P\bigl[A\ \big|\ B\bigr] := \frac{\P[A \cap B]}{\P[B]}
$$
\subtext{$A,B \in \F,\quad \P[B] > 0$}

\begin{footnotesize}
    \lemma $\P\bigl[ A \big| A \bigr] = 1$
    \color{gray}
    $\qquad\P[A] > 0$
    \color{black}
\end{footnotesize}

\lemma \textbf{Totale Wahrscheinlichkeit}
$$
    \forall A \in \F:\quad \P[A] = \sum_{i=1}^{n}\P\bigl[ A \big| B_i \bigr] \cdot \P[B_i]
$$
\subtext{$B_1,\cdots,B_n$ sind eine Partition von $\Omega$, $\P[B_i] > 0$.}

\lemma \textbf{Bayes}
$$
    \forall i = 1,\cdots,n:\quad \P\bigl[ B_i \big| A \bigr] = \frac{\P\bigl[ A \big| B_i \bigr] \cdot \P[B_i]}{\sum_{j=1}^{n}\P\bigl[ A \big| B_j \bigr] \cdot \P[B_j]}
$$
\subtext{$B_1,\cdots,B_n$ sind eine Partition von $\Omega$, $\P[B_i] > 0$, $\P[A] > 0$.}

\newpage

\definition \textbf{Unabhängigkeit}
$$
    A, B \text{ unabhängig } \iffdef \P[A \cap B] = \P[A] \cdot \P[B]
$$

\lemma \textbf{Äquivalente Aussagen zur Unabhängigkeit}

\begin{tabular}{lll}
    (i)     & $\P[A \cap B] = \P[A] \cdot \P[B]$ & \subtext{(Defintion)}                 \\
    (ii)    & $\P[A | B] = \P[A]$                & \subtext{($B$ kein Einfluss auf $A$)} \\
    (iii)   & $\P[B | A] = \P[B]$                & \subtext{($A$ kein Einfluss auf $B$)} \\
\end{tabular}

\subtext{$A, B \in \F,\quad \P[A], \P[B] > 0$}

\definition \textbf{Unabhängigkeit} für Ereignissmengen
$$
    (A_i)_{i \in I} \text{ unabhängig } \iffdef \forall J \subseteq I: \P \Biggl[ \underset{j \in J}{\bigcap} A_j \Biggr] = \prod_{j \in J} \P[A_j]
$$
\subtext{$I$ ist eine Indexmenge. Dies muss für \textit{alle} $J \subseteq I$ (endlich) gelten.}