\subsection{Eigenschaften Wahrscheinlichkeitsmasse}
\shorttheorem $\P$ Wahrscheinlichkeitsmass auf $(\Omega, \cF)$, $A$ Ereignis:
\begin{enumerate}[label=\textbf{E\arabic*.}]
    \item Es gilt $\P[\varnothing] = 0$
    \item \bi{Additivität} $k \geq 1$, $A_1, \ldots, A_k$ paarw. disj. Ereignisse:\\
          $\P[A_1 \cup \dots \cup A_k] = \P[A_1] + \dots + \P[A_k]$
    \item $\P[A^C] = 1 - \P[A]$
    \item $B$ Ereignis, dann $\P[A \cup B] = \P[A] + \P[B] - \P[A \cap B]$
\end{enumerate}

\newpage
\subsubsection{Nützliche Ungleichungen}
\shorttheorem[Monot.] $A, B \in \cF$, dann $A \subseteq B \Rightarrow \P[A] \leq \P[B]$

\shorttheorem[Union Bound] Für $A_1, A_2, \ldots$ (mögl. disj.) gilt:
$\P\left[ \bigcup_{i = 1}^\infty A_i \right] \leq \sum_{i = 1}^{\infty} \P[A_i]$.
Auch für endl. n.-leere Ereignisse
\stepLabelNumber{combined}


\subsubsection{Anwendungen der Ungleichungen}
Sie sind nützlich für schwer zu berechnende W.

\shorttheorem $(A_n)$ mit $A_n \subseteq A_{n + 1}$ (mon. wachsend). Dann:

{\centering $\limni P[A_n] = \P\left[ \bigcup_{n = 1}^\8 A_n \right]$\\}

und für $(B_n)$ mit $B_n \supseteq B_{n + 1}$ (mon. fallend) gilt:\\
$\limni P[B_n] = \P\left[ \bigcap_{n = 1}^\8 B_n \right]$

\shortremark Mit Monotonie: $\P[A_n] \leq \P[A_{n + 1}]$ und\\
$\P[B_n] \geq \P[B_{n + 1}]$. Grenzwerte oben wohldefiniert.
