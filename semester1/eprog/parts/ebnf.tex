\section{EBNF}\label{ebnf}
= Extended Backus Naur Form / Extended Backus Normal Form

\begin{itemize}
    \item Can be used to describe automatic tests
\end{itemize}

\subsection{Rules}
\begin{definition}[]{EBNF-Rule}
    A rule is described as follows

    \begin{verbatim}
LHS <- RHS
    \end{verbatim}

    where LHS is the left hand side, which gives the name of the rule. That name is usually written in italic or inside \texttt{\textless{}...\textgreater{}}. The RHS is the description of the rule using elements, which are literals, names of other EBNF rules defined previously or after or a combination of the four control forms
\end{definition}

A character as EBNF rule would be something like

\begin{verbatim}
<digit_zero> <- 0
\end{verbatim}

Another EBNF rule as RHS would be something like:

\begin{verbatim}
<number_zero> <- <digit_zero>
\end{verbatim}

We have to also define an entry rule, which, if nothing else is specified, is the last rule. It marks the entry point from where the checks are run.

\subsection{Control forms}

\subsubsection{Sequence}
\begin{definition}[]{Sequence}
    Concatenation of the elements. Noted with a space between the elements.

    \begin{verbatim}
<rule> <- E1 E2 E3
    \end{verbatim}
\end{definition}

\subsubsection{Decision}
This control form has two options, namely selection and option.

\begin{definition}[]{Selection}
    We can choose \textit{exactly} one of the elements listed with pipe characters (called stroke).

    \begin{verbatim}
<rule> <- E1 | E2 | E3
    \end{verbatim}
\end{definition}

\begin{definition}[]{Option}
    We can either choose to use the element or not. Noted with square brackets. $\epsilon$ is the empty element, which is returned if the option is not chosen.

    \begin{verbatim}
<rule> <- [E]
    \end{verbatim}
\end{definition}

\subsubsection{Repetition}
\begin{definition}[]{Repetition}
    Repeat an element $n \in \N_0$ times, so repeating it $0$ times is also valid. Noted using curly brackets.

    \begin{verbatim}
<rule> <- {E}
    \end{verbatim}
\end{definition}


\subsubsection{Recursion}
\begin{definition}[]{Recursion}
    Recursively repeat an element, by adding it to the RHS of itself

    \begin{verbatim}
<rule> <- <digit> <rule>
    \end{verbatim}
\end{definition}

We can have direct or indirect recursion, where indirect recursion is when the recursion is not directly on the rule itself, but through elements of the RHS of the rule.

\subsection{Parenthesis}
There are precedences, but for where they have not been covered, parenthesis are used to clarify what is meant. Always put them there, if it is not 100\% clear what is meant.

\fhlc{Aquamarine}{Covered precedences:}
Sequence binds weaker than everything else.


\subsection{Special caracters}
We use boxes around them if we want to use them as literals, not as EBNF characters, i.e. if we want to write \texttt{\{\}}, then we write \fbox{\{\}}, same goes for space, which we can also write using the space symbol, \begin{Large}\textvisiblespace\end{Large}

\newpage
\subsection{Nomenclature \& Notation}
\begin{terms}[]{EBNF}
    \begin{itemize}
        \item \textbf{\textit{Legal}}: A word is considered legal if there is a derivation of the character sequence
        \item \textbf{\textit{Derivation}}: Sequence of derivation steps
        \item \textbf{\textit{Derivation step}}: 
            \begin{itemize}
                \item Replace rules with their definition (RHS)
                \item \textit{Selection}: Select element in a selection
                \item \textit{Option}: Choose whether to select an optional element
                \item \textit{Repetition}: Decide on the number of repetitions of the element
            \end{itemize}
    \end{itemize}
\end{terms}


\subsubsection{Notation of derivation}
\begin{notation}[]{Derivation}
    \begin{itemize}
        \item Derivation table
            \begin{itemize}
                \item First row is the entry rule
                \item Last row is a sequence of characters
                \item Transition between rows is a derivation step
            \end{itemize}
        \item Derivation tree
            \begin{itemize}
                \item Root is the name of the entry rule
                \item Leafs are characters
                \item Connections are the derivation steps
            \end{itemize}
    \end{itemize}
\end{notation}

\subsubsection{Syntax vs. Semantics}
Syntax = Structure / Form (Grammar), Semantics = Meaning / Interpretation

\subsection{Graphical noatation of EBNF}
With syntax graph. We can replace names in the syntax graph with another graph.

\fhlc{ForestGreen}{Sequence}: \texttt{A B C} $\Longrightarrow$ $\rightarrow$ A - B - C $\rightarrow$

\fhlc{ForestGreen}{Selection}: \texttt{A | B | C} $\Longrightarrow$ $\rightarrow$ Parallel circuit-like notation $\rightarrow$

\fhlc{ForestGreen}{Option}: \texttt{[A]} $\Longrightarrow$ $\rightarrow$ Parallel circuit-like notation, two options, top is the element, bottom is empty $\rightarrow$

\fhlc{ForestGreen}{Repetition}: \texttt{\{A\}} $\Longrightarrow$ $\rightarrow$ Similar to option, but there is a loop back at the top element (allowing repetition) $\rightarrow$
