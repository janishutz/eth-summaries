



\newsection
\section{\tr{Table of derivatives and Antiderivatives}{Tabelle von Ableitungen und Stammfunktionen}}
\begin{multicols}{2}
    \begin{tables}{lll}{\tr{Antiderivative}{Stammfunktion}                    & \tr{Function}{Funktion}                       & \tr{Derivative}{Ableitung}}
              $\displaystyle \frac{x^{n + 1}}{n + 1}$                     & $x^n$                                         & $n \cdot x^{n - 1}$                        \\
              $\ln|x|$                                                    & $\displaystyle \frac{1}{x} = x^{-1}$          & $\displaystyle -x^{-2} = -\frac{1}{x^2}$   \\[0.2cm]
              $\frac{2}{3} x^{\frac{3}{2}}$                               & $\displaystyle \sqrt{x} = x^{\frac{1}{2}}$    & $\displaystyle \frac{1}{2 \cdot \sqrt{x}}$ \\[0.3cm]
              $\frac{n}{n + 1} x^{\frac{1}{n} + 1}$                       & $\displaystyle \sqrt[n]{x} = x^{\frac{1}{n}}$ & $\frac{1}{n} x^{\frac{1}{n} - 1}$          \\[0.3cm]
              \hline                                                                                                                                                   \\[-0.2cm]
              $e^x$                                                       & $e^x$                                         & $e^x$                                      \\
              $\exp(x)$                                                   & $\exp(x)$                                     & $\exp(x)$                                  \\
              $\frac{1}{a \cdot (n + 1)}(ax + b)^{n + 1}$                 & $(ax + b)^n$                                  & $n\cdot (ax + b)^{n - 1} \cdot a$          \\
              $x \cdot (\ln|x| - 1)$                                      & $\ln(x)$                                      & $\frac{1}{x} = x^{-1}$                     \\
              $\displaystyle \frac{1}{\ln(a)}\cdot a^x$                   & $a^x$                                         & $a^x \cdot \ln(a)$                         \\
              $\frac{x}{\ln(a)} \cdot (\ln|x| - 1)$                       & $\log_a|x|$                                   & $\displaystyle \frac{1}{x \cdot \ln(a)}$   \\[0.3cm]
              \hline                                                                                                                                                   \\[-0.2cm]
              $-\cos(x)$                                                  & $\sin(x)$                                     & $\cos(x)$                                  \\
              $\sin(x)$                                                   & $\cos(x)$                                     & $-\sin(x)$                                 \\
              $-\ln|\cos(x)|$                                             & $\tan(x)$                                     & $\displaystyle \frac{1}{\cos^2(x)}$        \\[0.3cm]
              $x \cdot \arcsin(x) + \sqrt{1 - x^2}$                       & $\arcsin(x)$                                  & $\displaystyle\frac{1}{\sqrt{1 - x^2}}$    \\
              $x \cdot \arccos(x) - \sqrt{1 - x^2}$                       & $\arccos(x)$                                  & $\displaystyle -\frac{1}{\sqrt{1 - x^2}}$  \\
              $\displaystyle x \cdot \arctan(x) - \frac{\ln(x^2 + 1)}{2}$ & $\arctan(x)$                                  & $\displaystyle \frac{1}{x^2 + 1}$          \\[0.2cm]
              $\ln|\sin(x)|$                                              & $\cot(x)$                                     & $\displaystyle -\frac{1}{\sin^2(x)}$       \\
              $\cosh(x)$                                                  & $\sinh(x)$                                    & $\cosh(x)$                                 \\
              $\sinh(x)$                                                  & $\cosh(x)$                                    & $\sinh(x)$                                 \\
              $\ln|\cosh(x)|$                                             & $\tanh(x)$                                    & $\displaystyle \frac{1}{\cosh^2(x)}$       \\
                                                                          & $\arcsinh(x)$                                 & $\frac{1}{\sqrt{1 + x^2}}$                 \\
                                                                          & $\arccosh(x)$                                 & $\frac{1}{\sqrt{x^2 - 1}}$                 \\
                                                                          & $\arctanh(x)$                                 & $\frac{1}{1 - x^2}$                        \\
    \end{tables}
    \shade{teal}{\tr{Logarithms}{Logarithmen}}\\
    \textit{(\tr{Change of base}{Basiswechsel})} $\log_a(x) = \frac{\ln(x)}{\ln(a)}$
    \textit{(\tr{Powers}{Potenzen})} $\log_a(x^y) = y\log_a(x)$
    \textit{(Div, Mul)} $\log_a(x \cdot (\div) y) = \log_a(x) +(-) \log_a(y)$\\
    $\log_a(1) = 0 \smallhspace \forall a \in \N$

    \shade{teal}{\tr{Integration by parts}{Partielle Integration}}
    \tr{Should we get unavoidable cycle, where we have to integrate the same thing again, we may simply add the integral to both sides, and we thus have $2$ times the integral on the left side and then finish the integration by parts on the right hand side and in the end divide by the factor up front to get the result}
    {Sollte sich ein unvermeidbarer Zyklus, wo wir immer wieder denselben Integral erhalten, bilden, können wir einfach das Integral zu beiden Seiten addieren und erhalten so $2$ mal das Integral auf der linken Seite und können dann die partielle Integration auf der rechten Seite abschliessen und schliesslich durch den Faktor auf der linken Seite dividieren, um das Resultat zu erhalten}.

    \shade{teal}{\tr{Inverse hyperbolic functions}{Umkehrfunktion der Hyperbelfunktionen}}
    \vspace{-0.5pc}
    \begin{itemize}
        \item $\arcsinh(x) = \ln \left( x + \sqrt{x^2 + 1} \right)$
        \item $\arccosh(x) = \ln \left( x + \sqrt{x^2 - 1} \right)$
        \item $\arctanh(x) = \frac{1}{2} \ln \left( \frac{1 + x}{1 - x} \right)$
    \end{itemize}

    \shade{teal}{\tr{Complement trick}{Komplement-Trick}}
    $\sqrt{ax + b} - \sqrt{cx + d} = \frac{ax + b - (cx + d)}{\sqrt{ax +b} + \sqrt{cx + d}}$


    \shade{teal}{\tr{Values of trigonometric functions}{Werte der trigonometrischen Funktionen}}
    \begin{tables}{ccccc}{° & rad              & $\sin(\xi)$          & $\cos(\xi)$           & $\tan(\xi)$}
              0°        & $0$              & $0$                  & $1$                   & $1$                   \\
              \hline
              30°       & $\frac{\pi}{6}$  & $\frac{1}{2}$        & $\frac{\sqrt{3}}{2}$  & $\frac{\sqrt{3}}{2}$  \\
              \hline
              45°       & $\frac{\pi}{4}$  & $\frac{\sqrt{2}}{2}$ & $\frac{\sqrt{2}}{2}$  & $1$                   \\
              \hline
              60°       & $\frac{\pi}{3}$  & $\frac{\sqrt{3}}{3}$ & $\frac{1}{2}$         & $\sqrt{3}$            \\
              \hline
              90°       & $\frac{\pi}{2}$  & $1$                  & $0$                   & $\varnothing$         \\
              \hline
              120°      & $\frac{2\pi}{3}$ & $\frac{\sqrt{3}}{2}$ & $-\frac{1}{2}$        & $-\sqrt{3}$           \\
              \hline
              135°      & $\frac{3\pi}{4}$ & $\frac{\sqrt{2}}{2}$ & $-\frac{\sqrt{2}}{2}$ & $-1$                  \\
              \hline
              150°      & $\frac{5\pi}{6}$ & $\frac{1}{2}$        & $-\frac{\sqrt{3}}{2}$ & $-\frac{\sqrt{3}}{2}$ \\
              \hline
              180°      & $\pi$            & $0$                  & $-1$                  & $0$                   \\
    \end{tables}
\end{multicols}

\vspace{3mm}
\hrule

\begin{multicols}{2}
    \shade{teal}{\tr{Trigonometrie}{Trigonometrie}} 
    $\cot(\xi) = \displaystyle\frac{\cos(\xi)}{\sin(\xi)}, \tan(\xi) = \frac{\sin(\xi)}{\cos(\xi)}$

    $\sinh(x) := \frac{e^x - e^{-x}}{2} : \R \rightarrow \R$,
    $\cosh(x) := \frac{e^x + e^{-x}}{2} : \R \rightarrow [1, \infty]$,
    $\cosh(x) := \frac{\sinh(x)}{\cosh(x)} = \frac{e^x - e^{-x}}{e^x + e^{-x}} : \R \rightarrow [-1, 1]$

    \begin{enumerate}
        \item $\cos(x) = \cos(-x)$ \trand $\sin(-x) = -\sin(x)$
        \item $\cos(\pi - x) = -\cos(x)$ \trand $\sin(\pi - x) \sin(x)$
        \item $\sin(x + w) = \sin(x) \cos(w) + \cos(x) \sin(w)$
        \item $\cos(x + w) = \cos(x) \cos(w) - \sin(x) \sin(w)$
        \item $\cos(x)^2 + \sin(x)^2 = 1$
        \item $\sin(2x) = 2 \sin(x) \cos(x)$
        \item $\cos(2x) = \cos(x)^2 - \sin(x)^2$
    \end{enumerate}
\end{multicols}

\hrule

\shade{teal}{\tr{Further derivatives}{Weitere Ableitungen}} 
\begin{multicols}{2}
    \begin{tables}{cc}{$F(x)$                                                            & $f(x)$}
              $\frac{1}{a} \ln|ax + b|$                                              & $\frac{1}{ax + b}$                          \\
              $\frac{ax}{c} - \frac{ad - bc}{c^2} \ln|cx + d|$                       & $\frac{a (cx + d) - c(ax + b)}{(cx + d)^2}$ \\
              $\frac{x}{2} f(x) + \frac{a^2}{2} \ln|x + f(x)|$                       & $\sqrt{a^2 + x^2}$                          \\
              $\frac{x}{2} f(x) - \frac{a^2}{2} \arcsin\left( \frac{x}{|a|} \right)$ & $\sqrt{a^2 - x^2}$                          \\
              $\frac{x}{2} f(x) - \frac{a^2}{2} \ln|x + f(x)|$                       & $\sqrt{x^2 - a^2}$                          \\
              $\ln(x + \sqrt{x^2 \pm a^2})$ & $\frac{1}{\sqrt{x^2 \pm a^2}}$\\
              $\arcsin \left( \frac{x}{|a|} \right)$ & $\frac{1}{\sqrt{x^2 - a^2}}$\\
              $\frac{1}{a}\arctan \left( \frac{x}{|a|} \right)$ & $\frac{1}{a^2 - x^2}$\\
    \end{tables}
    \begin{tables}{cc}{$F(x)$                                                            & $f(x)$}
        $-\frac{1}{a} \cos(ax + b)$ & $\sin(ax + b)$\\
        $\frac{1}{a} \sin(ax + b)$ & $\cos(ax + b)$\\[1mm]
        \hline
        $x^x$ & $x^x \cdot (1 + \ln|x|)$\\
        $(x^x)^x$ & $(x^x)^x \cdot (x + 2x\ln|x|)$\\
        $x^{(x^x)}$ & $x^{(x^x)} \cdot (x^{x - 1} + \ln|x| \cdot x^x (1 + \ln|x|))$\\
        \hline\\[-3mm]
        $\frac{1}{2}(x - \frac{1}{2} \sin(2x))$ & $\sin(x)^2$\\[1mm]
        $\frac{1}{2}(x + \frac{1}{2} \sin(2x))$ & $\cos(x)^2$\\
    \end{tables}
\end{multicols}
