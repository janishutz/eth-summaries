\newpage
\fhlc{Cyan}{Bipartite Matching as Flow-Problem}

We can use the concepts of flows to determine matchings in bipartite graphs.

Let $G = (V, E)$ be a bipartite graph, i.e. $\exists$ Partition $(\mathcal{U}, \mathcal{W})$ of $V$ such that $E = \{\{u, w\} \divides u \in \mathcal{U}, w \in \mathcal{W}\}$.
We construct a network $N = (V \dot{\cup} \{s, t\}, \mathcal{A}, c, s, t)$, i.e. to the vertices of $G$, we added a source and a target.
The capacity function is $c(e) = 1$.
We copy the edges from $G$, having all edges be directed ones from vertices in $\mathcal{U}$ to ones in $\mathcal{W}$.
We add edges from the source $s$ to every vertex in $\mathcal{U}$ and from every vertex in $\mathcal{W}$ to the target $t$.

\begin{lemma}[]{Bipartite Matching - Max-Flow}
    The maximum matching in a bipartite graph $G$ is equal to the maximum flow in the network $N$ as described above
\end{lemma}


\fhlc{Cyan}{Edge- and Vertex-disjoint paths}

We can determine the degree of the connectivity of a graph (i.e. the number of vertices / edges that have to be removed that the graph becomes disconnected) by determining how many edge- or vertex-disjoint paths exist between two vertices.
Again, using max-flow, we can solve this problem as follows:

Given an undirected graph $G = (V, E)$ and two vertices $u, v \in V : u \neq v$, we need to define our network $N = (\mathcal{V}, \mathcal{A}, c, s, t)$:
\begin{itemize}
    \item Copy the vertex set $V$ and union it with two new vertices $s$ and $t$ and we thus have $\mathcal{V} = V \cup \{s, t\}$
    \item Add two edges for each undirected edge in $G$, i.e. $\mathcal{A} = E \cup E'$ where $E'$ is has all edge directions reversed
    \item We define the capacity function as $c(e) = 1$
    \item We add two edges $(s, u)$ and $(v, t)$ and set the capacity of these edges to $|V|$. These are the two vertices between which we evaluate the edge-disjoint paths
\end{itemize}

If instead of edge-disjoint paths, we want to find \textit{vertex}-disjoint paths, we simply replace each vertex $x \in V\backslash\{u, v\}$ by $x_{\text{in}}$ and $x_{\text{out}}$ and connect all input-edges to $x_{\text{in}}$ and all output-edges of $x$ to $x_{\text{out}}$


\fhlc{Cyan}{Image segmentation}

We can also use cuts to solve image segmentation, i.e. to split background from foreground. We can translate an image to an undirected graph, since every pixel has four neighbours.
Whatever the pixel values mean in the end, we assume we can deduce two non-negative numbers $\alpha_p$ and $\beta_p$ denoting the probability that $p$ is in the foreground or background respectively.

Since this topic looks to not be too relevant for the exam, a full explanation of this topic can be found in the script on page 186-189


\fhlc{Cyan}{Flows and convex sets}

From the definition of flows we have seen, there is always \textit{at least} one flow, the flow \textbf{0}.

\begin{lemma}[]{Flows}
    Let $f_0$ and $f_1$ be flows in a network $N$ and let $\lambda \in \R : 0 < \lambda < 1$, then the flow $f_{\lambda}$ given by
    \begin{align*}
        \forall e \in \mathcal{A} : f_{\lambda}(e) := (1 - \lambda)f_0(e) + \lambda f_1(e)
    \end{align*}
    is also a flow in $N$. We have
    \begin{align*}
        \text{val}(f_{\lambda}) = (1 - \lambda) \cdot \text{val}(f_0) + \lambda \cdot \text{val}(f_1)
    \end{align*}
\end{lemma}

\begin{corollary}[]{Number of flows in networks}
    \begin{enumerate}[label=(\roman*)]
        \item A network $N$ has either exactly \textit{one} flow (the flow \textbf{0}) or infinitely many flows
        \item A network $N$ has either exactly \textit{one} maximum flow or infinitely many maximum flows
    \end{enumerate}
\end{corollary}


\shade{ForestGreen}{Convex sets}
We define a function $f: \mathcal{A} \rightarrow \R$ that induces a vector $v_f := (f(e_1), f(e_2), \ldots, f(e_m)) \in \R^m$ whereas $e_1, \ldots, e_m$ is an ordering of the vertices of $\mathcal{A}$ where $m = |\mathcal{A}|$.
We can interpret the set of (maximum) flows as a subset of $\R^m$

\begin{definition}[]{Convex set}
    Let $m \in \N$
    \begin{enumerate}[label=(\roman*)]
        \item For $v_0, v_1 \in \R^m$ let $\overline{v_0v_1}:=\{(1 - \lambda v_0) + \lambda v_1 \divides \lambda \in \R, 0 \leq \lambda \leq 1\}$ be the \textit{line segment} connecting $v_0$ and $v_1$
        \item A set $\mathcal{C} \subseteq \R^m$ is called \textit{convex} if for all $v_0, v_1 \in \mathcal{C}$ the whole line segment $\overline{v_0 v_1}$ is in $\mathcal{C}$
    \end{enumerate}
\end{definition}
\textbf{Examples:} Spheres or convex Polytopes (e.g. dice or tetrahedra in $\R^3$)
\begin{theorem}[]{Convex sets}
    The set of flows of a network with $m$ edges, interpreted as vectors is a convex subset of $\R^m$. The set of all maximum flows equally forms a convex subset of $\R^m$
\end{theorem}


\newpage
\subsubsection{Min-Cuts in graphs}
In the following section we use \textit{multigraphs}.
\begin{recall}[]{Multigraph}
    A multigraph is an undirected, unweighted and acyclic graph $G = (V, E)$, where multiple edges are allowed to exist between the same pair of vertices.

    \textit{(Instead of multiple edges, we could also allow weighted edges, but the algorithms and concepts presented here are more easily understandable using multiple edges)}
\end{recall}

\fhlc{Cyan}{Min-Cut Problem}

We define $\mu(G)$ to be the cardinality of the \textit{min-cut} (this is the problem).
This problem is similar to the min-cut problem for flows, only that we have a multigraph now. We can however replace multiple edges with a single, weighted edge, allowing us to use the algorithms discussed above.
Since we need to compute $(n - 1)$ $s$-$t$-cuts, our total time complexity is \tco{n^4 \log(n)}, since we can compute $s$-$t$-cuts in \tco{n^3\log(n)} = \tco{n\cdot m\log(n)}



\fhlc{Cyan}{Edge contraction}

Let $e = \{u, v\}$ be an edge of our usual multigraph $G$.
The \textit{contraction of} $e$ replaces the two vertices $u$ and $v$ with a single vertex denoted $x_{u, v}$, which is incident to all edges any of the two vertices it replaced were incident to, apart from the ones between the two vertices $u$ and $v$.
We call the new graph $G/e$ and $\deg_{G/e}(x_{u, v}) = \deg_G(u) + \deg_G(v) - 2k$ where $k$ denotes the number of edges between $u$ and $v$.

Of note is that there is a bijection: $\text{Edges in G without the ones between $u$ and } v \leftrightarrow \text{Edges in $G/e$}$

\begin{lemma}[]{Edge contraction}
    Let $G$ be a graph and $e$ be an edge of $G$. Then we have that $\mu(G/e) \geq \mu(G)$ and we have equality if $G$ contains a min-cut $\mathcal{C}$ with $e \notin \mathcal{C}$.
\end{lemma}


\fhlc{Cyan}{Random edge contraction}

\begin{algorithm}
    \caption{Random Cut where $G$ is a connected Multigraph}
    \begin{algorithmic}[1]
        \Procedure{Cut}{$G$}
            \While{$|V(G)| > 2$} \Comment{Vertices of $G$}
                \State $e \gets$ uniformly random edge in $G$
                \State $G \gets G/e$
            \EndWhile
            \State \Return Size of a unique cut of $G$
        \EndProcedure
    \end{algorithmic}
\end{algorithm}

If we assume that we can perform edge contraction in \tco{n} and we can choose a uniformly random edge in $G$ in \tco{n} as well, it is evident that we can compute \textsc{Cut}($G$) in \tco{n^2}

\begin{lemma}[]{Random edge contraction}
    If $e$ is uniformly randomly chosen from the edges of multigraph $G$, then we have
    \begin{align*}
        \Pr[\mu(G) = \mu(G/e)] \geq 1 - \frac{2}{n}
    \end{align*}
\end{lemma}

\newpage
\begin{lemma}[]{Correctness of \textsc{Cut}$(G)$}
    To evaluate the correctness of \textsc{Cut}$(G)$, we define
    \begin{align*}
        \hat{p}(G) := \text{Probability that \textsc{Cut}($G$) returns the value } \mu(G)
    \end{align*}
    and let
    \begin{align*}
        \hat{p}(n) := \inf_{G=(V, E), |V| = n}\hat{p}(G)
    \end{align*}
    Then, for all $n \geq 3$ we have
    \begin{align*}
        \hat{p} \geq \left( 1 - \frac{2}{n} \right) \cdot \hat{p}(n - 1)
    \end{align*}
\end{lemma}

\begin{lemma}[]{Probability of Correctness of \textsc{Cut}$(G)$}
    For all $n \geq 2$ we have $\displaystyle \hat{p}(n) \geq \frac{2}{n(n - 1)} = \frac{1}{{n \choose 2}}$
\end{lemma}
Thus, we repeat the algorithm \textsc{Cut}$(G)$ $\lambda {n \choose 2}$ times for $\lambda > 0$ and we return the smallest value we got.
\begin{theorem}[]{\textsc{Cut}$(G)$}
    For the algorithm that runs \textsc{Cut}$(G)$ $\lambda{n \choose 2}$ times we have the following properties:
    \begin{enumerate}[label=(\arabic*)]
        \item Time complexity: \tco{\lambda n^4}
        \item The smallest found value is with probability at least $1 - e^{-\lambda}$ equal to $\mu(G)$
    \end{enumerate}
\end{theorem}
If we choose $\lambda = \ln(n)$, we have time complexity \tco{n^4 \ln(n)} with error probability \textit{at most} $\frac{1}{n}$

Of note is that for low $n$, it will be worth it to simply deterministically determine the min-cut
