\newpage
\subsection{Colourings}
Good examples for problems that can be solved using colourings are the channel picking of wireless devices, for compilers to pick registers and for creating timetables and exam schedules.

\begin{definition}[]{Colouring}
    A \textit{(vertex-)colouring} of a graph $G$ with $k$ colours is an image $c: V \rightarrow [k]$ such that
    \begin{align*}
        c(u) \neq c(v) \smallhspace \forall \{u, v\} \in E
    \end{align*}
    The chromatic number $\mathscr{X}(G)$ is the minimum number of colours that are required to colour the graph $G$
\end{definition}
\stepcounter{all}
Graphs with chromatic number $k$ are also called $k$-\textit{partite}, where from the naming of a \textit{bipartite} graph comes, since we can \textit{partition} the graph into $k$ separate sets
\begin{theorem}[]{Bipartite graph}
    A graph $G = (V, E)$ is bipartite if and only if it does not contain a cycle of uneven length as sub-graph
\end{theorem}
On political maps, two neighbouring countries are coloured in different colours.
We assume that the territories of every country is connected and that all countries that only touch at one point can be coloured with the same colour.
\begin{theorem}[]{Four colour theorem}
    Every land map can be coloured in four colours
\end{theorem}
Again, the problem of determining if the chromatic number of a graph is smaller than a value $t$ is another $\mathcal{N}\mathcal{P}$-complete problem. We thus have to again proceed with an approximation.

The following algorithm correctly computes \textit{\textbf{a}} valid colouring.
\begin{theorem}[]{Greedy-Colouring Algorithm}
    For the number of colours $C(G)$ the \textsc{Greedy-Colouring} needs to colour the connected graph $G$ we have
    \begin{align*}
        \mathscr{X}(G) \leq C(G) \leq \Delta(G) + 1
    \end{align*}
    where $\Delta(G) := \max_{v\in V}\deg(v)$ is the maximum degree of a vertex in $G$. If the graph is stored as an adjacency list, the algorithm finds a colouring \tco{|E|}
\end{theorem}

\begin{algorithm}
    \caption{\textsc{Greedy-Colouring}$(G)$}
    \begin{algorithmic}[1]
        \State Choose an arbitrary order of vertices $V = \{v_1, \ldots, v_n\}$
        \State $c[v_1] \gets 1$
        \For{$i = 2, \ldots, n$}
            \State $c[v_i] \gets \min \{k \in \N \divides k \neq c(u) \smallhspace \forall u \in N(v_i) \cap \{v_1, \ldots, v_{i - 1}\}\}$ \Comment{Find minimum available colour}
        \EndFor
    \end{algorithmic}
\end{algorithm}

\begin{theorem}[]{Brook's theorem}
    Let $G$ be a connected graph that is neither complete nor a cycle of uneven length (uneven cycle), we have
    \begin{align*}
        \mathscr{X}(G) \leq \Delta(G)
    \end{align*}
    and there is an algorithm that colours the graph using $\Delta(G)$ colours in \tco{|E|}. Otherwise $\mathscr{X}(G) \leq \Delta(G) + 1$
\end{theorem}
Of note is that a graph with an even number of vertices and edges does not contain an uneven cycle, so for an incomplete graph with an even number of edges and vertices, we always have that $\mathscr{X}(G) \leq \Delta(G)$

\begin{theorem}[]{Maximum degree}
    Let $G$ be a graph and $k \in \N$ the number representing the maximum degree of any vertex of any induced subgraph of $G$. Then we have $\mathscr{X}(G) \leq k + 1$ and a $(k + 1)$-coloring can be found in \tco{|E|}
\end{theorem}

\begin{theorem}[]{Mycielski-Construction}
    For all $k \geq 2$ there exists a triangle-free graph $G_k$ with $\mathscr{X}(G_k) \geq k$
\end{theorem}
It is even possible to show that for all $k, l \geq 2$ there exists a graph $G_{k, l}$ such that said graph doesn't contain a cycle of length \textit{at most} $l$, but we still have $\mathscr{X}(G_{k, l}) \geq k$.

To conclude this section, one last problem:

We are given a graph $G$ which we are told has $\mathscr{X}(G) = 3$. This means, we know that there is exists an order of processing for the \textsc{Greedy-Colouring} algorithm that only uses three colours. We don't know the colours, but we can find an upper bound for the number of colours needed
\begin{theorem}[]{$3$-colourable graphs}
    Every $3$-colourable graph $G$ can be coloured in time \tco{|E|} using at most \tco{\sqrt{|V|}} colours
\end{theorem}
Since the graph has to be bipartite (because for each vertex $v$, its neighbours can only be coloured in $2$ other colours, because the graph can be $3$-coloured), we can use BFS and thus have linear time. The algorithm works as follows: We choose the vertices with the largest degree and apply three colours to them. For the ones of smaller degree, we apply Brook's theorem.
