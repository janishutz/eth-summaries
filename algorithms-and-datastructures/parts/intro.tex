\newsection
\section{Introduction}

\subsection{Sufficiency \& Necessity}
\begin{definition}[]{Sufficiency}
    A condition $P$ is called \textit{sufficient} for $Q$ if knowing $P$ is true is enough evidence to conclude that $Q$ is true.

    This is equivalent to saying $Q \Rightarrow P$.
\end{definition}

\begin{definition}[]{Necessity}
    A condition $P$ is called \textit{necessary} for $Q$ if $Q$ cannot occur unless $P$ is true, but doesn't imply that $Q$ is true, only that it is false if $P$ is false.

    This is equivalent to saying $P \Rightarrow Q$
\end{definition}

\subsection{Asymptotic Growth}
$f$ grows asymptotically slower than $g$ if $\displaystyle\lim_{m \rightarrow \infty} \frac{f(m)}{g(m)} = 0$.
We can remark that $f$ is upper-bounded by $g$, thus $f \leq$\tco{g} and we can say $g$ is lower bounded by $f$, thus $g \geq$ \tcl{f}.
If two functions grow equally fast asymptotically, \tct{f} $= g$


\subsection{Runtime evaluation}
Identify the basic operations (usually given by the task), then count how often they are called and express that as a function in $n$.
It is easier to note that in sum notation, then simplify that sum notation into a formula not containing any summation symbols.


% ────────────────────────────────────────────────────────────────────
\subsection{Tips for Converting Summation Notation into Summation-Free Notation}

\subsubsection{Identify the Pattern:}
\begin{itemize}
    \item Examine the summand.
    \item Look for patterns related to the index variable (usually $i$, $j$, etc.). Is it a linear function, a power of $i$, a combination?
\end{itemize}


\subsubsection{Arithmetic Series Formula}
If the summand is a simple arithmetic progression (e.g., $a + bi$ where $a$ and $b$ are constants), use the formula:
\[
    \sum_{i=m}^{n} (a + bi) = (n - m + 1)\left(a + b\frac{m + n}{2}\right)
\]


\subsubsection{Power Rule for Sums}
\begin{itemize}
    \item For sums involving powers of $i$, you can use the following pattern:
          \[
              \sum_{i=1}^{n} i^k = \frac{n^{k+1}}{k+1}
          \]
    \item Remember that this rule only applies when the index starts at 1.
\end{itemize}


\subsubsection{Telescoping Series}
Look for terms in consecutive elements of the summand that cancel out, leaving a simpler expression after expanding. This is particularly helpful for fractions and ratios.


\subsubsection{Geometric Series Formula}
For sums involving constant ratios (e.g., $a \cdot r^i$ where $r$ is the common ratio), use:
\[
    \sum_{i=0}^{n} a \cdot r^i = a \frac{1 - r^{n+1}}{1-r}
\]


\subsubsection{Gaussian Formula}
If $S$ is an arithmetic series with $n$ terms, then $S = \frac{n}{2} * (a + 1)$


\subsubsection{Examples}
The only other way (other than learning these tips) in which you are going to get better at this is by parctising.
Work through examples, starting with simpler ones and moving towards more complex expressions.


\fhlc{Aquamarine}{Example:}

Let's convert the summation: $\sum_{i=1}^{5} i$

\begin{enumerate}
    \item \textbf{Pattern:} The summand is simply $i$, which represents a linear arithmetic progression.

    \item \textbf{Arithmetic Series Formula:} Applying the formula with $a = 1$, $b = 1$, $m = 1$, and $n = 5$:
          \[
              \sum_{i=1}^{5} i = (5 - 1 + 1)\left(1 + 1 \cdot \frac{1 + 5}{2}\right) = 5 \cdot 3 = 15
          \]
\end{enumerate}

Therefore, the summation evaluates to $15$.

\subsection{Specific examples}
\begin{align*}
    \frac{n}{\log(n)} \geq \Omega(\sqrt{n}) \Leftrightarrow \sqrt{n} \leq \text{\tco{\frac{n}{\log(n)}}}
\end{align*}
